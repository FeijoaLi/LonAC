\documentclass{ctexart} % 子文件也要声明类型

% === 引入公共配置 ===
% 这样单独编译时,它就有颜色和代码高亮了
\usepackage{my_style} 

\begin{document} 
% 注意:主文件读取时,会忽略上面的所有内容,直接从这里开始读取

\section{[CSP-J 2025] 异或和 / xor} 

\subsection*{题目描述}
小 R 有一个长度为 $n$ 的非负整数序列 $a_1, a_2, \dots, a_n$。定义一个区间 $[l, r]$ ($1 \leq l \leq r \leq n$) 的权值为 $a_l, a_{l+1}, \dots, a_r$ 的二进制按位异或和,即 $a_l \oplus a_{l+1} \oplus \dots \oplus a_r$,其中 $\oplus$ 表示二进制按位异或。

小 X 给了小 R 一个非负整数 $k$。小 X 希望小 R 选择序列中尽可能多的不相交的区间,使得每个区间的权值均为 $k$。两个区间 $[l_1, r_1], [l_2, r_2]$ 相交当且仅当两个区间同时包含至少一个相同的下标,即存在 $1 \leq i \leq n$ 使得 $l_1 \leq i \leq r_1$ 且 $l_2 \leq i \leq r_2$。

例如,对于序列 $[2, 1, 0, 3]$,若 $k = 2$,则小 R 可以选择区间 $[1, 1]$ 和区间 $[2, 4]$,权值分别为 $2$ 和 $1 \oplus 0 \oplus 3 = 2$;若 $k = 3$,则小 R 可以选择区间 $[1, 2]$ 和区间 $[4, 4]$,权值分别为 $1 \oplus 2 = 3$ 和 $3$。

\subsection*{输入输出格式}
\textbf{输入:}第一行,两个非负整数 $n, k$,分别表示序列长度和给定的非负整数。\par
\qquad\quad 第二行,$n$ 个非负整数 $a_1, a_2, \dots, a_n$,表示小 R 的序列。\par
\textbf{输出:}一行一个非负整数,表示能选出的不相交区间数量的最大值。

\begin{table}[h]
\centering
\begin{tabularx}{\textwidth}{|X|X|}
\hline
\textbf{输入示例} & \textbf{输出示例}     \\    
\hline
4 2 & 2   \\ 
2 1 0 3 & \\ 
\hline
4 3 & 2   \\ 
2 1 0 3 & \\ 
\hline
4 0 & 1   \\ 
2 1 0 3 & \\ 
\hline
\end{tabularx}  
\end{table}

\subsection*{样例解释}
\textbf{样例1:}可以选择区间 $[1, 1]$ 和区间 $[2, 4]$,异或和分别为 $2$ 和 $1 \oplus 0 \oplus 3 = 2$,且两个区间不相交。

\textbf{样例2:}可以选择区间 $[1, 2]$ 和区间 $[4, 4]$,异或和分别为 $1 \oplus 2 = 3$ 和 $3$,且两个区间不相交。

\textbf{样例3:}可以选择区间 $[3, 3]$,异或和为 $0$。不能同时选择区间 $[3, 3]$ 和区间 $[1, 4]$,因为这两个区间相交。

\subsection*{算法分析}
\begin{enumerate}
\item \textcolor{important}{\textbf{前缀异或性质}}\\
定义 $pre[i]$ 为序列前 $i$ 个元素的异或和(即 $a_1 \oplus a_2 \oplus \dots \oplus a_i$),规定 $pre[0] = 0$。
根据异或的性质,任意区间 $[l, r]$ 的异或和可以表示为:
$$a_l \oplus \dots \oplus a_r = pre[r] \oplus pre[l-1]$$
题目要求区间异或和为 $k$,即 $pre[r] \oplus pre[l-1] = k$,移项得:
\textcolor{important}{$$pre[l-1] = pre[r] \oplus k$$}

\item \textcolor{important}{\textbf{动态规划状态定义}}\\
设 $dp[i]$ 表示在序列的前 $i$ 个数中,能够选出的满足条件的不相交区间的\textbf{最大数量}。

\item \textcolor{important}{\textbf{状态转移方程}}\\
对于当前位置 $i$,我们有两种选择策略:
\begin{enumerate}
    \item \textbf{不以 $i$ 作为区间的结尾}:此时最大数量继承自前一个位置,即 $dp[i] = dp[i-1]$。
    \item \textbf{寻找一个区间以 $i$ 结尾}:我们需要找到一个下标 $j$ ($0 \le j < i$),使得区间 $[j+1, i]$ 的异或和为 $k$。
    \begin{itemize}
        \item 由前缀异或性质可知,需要满足 $pre[j] = pre[i] \oplus k$。
        \item 如果存在这样的 $j$,则可以转移:$dp[i] = \max(dp[i], dp[j] + 1)$。
    \end{itemize}
\end{enumerate}
综上,状态转移方程为:
$$dp[i] = \max(dp[i-1], \max_{j \text{ satisfying condition}} (dp[j] + 1))$$

\item \textcolor{important}{\textbf{哈希表优化}}\\
为了快速找到满足 $pre[j] = pre[i] \oplus k$ 
的最佳下标 $j$,我们使用哈希表 map$<$int, int$>$

\begin{itemize}
    \item \textbf{键 (Key)}:前缀异或值 $val$。
    \item \textbf{值 (Value)}:该前缀异或值对应的下标 $j$。
    \item \textbf{贪心更新策略}:为了保证 $dp[i]$ 最大,我们在哈希表中记录该前缀异或值出现时对应的\textbf{最大 $dp$ 值}的下标。即:当计算出新的 $dp[i]$ 后,如果它比之前记录的同前缀异或值的 $dp$ 结果更优,才更新哈希表。
\end{itemize}

\end{enumerate}

\subsection*{参考代码}
\begin{lstlisting}
#include "bits/stdc++.h"
using namespace std;

using u32 = unsigned;
using i32 = int;
using u64 = unsigned long long;
using i64 = long long;
using u128 = unsigned __int128;
using i128 = __int128;

#define int long long
#define endl "\n"

constexpr i64 inf = 1e18;

void slu() {
    int n, k;
    cin >> n >> k;
    vector<int> a(n);
    for (auto &x : a) cin >> x;
    vector<int> pre(n + 1, 0);
    map<int, int> mp;
    vector<int> dp(n + 1, 0);
    for (int i = 1; i <= n; i++) {
        pre[i] = pre[i - 1] ^ a[i - 1];
    }

    mp[0] = 0;
    for (int i = 1; i <= n; i++) {
        int l = pre[i] ^ k;
        dp[i] = dp[i - 1];
        if (mp.contains(l)) {
            dp[i] = max(dp[i], dp[mp[l]] + 1);
        }
        if (dp[i] > dp[mp[pre[i]]]) {
            mp[pre[i]] = i;
        }
    }
    cout << dp[n];
}

signed main() {
    ios_base::sync_with_stdio(false);
    cin.tie(nullptr);
    cout.tie(nullptr);

    int t = 1;
    // cin >> t;

    while (t--) slu();
    return 0;
}

\end{lstlisting}

\subsection*{拓展思考}
\begin{itemize}
    \item \textcolor{important}{\textbf{优化空间}}:可以使用 \texttt{unordered\_map} 替代 \texttt{map},将单次查找的时间复杂度从 $O(\log n)$ 优化到期望 $O(1)$,从而将总时间复杂度优化到 $O(n)$。
\end{itemize}

\end{document}
% 主文件读取到这里就会停止