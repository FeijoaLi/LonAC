\documentclass{ctexart} % 子文件也要声明类型

% === 引入公共配置 ===
% 这样单独编译时,它就有颜色和代码高亮了
\usepackage{my_style} 

\begin{document} 
% 注意:主文件读取时,会忽略上面的所有内容,直接从这里开始读取

\section{\raggedright L1438 最小翻转次数 \ding{78}\ding{78}\ding{78}}

\subsection*{题目描述}
给定一个长度为 $n$ 的二进制字符串 $s$(仅包含字符 0 和 1)和一个整数 $k$。

你可以执行一种操作:选择字符串中的任意一个长度为 $k$ 的连续子串,并将该子串中的每一个字符进行翻转(即 0 变成 1,1 变成 0)。

你的目标是通过最少的操作次数,将整个字符串 $s$ 变成全 0 字符串。如果无法实现,输出 -1。

\subsection*{输入输出格式}
\textbf{输入:}第一行包含两个整数 $n$ 和 $k$($1 \le k \le n \le 10^5$),分别表示字符串长度和操作长度。
第二行包含一个长度为 $n$ 的字符串 $s$。

\textbf{输出:}输出一个整数,表示最少操作次数。如果无法实现,输出 -1。

\vspace{12pt}
\begin{table}[h]
\centering
\begin{tabularx}{0.85\textwidth}{|X|X|}
\hline
\textbf{输入示例} & \textbf{输出示例}     \\    
\hline
5 3 & 3   \\ 
10101 &   \\ 
\hline
\end{tabularx}  
\end{table}

\subsection*{样例解释}

初始字符串为 \texttt{10101},目标是全 \texttt{0},每次翻转长度 $k=3$。

\begin{table}[h]

\centering

\begin{tabularx}{0.95\textwidth}{|c|>{\centering\arraybackslash}X|>{\centering\arraybackslash}X|>{\centering\arraybackslash}X|}

\hline

\textbf{步骤} & \textbf{翻转区间} & \textbf{翻转后状态} \\

\hline

1 & [0, 2] & 01001 \\

\hline

2 & [1, 3] & 00111 \\

\hline

3 & [2, 4] & 00000 \\

\hline

\end{tabularx}  

\end{table}

\subsection*{算法分析}
\begin{enumerate}
\item \textbf{贪心策略证明} \\ 
本题目具有以下两个关键性质:
\begin{itemize}
    \item \textbf{顺序无关性}:操作的顺序不影响最终结果。先翻转区间 A 再翻转区间 B,和先翻转 B 再翻转 A,得到的结果是一样的。
    \item \textbf{操作的唯一性}:
    假设我们要处理当前最左边的一个位置 $i$,且下标 $0 \dots i-1$ 的部分已经全部变成 0 了。此时:
    \begin{enumerate}
        \item 如果 $s[i]$ 是 1,为了让它变成 0,我们必须翻转一个包含 $i$ 的区间。
        虽然包含 $i$ 的区间有很多(例如 $[i-1, i+k-2]$ 等),但只要区间的起点\textbf{小于} $i$,它就一定会覆盖到 $i$ 左侧那些\textcolor{blue}{已经处理好变为 0 的位置}。
        \textbf{翻转这些位置会破坏我们之前的成果}(将 0 变回 1)。
        \item \textbf{结论}:为了修复 $s[i]$ 且\textbf{绝对不破坏} $i$ 左边已完成的全 0 前缀,唯一的选择就是翻转以 $i$ 为起点的区间 $[i, i+k-1]$。
    \end{enumerate}
\end{itemize}
因此,每一个位置的操作决策是\textcolor{highlightpink}{唯一确定}的:遇到 1 必须翻转,且必须翻转以当前位置为开头的区间。

\item \textbf{算法选择}
\begin{enumerate}
\item \textbf{解法一 - 暴力模拟 (TLE) : }  
从左到右遍历,遇到 1 就翻转接下来的 $k$ 个字符。
单次翻转复杂度 $O(k)$,总复杂度 $O(nk)$。当 $n=10^5$ 时会超时。

\item \textbf{解法二 - 计数器优化 : }  
我们不需要真的去改变数组里的每一个值,只需要记录\textcolor{highlightpink}{当前位置受到了多少次翻转操作的影响}。
我们可以引入一个“计数器”和“过期标记”的概念,将复杂度降为 $O(n)$。
\end{enumerate}

\item \textbf{实现思路 (计数器 + 标记)}
\begin{itemize}
    \item \textbf{核心逻辑}
    想象一个\textcolor{blue}{滑动窗口},
    所有针对当前位置 $i$ 生效的翻转操作,都在这个窗口内。
    如果 [$i$ , $i+k-1$] 区间需要翻转,我们的操作就是:
    \begin{enumerate}
        \item \textbf{计数器自加 (cur++)}:当前位置 $i$ 的翻转次数 +1。
        \item \textbf{区间结束标记}:在 $i+k-1$ 处打一个标记,表示翻转区间在此处结束。
    \end{enumerate}

\setlength{\itemsep}{1pt}
\setlength{\parskip}{0pt}
\item \textbf{遍历字符串}:
当我们走到位置 $i$ 时,首先检查有没有操作在这里失效:
\begin{lstlisting}[language=C++]
cur += pre[i]; `// 加上那些在当前位置结束影响的操作`
\end{lstlisting}

\item \textbf{判断是否需要翻转}:
如果当前是 ‘1’ 且翻转次数是偶数 , 或当前是 ‘0’ 且翻转次数是奇数,则需要翻转。
反之不需要翻转。
\begin{lstlisting}[language=C++]
int ch = s[i] - '0';
if ((ch ^ cur) & 1) { `使用位运算简化判断逻辑`
    cnt++;      `// 记录操作次数`
    cur++;      `// 新增一个操作,当前位置的翻转次数+1`
    if (i + m > n) return -1; `// 长度不够,无法翻转,直接返回-1`
    pre[i + m]--; `// 并在 i+m 处打个标记:这个操作到时候要失效`
}
\end{lstlisting}

\end{itemize}

\end{enumerate}

\subsection*{拓展思考}
\begin{itemize}
\item 如果题目要求将字符串变成全 1,代码需要做哪些微调?
\end{itemize}
\newpage

\end{document}