\documentclass{ctexart} % 子文件也要声明类型

% === 引入公共配置 ===
% 这样单独编译时,它就有颜色和代码高亮了
\usepackage{my_style} 

\begin{document} 
% 注意:主文件读取时,会忽略上面的所有内容,直接从这里开始读取

\section{[CSP-J 2025] 拼数 / number} 

\subsection*{题目描述}
小 R 正在学习字符串处理。小 X 给了小 R 一个字符串 $s$,其中 $s$ 仅包含小写英文字母及数字,且包含至少一个 $1 \sim 9$ 中的数字。小 X 希望小 R 使用 $s$ 中的任意多个数字,按任意顺序拼成一个正整数。注意:小 R 可以选择 $s$ 中相同的数字,但每个数字只能使用一次。例如,若 $s$ 为 $\tt 1a01b$,则小 R 可以同时选择第 $1,3,4$ 个字符,分别为 $1,0,1$,拼成正整数 $101$ 或 $110$;但小 R 不能拼成正整数 $111$,因为 $s$ 仅包含两个数字 $1$。小 R 想知道,在他所有能拼成的正整数中,最大的是多少。你需要帮助小 R 求出他能拼成的正整数的最大值。

\subsection*{输入输出格式}
\textbf{输入:}第一行,一个字符串 $s$,表示小 X 给小 R 的字符串。\par
\textbf{输出:}一行,一个正整数,表示小 R 能拼成的正整数的最大值。

\begin{table}[h]
\centering
\begin{tabularx}{\textwidth}{|X|X|}
\hline
\textbf{输入示例} & \textbf{输出示例}     \\    
\hline
5 & 5   \\ 
\hline
290es1q0 & 92100   \\ 
\hline
\end{tabularx}  
\end{table}

\subsection*{样例解释}
\textbf{样例1:}字符串 $s = \text{"5"}$,只有一个数字5,所以最大正整数就是5。

\textbf{样例2:}字符串 $s = \text{"290es1q0"}$,包含数字 $2,9,0,1,0$。将这些数字从大到小排列得到 $9,2,1,0,0$,拼成的最大正整数为92100。

\subsection*{算法分析}
\begin{enumerate}
\item \textcolor{important}{\textbf{问题转化}}\\
从字符串中提取所有数字字符,然后将这些数字按照\textcolor{important}{\textbf{从大到小的顺序排列}},这样拼接起来的数字就是最大的正整数。

\item \textcolor{important}{\textbf{关键步骤}}
 \begin{enumerate}
    \item \textcolor{important}{\textbf{遍历字符串}},统计每个数字出现的次数
    \begin{lstlisting}
for (int i = 0; i < s.size(); i++) {
    if (s[i] >= '0' && s[i] <= '9') a[s[i] - '0']++;
}
    \end{lstlisting}
    \item 从数字9到数字0,依次将每个数字拼接到结果字符串中
\begin{lstlisting}
for (int i = 9; i >= 0; i--) {
    while (a[i]-- > 0) res += i + '0';
}
\end{lstlisting}    
    \item \textcolor{important}{\textbf{输出结果字符串}}
 \end{enumerate}

\item \textcolor{important}{\textbf{算法正确性证明}}
\begin{enumerate}
\item \textbf{贪心策略的正确性}\\
要得到最大的正整数,首先就是\textcolor{important}{\textbf{位数要尽可能多}},意味着每一个数字都要用到,其次应该让\textcolor{important}{\textbf{高位数字尽可能大}}。因此,将数字从大到小排列是最优策略。
\end{enumerate}
\end{enumerate}

\subsection*{参考实现}
\begin{lstlisting}
#include "bits/stdc++.h"
using namespace std;

using u64 = unsigned long long;
using i64 = long long;

// std::mt19937_64 rng {std::chrono::steady_clock::now().time_since_epoch().count()};
#define int long long
#define endl "\n"
constexpr i64 inf = 1e18;

void slu() {
    string s;
    cin >> s;
    vector<int> a(10, 0);
    for (int i = 0; i < s.size(); i++) {
        if (s[i] >= '0' && s[i] <= '9') a[s[i] - '0']++;
    }
    string res = "";
    for (int i = 9; i >= 0; i--) {
        while (a[i]-- > 0) res += i + '0';
    }
    cout << res << endl;
}

signed main() {
    ios_base::sync_with_stdio(false);
    cin.tie(nullptr);
    cout.tie(nullptr);

    int t = 1;
    // cin >> t;

    while (t--) slu();
    return 0;
}
\end{lstlisting}

\subsection*{拓展思考}
\begin{itemize}
    \item 如果题目修改为输出最小正整数,该如何实现?
\end{itemize}
\end{document}