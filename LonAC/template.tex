\documentclass{ctexart} % 子文件也要声明类型

% === 引入公共配置 ===
% 这样单独编译时,它就有颜色和代码高亮了
\usepackage{template} 

\begin{document} 
% 注意:主文件读取时,会忽略上面的所有内容,直接从这里开始读取

\section{[CSP-J 2025] 拼数 / number} 

\subsection*{题目描述}
小 R 正在学习字符串处理。小 X 给了小 R 一个字符串 $s$,其中 $s$ 仅包含小写英文字母及数字,且包含至少一个 $1 \sim 9$ 中的数字。小 X 希望小 R 使用 $s$ 中的任意多个数字,按任意顺序拼成一个正整数。注意:小 R 可以选择 $s$ 中相同的数字,但每个数字只能使用一次。例如,若 $s$ 为 $\tt 1a01b$,则小 R 可以同时选择第 $1,3,4$ 个字符,分别为 $1,0,1$,拼成正整数 $101$ 或 $110$;但小 R 不能拼成正整数 $111$,因为 $s$ 仅包含两个数字 $1$。小 R 想知道,在他所有能拼成的正整数中,最大的是多少。你需要帮助小 R 求出他能拼成的正整数的最大值。

\subsection*{输入输出格式}
\textbf{输入:}第一行,一个字符串 $s$,表示小 X 给小 R 的字符串。\par
\textbf{输出:}一行,一个正整数,表示小 R 能拼成的正整数的最大值。

\begin{table}[h]
\centering
\begin{tabularx}{\textwidth}{|X|X|}
\hline
\textbf{输入示例} & \textbf{输出示例}     \\    
\hline
5 & 5   \\ 
\hline
290es1q0 & 92100   \\ 
\hline
\end{tabularx}  
\end{table}

\subsection*{样例解释}
\textbf{样例1:}字符串 $s = \text{"5"}$,只有一个数字5,所以最大正整数就是5。

\textbf{样例2:}字符串 $s = \text{"290es1q0"}$,包含数字 $2,9,0,1,0$。将这些数字从大到小排列得到 $9,2,1,0,0$,拼成的最大正整数为92100。

\subsection*{算法分析}
\begin{enumerate}
\item \textcolor{important}{\textbf{问题转化}}\\
\item \textcolor{important}{\textbf{输出结果字符串}}
\end{enumerate}

\subsection*{参考实现}
\begin{lstlisting}

\end{lstlisting}

\subsection*{拓展思考}
\begin{itemize}
    \item 如果题目修改为输出最小正整数,该如何实现?
\end{itemize}
\end{document}