\documentclass{ctexart}
\usepackage{template}
\usepackage{float}
\usepackage{needspace} 
\usepackage{xcolor}

\begin{document}

\section*{\raggedright 体育课的队列 \ding{78}\ding{78}}

\subsection*{题目描述}

班级里共有 $N$ 名同学,每位同学都有一个学号(从 $1$ 到 $N$)、性别(用整数表示,$1$ 代表男生,$0$ 代表女生)和身高(整数,单位 cm)。

体育老师要求的队列规则如下:

\begin{enumerate}
    \item \textbf{性别优先}:所有的男生必须排在队伍的前半部分,所有的女生必须排在队伍的后半部分。
    \item \textbf{男生排序}:在前半部分的男生中,按照身高从高到低(\textbf{降序})排列。
    \item \textbf{女生排序}:在后半部分的女生中,按照身高从低到高(\textbf{升序})排列。
    \item \textbf{同等条件}:如果性别相同且身高也相同,则按照学号从小到大(\textbf{升序})的顺序排列。
\end{enumerate}

请你输出排好队后,队伍中从前到后依次对应的学号。

\subsection*{输入格式}

第一行包含一个整数 $N$,表示班级人数。

接下来 $N$ 行,每行包含两个整数 $g_i$ 和 $h_i$。

第 $i$ 行(输入数据的第 $2$ 到 $N+1$ 行)的数据对应学号为 $i$ 的同学的信息:
\begin{itemize}
    \item $g_i$ 表示性别($1$ 为男生,$0$ 为女生)。
    \item $h_i$ 表示身高。
\end{itemize}

\subsection*{输出格式}

共一行,包含 $N$ 个整数,表示排序后队列中同学的学号。整数之间用一个空格分隔。

\subsection*{输入输出样例}

% --- 表格 1 ---
\begin{table}[H]
\centering
\renewcommand{\arraystretch}{1.2} 
\setlength{\tabcolsep}{10pt}

\begin{tabular}{|p{0.45\textwidth}|p{0.45\textwidth}|}
\hline
\textbf{输入 \#1} & \textbf{输出 \#1} \\
\hline
\begin{minipage}[t]{\linewidth}
\begin{verbatim}
5
1 175
0 160
1 180
0 165
1 175
\end{verbatim}
\end{minipage}
\vspace{0.8em}
& 
\begin{minipage}[t]{\linewidth}
\begin{verbatim}
3 1 5 2 4
\end{verbatim}
\end{minipage} 
\vspace{0.8em}
\\
\hline
\end{tabular}
\end{table}

\subsection*{算法分析}
本题属于典型的 \textbf{\textcolor{highlightpink}{结构体多关键字排序}} 问题。解题核心在于自定义比较函数(Comparator)。



\begin{enumerate}
\item \textbf{数据结构设计} \\ 
为了方便排序时同时移动学号、性别和身高,我们需要定义一个结构体(struct)来存储每个学生的信息。
\begin{lstlisting}[language=C++]
struct Student {
    int id;     `// 学号`
    int gender; `// 性别:1男 0女`
    int height; `// 身高`
};
Student a[100005];
\end{lstlisting}

\item \textbf{比较逻辑 (\textcolor{blue}{$O(N \log N)$})} \\ 
我们需要实现一个比较函数 \texttt{cmp},配合 \texttt{sort} 使用。比较逻辑遵循题目要求的四个优先级:
\begin{itemize}
    \item \textbf{第一优先级(性别)}:
    男生排在女生前面。若 $A$ 是男生(1),$B$ 是女生(0),则 $A$ 优于 $B$。
    
    \item \textbf{第二优先级(身高)}:
    当性别相同时,需要分情况讨论:
    \begin{itemize}
        \item 若都是男生 ($gender == 1$):身高从高到低,即 $A.height > B.height$。
        \item 若都是女生 ($gender == 0$):身高从低到高,即 $A.height < B.height$。
    \end{itemize}

    \item \textbf{第三优先级(学号)}:
    当性别和身高都相同时,学号小的在前,即 $A.id < B.id$。
\end{itemize}

\Needspace{10\baselineskip}
\begin{lstlisting}[language=C++]
bool cmp(const Student &x, const Student &y) {
    `// 1. 性别不同,男生(1)排在女生(0)前`
    if (x.gender != y.gender) {
        return x.gender > y.gender; 
    }
    
    `// 2. 性别相同,判断身高`
    if (x.height != y.height) {
        if (x.gender == 1) {
            `// 男生:身高降序`
            return x.height > y.height;
        } else {
            `// 女生:身高升序`
            return x.height < y.height;
        }
    }
    
    `// 3. 兜底:学号升序`
    return x.id < y.id;
}
\end{lstlisting}

\item \textbf{主函数实现} \\
读入数据时注意手动赋值 $`id`$,然后调用 $sort$ 排序,最后输出 $`id`$ 即可。

\Needspace{8\baselineskip}
\begin{lstlisting}[language=C++]
    for (int i = 1; i <= n; i++) {
        a[i].id = i; `// 记录原始学号`
        cin >> a[i].gender >> a[i].height;
    }
    
    sort(a + 1, a + 1 + n, cmp);
    
    for (int i = 1; i <= n; i++) {
        cout << a[i].id << (i == n ? "" : " ");
    }
\end{lstlisting}
\end{enumerate}

\subsection*{拓展思考}

\subsubsection*{1. 进阶写法:重载小于号运算符 ($operator<$)}

在竞赛中,我们常常不写独立的 \texttt{cmp} 函数,而是直接在结构体内部\textbf{重载}小于号运算符。这样做的好处是结构体“自带”了排序规则,不仅可以直接调用 \texttt{sort},还能直接放入 \texttt{priority\_queue}(优先队列)或 \texttt{set} 中使用。

\begin{lstlisting}[language=C++]
struct Student {
    int id, gender, height;
    bool operator < (const Student &other) const {
        if (gender != other.gender) return gender > other.gender;
        if (height != other.height) {
            return gender == 1 ? height > other.height : height < other.height;
        }
        return id < other.id;
    }
};
\end{lstlisting}


\subsubsection*{2. 变式思考:关于“稳定性” (Stability)}



\textbf{问题}:如果题目删去规则$4$(\textit{“如果性别相同且身高也相同,则按照学号从小到大的顺序排列”}),改为\textbf{“保留输入时的相对顺序”},代码需要修改吗?

\textbf{分析}:
\begin{itemize}
    \item 既然 $id$ 就是按照输入顺序生成的($1 \dots N$),那么原本的规则 4 实际上就是“保留输入相对顺序”的显式表达,所以原代码依然有效。
    \item \textbf{如果数据中没有 $id$ 这一项呢?} 
    此时不能使用普通的 \texttt{sort},因为它是\textbf{不稳定排序}(Unstable Sort,基于快排),相等元素的相对位置可能会被打乱。
    \item \textbf{解决方案}:
    使用 C++ STL 提供的 \texttt{\textcolor{purple}{stable\_sort}}。它的用法与 \texttt{sort} 完全一致,但底层通常使用归并排序,能够保证“相等元素的相对位置不变”。
    \item \textbf{课后思考}:常见的诸如 \textbf{冒泡排序} (Bubble Sort)、\textbf{插入排序} (Insertion Sort) 是稳定的吗?快速排序 (Quick Sort) 呢?为什么?
\end{itemize}

\end{document}