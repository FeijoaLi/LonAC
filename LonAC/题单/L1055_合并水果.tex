%L1055 : 合并水果
\documentclass{ctexart}
\ctexset{
    section = {
        titleformat = \raggedright,
        name = {,、},
        number = \chinese{section}
    }
}
\usepackage[a4paper,top=2.54cm,bottom=2.54cm,left=3.18cm,right=3.18cm]{geometry}
\usepackage{eso-pic,graphicx}
\usepackage{amsmath,amsfonts,amssymb,amsthm}
\usepackage{tikz}
\usepackage{multirow}
\usepackage{array}
\usepackage{fancyhdr}
\usepackage{lastpage}
\usepackage{tabularx}
\usepackage{graphicx}
\usepackage{caption}
\usepackage{pifont}  
\usepackage{listings}
\usepackage{xcolor}




\definecolor{highlightpink}{RGB}{255,0,180}  % 粉红色
\definecolor{codebg}{RGB}{245,245,245}       % 代码背景色
\definecolor{lightgray}{gray}{0.8}           % 浅灰色(替代gray!20)
\definecolor{commentcolor}{RGB}{0,128,0}     % 注释颜色(深绿色)

% 设置 listings 样式
\lstset{
basicstyle=\ttfamily\small,
language=C++,
frame=single,
breaklines=true,
backgroundcolor=\color{codebg},
escapeinside=``,
commentstyle=\color{commentcolor},
keywordstyle=\color{blue},
stringstyle=\color{red},
rulecolor=\color{black},
framesep=5pt,
xleftmargin=10pt,
xrightmargin=10pt
}

\pagestyle{fancy}
\fancyhf{}
\cfoot{解码未来AI教育 \quad 编程题解}
\renewcommand{\headrulewidth}{0pt}
\renewcommand{\labelenumii}{\Alph{enumii}.}

\begin{document}

\section*{合并水果 \ding{78}\ding{78}} 

\subsection*{题目描述}
在一个果园里,张力恒同学已经将所有的果子打了下来,而且按果子的不同种类分成了不同的堆。张力恒决定把所有的果子合成一堆。

每一次合并,张力恒可以把两堆果子合并到一起,消耗的体力等于两堆果子的重量之和。可以看出,所有的果子经过 $n-1$ 次合并之后,就只剩下一堆了。张力恒在合并果子时总共消耗的体力等于每次合并所耗体力之和。

因为还要花大力气把这些果子搬回家,所以张力恒在合并果子时要尽可能地节省体力。假定每个果子重量都为 1,并且已知果子的种类数和每种果子的数目,你的任务是设计出合并的次序方案,使多多耗费的体力最少,并输出这个最小的体力耗费值。

\subsection*{输入输出格式}
\textbf{输入:}第一行,一个整数 $n(1 \leq n \leq 10000)$,表示果子的种类数;  
第二行,包含 $n$ 个整数,用空格分隔,第 $i$ 个整数 $a_i(1 \leq a_i \leq 20000)$,表示第 $i$ 种果子的数目。  

\textbf{输出:}一个整数,也就是最小的体力耗费值。输入数据保证这个值小于 $2^{31}$。

\vspace{12pt}
\begin{table}[h]
\centering
\begin{tabularx}{0.85\textwidth}{|X|X|}
\hline
\textbf{输入示例} & \textbf{输出示例}     \\    
\hline
3 & 15   \\ 
1 2 9 & \\ 
\hline
\end{tabularx}  
\end{table}

\subsection*{样例解释}
\begin{table}[h]
\centering
\begin{tabularx}{0.95\textwidth}{|c|>{\centering\arraybackslash}X|>{\centering\arraybackslash}X|>{\centering\arraybackslash}X|>{\centering\arraybackslash}X|}
\hline
\textbf{步骤} & \textbf{当前果堆状态} & \textbf{合并的果堆} & \textbf{本次消耗体力} & \textbf{累计消耗体力} \\ 
\hline
初始状态 & 1, 2, 9 & - & 0 & 0 \\ 
\hline
第一步 & 1, 2, 9 & 1 + 2 = 3 & 3 & 3 \\ 
\hline
第二步 & 3, 9 & 3 + 9 = 12 & 12 & 15 \\ 
\hline
\end{tabularx}  
\end{table}

\subsection*{算法分析}
\begin{enumerate}
\item \textbf{问题分析} \\ 
要使\textcolor{highlightpink}{总体力消耗最小},每次需要合并重量\textcolor{highlightpink}{最小}的两堆果子,重量小的果子被合并次数多,重量大的果子被合并次数少,总体力消耗最小。

\item \textbf{算法选择}
\begin{enumerate}
\item \textbf{朴素思想 : }  
每次从果堆中选出\textcolor{highlightpink}{最小}的两堆进行合并,再将合并后的重量插入数组并\textcolor{highlightpink}{重新排序},
重复操作直到只剩一堆。
每次排序花费时间为 $O(n \log n)$ , 一共进行 $n - 1$ 次合并,
所以总体时间复杂度为 $O(n^2 \log n)$ ,无法在 $1s$ 内执行完。

\item \textbf{使用优先队列(\textcolor{blue}{priority\_queue}) 优化 : }  
使用\textcolor{blue}{最小堆}维护\textcolor{highlightpink}{最小元素},每次取出和插入元素均 $O(\log n)$,一共进行 $n - 1$ 次取出和插入,
因此总体复杂度 $O(n \log n)$ ,可以在 $1s$ 内执行完毕。

\item \textbf{实现思路}
\begin{itemize}
\setlength{\itemsep}{1pt}
\setlength{\parskip}{0pt}
\item 将所有果堆的{重量}放入\textcolor{blue}{最小堆}中
\item 每次取出两个\textcolor{highlightpink}{最小}元素,{合并}并累加\textcolor{highlightpink}{体力消耗}
\item 将合并后的{新堆重量}放回堆中,重复直到堆只剩一个元素
\end{itemize}
\end{enumerate}
\end{enumerate}

\subsection*{背景知识}
\begin{itemize}
    \item \textbf{\textcolor{blue}{优先队列(堆)}}:一种特殊的数据结构,可以快速获取和删除\textcolor{highlightpink}{最小(或最大)元素}。
\begin{lstlisting}
priority_queue<int, vector<int>, greater<int>> pq;
\end{lstlisting}

\begin{itemize}
  \setlength{\itemsep}{2pt} 
  \setlength{\parskip}{0pt}

  \item \textbf{\textcolor{blue}{第一个参数 \texttt{int}}}:指定队列中存储的元素类型
  \item \textbf{\textcolor{blue}{第二个参数 \texttt{vector<int>}}}:指定底层使用的容器,通常是vector
  \item \textbf{\textcolor{blue}{第三个参数 \texttt{greater<int>}}}:比较函数,决定元素的排列顺序
    \begin{itemize}
      \setlength{\itemsep}{1pt} 
      \setlength{\parskip}{0pt}
      \item \textcolor{highlightpink}{\texttt{less<int>}}:\textcolor{blue}{最大堆},最大的元素在顶部(默认值)
      \item \textcolor{highlightpink}{\texttt{greater<int>}}:\textcolor{blue}{最小堆},最小的元素在顶部
    \end{itemize}
\end{itemize}

\textbf{常用操作:}

\begin{table}[h]
\centering
\renewcommand{\arraystretch}{1.3}
\setlength{\tabcolsep}{10pt}
\begin{tabularx}{0.95\textwidth}{>{\centering\arraybackslash}p{3cm}|X|>{\centering\arraybackslash}p{2.5cm}}
\textbf{操作} & \textbf{功能说明} & \textbf{时间复杂度} \\
\hline
\texttt{pq.push(x)} & 将元素 \texttt{x} 插入堆 & $O(\log n)$ \\
\hline
\texttt{pq.pop()} & 删除堆顶部(最小或最大)元素 & $O(\log n)$ \\
\hline
\texttt{pq.top()} & 获取堆顶部元素 & $O(1)$ \\
\hline
\texttt{pq.empty()} & 判断堆是否为空 & $O(1)$ \\
\hline
\texttt{pq.size()} & 获取堆中元素个数 & $O(1)$ \\
\hline
\end{tabularx}
\end{table}
\end{itemize}



\subsection*{核心代码}
\begin{lstlisting}
while (pq.size() >= 2) { `// 重复操作直到堆中只剩一个元素`
    int u = pq.top();
    pq.pop();
    int v = pq.top();
    pq.pop();
    cnt += u + v; `// 取出两个最小的元素,合并它们,并累加体力消耗`
    pq.push(u + v); `// 合并结果放回堆中`
}
\end{lstlisting}

\subsection*{拓展思考}
\begin{itemize}
\item 如果每次合并三堆、四堆,甚至 k($k \leq 10^5$) 堆果子,算法该如何修改?
\item 如果要求改成求消耗体力的\textcolor{highlightpink}{最大值},该如何实现?
\end{itemize}

\end{document}