\documentclass{ctexart}
\ctexset{
    section = {
        titleformat = \raggedright,
        name = {,、},
        number = \chinese{section}
    }
}
\usepackage[a4paper,top=2.54cm,bottom=2.54cm,left=3.18cm,right=3.18cm]{geometry}
\usepackage{amsmath,amsfonts,amssymb,amsthm}
\usepackage{tikz}
\usepackage{multirow}
\usepackage{array}
\usepackage{fancyhdr}
\usepackage{lastpage}
\usepackage{tabularx}
\usepackage{graphicx}
\usepackage{caption}
\usepackage{pifont}  
\usepackage{xcolor}
\usepackage{listings}
\usepackage{enumitem}

\definecolor{lightpink}{RGB}{184,83,182}
\definecolor{codebg}{RGB}{245,245,245}
\definecolor{emphcolor}{RGB}{199,21,133}
\definecolor{important}{RGB}{220,20,60}
\definecolor{highlight}{RGB}{0,100,0}
\definecolor{sumregion}{RGB}{255,200,200}
\definecolor{currentcell}{RGB}{255,0,0}

\lstset{
basicstyle=\ttfamily\small,
language=C++,
frame=single,
breaklines=true,
backgroundcolor=\color{codebg},
escapeinside=``,
commentstyle=\color{green!60!black},
keywordstyle=\color{blue},
stringstyle=\color{red},
rulecolor=\color{black},
framesep=5pt,
xleftmargin=10pt,
xrightmargin=10pt
}

\setlist[enumerate,1]{label=\arabic*., leftmargin=2em}
\setlist[enumerate,2]{label=(\arabic*), leftmargin=3em}
\setlist[enumerate,3]{label=\roman*, leftmargin=4em}

\pagestyle{fancy}
\fancyhf{}
\cfoot{解码未来AI教育 \quad 编程题解}
\renewcommand{\headrulewidth}{0pt}

\begin{document}

\section*{L1500 : 搜索2 \ding{78}\ding{78}\ding{78}} 

\subsection*{题目描述}
有一个奇怪的电梯,大楼有 $N$ 层楼,每层楼有一个数字 $K_i$($0 \le K_i \le N$)。电梯只有四个按钮:开,关,上,下。上下的层数等于当前楼层上的那个数字。如果不能满足要求,相应的按钮就会失灵。

例如:$3,3,1,2,5$ 代表了 $K_i$($K_1=3$,$K_2=3$,……),从1楼开始。在1楼,按"上"可以到4楼,按"下"是不起作用的,因为没有-2楼。

那么,从A楼到B楼至少要按几次按钮呢?

\subsection*{输入输出格式}
\textbf{输入:}第一行为三个用空格隔开的正整数,
\par\qquad\quad 表示 $N, A, B$($1 \le N \le 200$,$1 \le A, B \le N$)。\par
\qquad\quad 第二行为 $N$ 个用空格隔开的非负整数,表示 $K_i$。\par
\textbf{输出:}一行,即最少按键次数,若无法到达,则输出 $-1$。\par

\begin{table}[h]
\centering
\begin{tabularx}{\textwidth}{|X|X|}
\hline
\textbf{输入示例} & \textbf{输出示例}       \\      
\hline
5 1 5 & 3   \\   
3 3 1 2 5 &   \\   
\hline
\end{tabularx}  
\end{table}

\subsection*{样例分析}
以输入为例:

\begin{center}
\begin{tikzpicture}[scale=0.8]
\node at (0,4) {楼层};
\node at (0,3) {5};
\node at (0,2) {4};
\node at (0,1) {3};
\node at (0,0) {2};
\node at (0,-1) {1};

\node at (2,4) {K值};
\node at (2,3) {5};
\node at (2,2) {2};
\node at (2,1) {1};
\node at (2,0) {3};
\node at (2,-1) {3};
\end{tikzpicture}
\end{center}

\textbf{最优路径:} \texttt{1 → 4 → 2 → 5}(3次按键)

\begin{enumerate}
\item \textcolor{blue}{\textbf{第一次按键}}:从1楼($K_1=3$)按"上"到4楼
\item \textcolor{green}{\textbf{第二次按键}}:从4楼($K_4=2$)按"下"到2楼
\item \textcolor{red}{\textbf{第三次按键}}:从2楼($K_2=3$)按"上"到5楼(目标)
\end{enumerate}



\subsection*{算法分析}

\begin{enumerate}
\item \textcolor{highlight}{\textbf{问题分析}}  \\  
这是一个典型的\textcolor{emphcolor}{\textbf{最短路径}}问题,可以使用\textcolor{emphcolor}{\textbf{广度优先搜索(BFS)}}来解决。每个楼层是一个状态,每次按按钮是状态转移。

\item \textcolor{highlight}{\textbf{状态转移}}  \\  
从当前楼层 $u$,可以转移到两个可能的楼层:
\begin{itemize}
\item 向上:$u + K[u]$(如果不超过 $N$)
\item 向下:$u - K[u]$(如果不小于 $1$)
\end{itemize}

\item \textcolor{highlight}{\textbf{BFS算法步骤}}
\begin{enumerate}
\item 初始化距离数组,所有楼层设为-1(表示未访问)
\item 将起始楼层加入队列,距离设为0
\item BFS遍历:
\begin{itemize}
\item 从队列中取出当前楼层
\item 如果到达目标楼层,返回当前距离
\item 否则,尝试向上和向下移动
\item 如果新楼层在范围内且未被访问,更新距离并入队
\end{itemize}
\end{enumerate}

\item \textcolor{highlight}{\textbf{边界条件处理}}
\begin{itemize}
\item 向上移动:检查是否不超过 $N$
\item 向下移动:检查是否不小于 $1$
\item 如果起点和目标相同,直接返回0
\end{itemize}

\end{enumerate}

\subsection*{代码实现}

\begin{lstlisting}[language=C++]
while (!q.empty()) {
    int u = q.front();
    if (u == y) {
        cout << cost[u];
        return;
    }
    q.pop();
    for (int i = 0; i < 2; i++) {
        int v = u + a[u][i];
        if (v >= n || v < 0) continue;
        if (cost[v] > cost[u] + 1) {
            cost[v] = cost[u] + 1;
            q.push(v);
        }
    }
}
\end{lstlisting}


\subsection*{拓展思考}
\begin{itemize}
\item 如果电梯可以有多个按钮,每个按钮对应不同的移动距离,该如何解决?
\item 如果每次移动的代价不同(比如上楼比下楼费更多能量),该如何修改算法?
\item 如果要求输出具体的按键序列而不仅仅是最少次数,该如何实现?
\end{itemize}

\end{document}