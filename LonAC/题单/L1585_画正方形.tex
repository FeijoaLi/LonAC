\documentclass{ctexart} % 子文件也要声明类型

% === 引入公共配置 ===
% 这样单独编译时,它就有颜色和代码高亮了
\usepackage{template} 

\begin{document} 
% 注意:主文件读取时,会忽略上面的所有内容,直接从这里开始读取

\section*{\raggedright L1585 画正方形 \ding{78}} 

\subsection*{题目描述}
输入一个正整数 $n$,要求输出一个 $n$ 行 $n$ 列的正方形图案(参考样例输入输出)。图案由大写字母组成。
其中,第 1 行以大写字母 A 开头,第 2 行以大写字母 B 开头,以此类推;在每行中,第 2 列为第 1 列的下一个字母,第 3 列为第 2 列的下一个字母,以此类推;特别的,规定大写字母 Z 的下一个字母为大写字母 A。

\subsection*{输入输出格式}
\textbf{输入:}输入一行,包含一个正整数 $n$。约定 $2 \le n \le 40$。

\textbf{输出:}输出符合要求的正方形图案。

\vspace{12pt}
\begin{table}[h]
\centering
\begin{tabularx}{0.85\textwidth}{|X|X|}
\hline
\textbf{输入示例} & \textbf{输出示例}     \\    
\hline
3 & ABC \newline BCD \newline CDE   \\ 
\hline
\end{tabularx}  
\end{table}

\subsection*{样例解释}
输入 $n=3$,需要输出 $3 \times 3$ 的矩阵。

\begin{table}[h]
\centering
\begin{tabularx}{0.95\textwidth}{|c|>{\centering\arraybackslash}X|>{\centering\arraybackslash}X|}
\hline
\textbf{行号} & \textbf{起始字母} & \textbf{该行内容} \\ 
\hline
第 1 行 & A (第0个字母) & A $\to$ B $\to$ C \\ 
\hline
第 2 行 & B (第1个字母) & B $\to$ C $\to$ D \\ 
\hline
第 3 行 & C (第2个字母) & C $\to$ D $\to$ E \\ 
\hline
\end{tabularx}  
\end{table}

\subsection*{算法分析}
\begin{enumerate}
\item \textbf{问题分析} \\ 
我们需要打印一个二维矩阵。对于矩阵中的每一个位置 $(i, j)$,最核心的难点在于如何处理\textbf{“Z 的下一个是 A”}这一规则。

\begin{itemize}
    \item 我们可以将 A $\sim$ Z 映射为数字 $0 \sim 25$。
    \item 这是一个典型的\textcolor{highlightpink}{循环队列}模型:当数字到达 25 (代表 Z) 时,再 $+1$ 应该回到 0 (代表 A)。
    \item \textbf{取模运算 ($\%$)} 完美解决了这个问题。$x = (x + 1) \% 26$。
\end{itemize}



\item \textbf{算法选择}
\begin{enumerate}
\item \textbf{解法 - 模拟 : }  
使用双重循环遍历行 $i$ 和列 $j$。
对于第 $i$ 行,起始数值为 $i$。后续每往后一列,数值加 1。
在输出时,通过模运算计算出对应的字母偏移量。
\end{enumerate}

\item \textbf{实现思路}
\begin{itemize}
    \item \textbf{核心逻辑}
    \begin{enumerate}
\setlength{\itemsep}{1pt}
\setlength{\parskip}{0pt}
\item \textbf{循环结构}:外层循环控制行(0 到 $n-1$),内层循环控制列(0 到 $n-1$)。
\begin{lstlisting}[language=C++]
for (int i = 0; i < n; i++) { ... }
\end{lstlisting}

\item \textbf{字符计算与取模原理}:
假设当前我们应该输出第 $cur$ 个字母(0对应A, 1对应B...):
\begin{itemize}
    \item 核心公式:\textcolor{blue}{$(cur \% 26)$}
    \item 如下图所示,当 $cur$ 增加到 26 时,取模结果自动变回 0(A)。
    \item \vspace{10pt}
\begin{center}
\begin{tikzpicture}[scale=0.9]
    \def \radius {2.2cm}
    % 定义节点样式
    \tikzstyle{node_style} = [circle, draw=black, thick, minimum size=0.8cm, fill=white]
    \tikzstyle{highlight_style} = [circle, draw=red, thick, minimum size=0.8cm, fill=highlightpink!10]

    % 绘制圆环
    \draw [dashed, gray] (0,0) circle (\radius);

    % 绘制节点
    \node[node_style] (A) at (90:\radius) {\textbf{A}};
    \node[above] at (A.north) {\small 0};

    \node[node_style] (B) at (50:\radius) {B};
    \node[right] at (B.east) {\small 1};

    \node[node_style] (C) at (10:\radius) {C};
    \node[right] at (C.east) {\small 2};

    \node[node_style] (Y) at (170:\radius) {Y};
    \node[left] at (Y.west) {\small 24};

    \node[highlight_style] (Z) at (130:\radius) {\textbf{Z}};
    \node[left] at (Z.west) {\small 25};

    % 绘制箭头
    \draw[->, thick, >=stealth] (A) to[bend left=20] (B);
    \draw[->, thick, >=stealth] (B) to[bend left=20] (C);
    \draw[->, thick, >=stealth] (Y) to[bend left=20] (Z);
    
    % 关键的 Z->A 箭头
    \draw[->, ultra thick, red, >=stealth] (Z) to[bend left=20] node[midway, below, font=\small, yshift=-5pt] {} (A);

    \node at (0,0) [align=center, font=\small\color{gray}] {字母表\\循环示意图};
\end{tikzpicture}
\end{center}

\end{itemize}
最后加上字符 $A$ 即可还原为大写字母。
\begin{lstlisting}[language=C++]
int cur = i; // `当前行从第 i 个字母开始`
for (int j = 0; j < n; j++) {
    // `先对26取余,确保数值限制在0-25之间,再转换为字母`
    std::cout << (char)((cur++ % 26) + 'A');
}
std::cout << '\n'; // `每行结束后换行`
\end{lstlisting}
\end{enumerate}
\end{itemize}
\end{enumerate}

\subsection*{拓展思考}
\begin{itemize}
\item 如何只输出正方形的\textcolor{highlightpink}{外框}(即第一行、最后一行、第一列、最后一列)?
\item 如果规则改为:Z $\to$ Y $\to \dots \to$ B $\to$ A $\to$ Z (逆序)。该做哪些修改?
\end{itemize}

\newpage

\end{document}