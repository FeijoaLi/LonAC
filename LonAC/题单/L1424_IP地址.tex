\documentclass{ctexart}
\ctexset{
    section = {
        titleformat = \raggedright,
        name = {,、},
        number = \chinese{section}
    }
}
\usepackage[a4paper,top=2.54cm,bottom=2.54cm,left=3.18cm,right=3.18cm]{geometry}
\usepackage{amsmath,amsfonts,amssymb,amsthm}
\usepackage{tikz}
\usepackage{multirow}
\usepackage{array}
\usepackage{fancyhdr}
\usepackage{lastpage}
\usepackage{tabularx}
\usepackage{graphicx}
\usepackage{caption}
\usepackage{pifont}  
\usepackage[table]{xcolor}
\usepackage{listings}
\usepackage{enumitem}
\usepackage{eso-pic} % 添加水印的包

% 定义所有使用的颜色
\definecolor{lightpink}{RGB}{184,83,182}
\definecolor{codebg}{RGB}{245,245,245}
\definecolor{emphcolor}{RGB}{199,21,133}
\definecolor{important}{RGB}{220,20,60}
\definecolor{highlight}{RGB}{0,100,0}
\definecolor{sumregion}{RGB}{255,200,200}
\definecolor{currentcell}{RGB}{255,0,0}

% 修复的水印设置 - 使用更兼容的方法
% \AddToShipoutPictureBG{%
%   \AtPageCenter{%
%     \makebox[0pt]{%
%       \includegraphics[width=0.6\paperwidth,height=0.6\paperheight,keepaspectratio]{sy.png}%
%     }%
%   }%
% }

% 如果上面的方法仍然有问题,可以使用TikZ来实现透明度
\AddToShipoutPictureFG{%
  \AtPageCenter{%
    \begin{tikzpicture}[remember picture,overlay]
      \node[opacity=0.1] at (0,0) {\includegraphics[width=0.9\paperwidth,height=0.5\paperheight,keepaspectratio]{sy.png}};
    \end{tikzpicture}%
  }%
}

\lstset{
basicstyle=\ttfamily\small,
language=C++,
frame=single,
breaklines=true,
backgroundcolor=\color{codebg},
escapeinside=``,
commentstyle=\color{green!60!black},
keywordstyle=\color{blue},
stringstyle=\color{red},
rulecolor=\color{black},
framesep=5pt,
xleftmargin=10pt,
xrightmargin=10pt
}

\setlist[enumerate,1]{label=\arabic*., leftmargin=2em}
\setlist[enumerate,2]{label=(\arabic*), leftmargin=3em}
\setlist[enumerate,3]{label=\roman*, leftmargin=4em}

\pagestyle{fancy}
\fancyhf{}
\cfoot{解码未来AI教育 \quad 编程题解}
\renewcommand{\headrulewidth}{0pt}

\begin{document}

\section*{字符串匹配2: IP地址 \ding{78}\ding{78}} 

\subsection*{题目描述}
给定一个字符串 $s$,请你判断它是否是一个有效的 IPv4 地址。

一个有效的 IPv4 地址由四个整数构成,这些整数位于 0 到 255 之间,用 '.' 分隔。

每个整数不能有前导零(除非该整数本身就是 0),并且每个部分必须是非空的。

\subsection*{输入输出格式}
\textbf{输入:}输入包含多组测试数据。每组测试数据占一行,包含一个字符串 $s$ ($1 \le |s| \le 15$)。\par
\qquad\quad 输入以文件结束符(EOF)结束。\par
\textbf{输出:}对于每组测试数据,如果输入的字符串 $s$ 是一个有效的 IPv4 地址,则输出 "YES";否则输出 "NO"。\par
\qquad\quad 每个输出占一行。

\begin{table}[h]
\centering
\begin{tabularx}{\textwidth}{|X|X|}
\hline
\textbf{输入示例} & \textbf{输出示例}     \\    
\hline
192.168.1.1 & YES \\ 
192.168.1.01 & NO \\ 
0.0.0.0 & YES \\ 
255.255.255.255 & YES \\ 
\hline
\end{tabularx}  
\end{table}

\subsection*{样例解释}
\begin{itemize}
\item \textbf{192.168.1.1}: 所有部分都在 0-255 范围内,没有前导零,格式正确,输出 YES
\item \textbf{192.168.1.01}: 第四部分 "01" 有前导零,不符合要求,输出 NO
\item \textbf{0.0.0.0}: 所有部分都是 0,允许单个 0,输出 YES
\item \textbf{255.255.255.255}: 所有部分都在 0-255 范围内,没有前导零,格式正确,输出 YES
\end{itemize}

\subsection*{算法分析}

\begin{enumerate}
\item \textcolor{highlight}{\textbf{验证条件}}\\
IPv4 地址验证需满足:
\begin{itemize}
\item 恰好包含 3 个点号 '.',分成 4 个部分
\item 每部分为 0-255 的整数且无前导零
\end{itemize}

\item \textcolor{highlight}{\textbf{实现思路}}\\
\begin{enumerate}
\item 按 '.' 分割字符串为 4 部分
\item 验证每部分是否都是数字 且 大小在 0-255 范围内
\item 验证无前导零
\end{enumerate}
\end{enumerate}

\subsection*{代码实现:}

\text{判断逻辑 : }
\begin{lstlisting}
bool check (string s) {
    if (s.size() > 3 || s.size() == 0) return false; 
    `//超出3位数,或者是空的 直接不满足[0,255]条件`
    for (int i = 0; i < s.size(); i++) {
        if (s[i] > '9' || s[i] < '0') return false;
        `//验证都是数字`
    }
    for (int i = 0; i < s.size(); i++) { `//验证无前导零`
        if (s[i] == '0' && s.size() > 1) {
            return false;
        } else if (s.size() > 1) {
            break;
        }
    }
    
    int num = stoi(s); `//字符串转换为数字`
    if (num < 0 || num > 255) return false; `//判断是否在[0,255]范围内`
    return true;
};
\end{lstlisting}

\text{字符串处理 : }
\begin{lstlisting}
while (cin >> s) {
    string t;
    bool ok = true;
    for (int i = 0; i < s.size(); i++) {
        if (s[i] == '.') { ` // 遇到分割符号,检查当前部分是否满足要求`
            ok &= check(t);
            t.clear();
        } else {
            t += s[i];
        }
    }

    ok &= check(t); ` // 检查最后一部分`

    if (ok) {
        cout << "YES\n";
    } else {
        cout << "NO\n";
    }
}
\end{lstlisting}

\subsection*{拓展思考 :}
\begin{itemize}
\item 给定任意一个字符串和
一个分割的字符集,
如何实现一个通用的字符串分割函数?
\end{itemize}

\end{document}