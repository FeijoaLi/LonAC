%L1056 : 学生统计
\documentclass{ctexart}
\ctexset{
    section = {
        titleformat = \raggedright,
        name = {,、},
        number = \chinese{section}
    }
}
\usepackage[a4paper,top=2.54cm,bottom=2.54cm,left=3.18cm,right=3.18cm]{geometry}
\usepackage{eso-pic,graphicx}
\usepackage{amsmath,amsfonts,amssymb,amsthm}
\usepackage{tikz}
\usepackage{multirow}
\usepackage{array}
\usepackage{fancyhdr}
\usepackage{lastpage}
\usepackage{tabularx}
\usepackage{graphicx}
\usepackage{caption}
\usepackage{pifont}  
\usepackage{listings}
\usepackage{xcolor}

\definecolor{highlightpink}{RGB}{255,0,180}  % 粉红色
\definecolor{codebg}{RGB}{245,245,245}       % 代码背景色
\definecolor{lightgray}{gray}{0.8}           % 浅灰色(替代gray!20)
\definecolor{commentcolor}{RGB}{0,128,0}     % 注释颜色(深绿色)

% 设置 listings 样式
\lstset{
basicstyle=\ttfamily\small,
language=C++,
frame=single,
breaklines=true,
backgroundcolor=\color{codebg},
escapeinside=``,
commentstyle=\color{commentcolor},
keywordstyle=\color{blue},
stringstyle=\color{red},
rulecolor=\color{black},
framesep=5pt,
xleftmargin=10pt,
xrightmargin=10pt
}

\pagestyle{fancy}
\fancyhf{}
\cfoot{解码未来AI教育 \quad 编程题解}
\renewcommand{\headrulewidth}{0pt}
\renewcommand{\labelenumii}{\Alph{enumii}.}

\begin{document}

\section*{L1426 学生统计 \ding{78}\ding{78}} 

\subsection*{题目描述}
快开学了,小徐同学想帮老师统计一下本学期学校的学生信息。

老师给了一张学生信息表,每一条数据包括学生姓名、年级、班级。

老师想知道本学期每个年级有多少个班以及多少个人。

\subsection*{输入输出格式}
\textbf{输入:}第一行,一个整数 $N$,表示有 $N$ 条数据;  
接下来 $N$ 行,每行包含三个字符串,分别表示姓名、年级、班级。  

\textbf{输出:}按年级从小到大,输出年级的班级数和总人数。  
(注意,年级、班级总数、年级总数之间有一个空格,每行末尾不要有空格)

\vspace{12pt}
\begin{table}[h]
\centering
\begin{tabularx}{0.85\textwidth}{|X|X|}
\hline
\textbf{输入示例} & \textbf{输出示例}     \\    
\hline
5 & G7 2 3   \\ 
Tom G7 A & G8 1 2   \\ 
Jerry G7 B & \\ 
Llice G7 A & \\ 
Bob G8 A & \\ 
Apple G8 A & \\ 
\hline
\end{tabularx}  
\end{table}

\subsection*{样例解释}
\begin{table}[h]
\centering
\begin{tabularx}{0.95\textwidth}{|c|>{\centering\arraybackslash}X|>{\centering\arraybackslash}X|>{\centering\arraybackslash}X|}
\hline
\textbf{年级} & \textbf{班级列表} & \textbf{班级数} & \textbf{学生人数} \\ 
\hline
G7 & A班:Tom、Llice \quad B班:Jerry & 2 & 3 \\ 
\hline
G8 & A班:Bob、Apple & 1 & 2 \\ 
\hline
\end{tabularx}  
\end{table}

\subsection*{算法分析}
\begin{enumerate}
\item \textbf{问题分析} \\ 
需要统计\textcolor{highlightpink}{每个年级的班级数}和\textcolor{highlightpink}{学生总数}。
同一个年级的不同班级需要去重统计,而同一个年级的学生人数直接累加。

\item \textbf{算法选择}
\begin{enumerate}
\item \textbf{解法一 - 排序 + 遍历统计 : }  
先将所有学生信息按\textcolor{highlightpink}{年级}和
\textcolor{highlightpink}{班级}排序,
将年级小的放在前面,年级相同的按照班级排序,
然后遍历统计每个年级的信息。

\item \textbf{解法二 - Map/Set 统计 : }  
使用 \texttt{map} 和 \texttt{set} 容器自动维护有序性和去重特性,将年级作为键,班级,学生人数作为值
\end{enumerate}

\item \textbf{实现思路}
\begin{itemize}
    \item \textbf{法一:结构体排序 + 遍历统计}
    \begin{enumerate}
\setlength{\itemsep}{1pt}
\setlength{\parskip}{0pt}
\item 定义结构体存储学生信息(姓名、年级、班级)
\begin{lstlisting}[language=C++]
struct Node {
    string g;`年级`
    string c;`班级数`
    string n;`总人数`
};

bool cmp(Node a, Node b) {
    if (a.g == b.g) return a.c < b.c;
    return a.g < b.g;
}
\end{lstlisting}
\item 按\textcolor{blue}{年级升序、班级升序}排序
\item 遍历排序后的数据,统计每个年级的班级数和学生数
\begin{lstlisting}[language=C++]
if (g != a[i].g) { `枚举到了新的年级`
    cout << g << "` `" << cnt_class << "` `" 
         << cnt_stu << endl;
    g = a[i].g, c = a[i].c;
    cnt_class = 1, cnt_stu = 0;`数据清零`
}
cnt_stu++;`总人数加1`
if (c != a[i].c) {`遇到新的班级,班级数加1`
    c = a[i].c;
    cnt_class++;
}
\end{lstlisting}
\end{enumerate}


\item \textbf{法二:map + set 统计}
\begin{enumerate}
\item 使用 \texttt{map<string, int>} 统计每个年级的学生总数
\begin{lstlisting}[language=C++]
map<string, int> s;` // 年级 -> 学生数`
s[grade]++;
\end{lstlisting}

\item 使用 \texttt{map<string, set<string>>} 统计每个年级的班级集合(自动去重)
\begin{lstlisting}[language=C++]
map<string, set<string>> c;`// 年级 -> 班级集合`
c[grade].insert(clazz);
\end{lstlisting}

\item 遍历 \texttt{map} 输出结果
\begin{lstlisting}[language=C++]
for (auto i : s) {
    cout << i.first << "` `" << c[i.first].size() << "` `" 
         << i.second << endl;
}
\end{lstlisting}


\end{enumerate}
\end{itemize}

\end{enumerate}




\subsection*{拓展思考}
\begin{itemize}
\item 如果要求按\textcolor{highlightpink}{班级数从大到小}输出,该如何修改?
\item 如果数据中存在重复记录(相同姓名、年级、班级),该如何去重?
\end{itemize}
\newpage

\end{document}
