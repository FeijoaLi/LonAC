\documentclass{ctexart}
\ctexset{
    section = {
        titleformat = \raggedright,
        name = {,、},
        number = \chinese{section}
    }
}
\usepackage[a4paper,top=2.54cm,bottom=2.54cm,left=3.18cm,right=3.18cm]{geometry}
\usepackage{amsmath,amsfonts,amssymb,amsthm}
\usepackage{tikz}
\usepackage{multirow}
\usepackage{array}
\usepackage{fancyhdr}
\usepackage{lastpage}
\usepackage{tabularx}
\usepackage{graphicx}
\usepackage{caption}
\usepackage{pifont}  
\usepackage{xcolor}
%\usepackage{colortbl}  % 添加这一行来支持 \cellcolor
\usepackage{listings}
\usepackage{enumitem}

\definecolor{lightpink}{RGB}{184,83,182}
\definecolor{codebg}{RGB}{245,245,245}
\definecolor{emphcolor}{RGB}{199,21,133}
\definecolor{important}{RGB}{220,20,60}
\definecolor{highlight}{RGB}{0,100,0}
\definecolor{sumregion}{RGB}{255,200,200}
\definecolor{currentcell}{RGB}{255,0,0}
\definecolor{keypoint}{RGB}{0,100,0}
\definecolor{warning}{RGB}{255,0,0}

\lstset{
basicstyle=\ttfamily\small,
language=C++,
frame=single,
breaklines=true,
backgroundcolor=\color{codebg},
escapeinside=``,
commentstyle=\color{green!60!black},
keywordstyle=\color{blue},
stringstyle=\color{red},
rulecolor=\color{black},
framesep=5pt,
xleftmargin=10pt,
xrightmargin=10pt
}

\setlist[enumerate,1]{label=\arabic*., leftmargin=2em}
\setlist[enumerate,2]{label=(\arabic*), leftmargin=3em}
\setlist[enumerate,3]{label=\roman*, leftmargin=4em}

\pagestyle{fancy}
\fancyhf{}
\cfoot{解码未来AI教育 \quad 编程题解}
\renewcommand{\headrulewidth}{0pt}

\begin{document}

\section*{L1506 : 数字选取 \ding{78}\ding{78}\ding{78}} 

\subsection*{题目描述}
给定正整数 $n$,现在有 $1,2,\dots,n$ 共计 $n$ 个整数。你需要从这 $n$ 个整数中选取一些整数,使得所选取的整数中任意两个不同的整数均互质(也就是说,这两个整数的最大公因数为 $1$)。请你最大化所选取整数的数量。

例如,当 $n=9$ 时,可以选择 $1,5,7,8,9$ 共计 $5$ 个整数。可以验证不存在数量更多的选取整数的方案。

\subsection*{输入输出格式}
\textbf{输入:}一行,一个正整数 $n$,表示给定的正整数。\par
\qquad\quad 对于 $40\%$ 的测试点,保证 $1 \le n \le 1000$。对于所有测试点,保证 $1 \le n \le 10^5$。\par
\textbf{输出:}一行,一个正整数,表示所选取整数的最大数量。\par

\begin{table}[h]
\centering
\begin{tabularx}{\textwidth}{|X|X|}
\hline
\textbf{输入示例} & \textbf{输出示例}     \\    
\hline
6 & 4 \\ 
\hline
\end{tabularx}  
\end{table}

\subsection*{样例解释}
当 $n=6$ 时,可以选择 $1,2,3,5$ 或 $1,3,4,5$ 等方案,都能选取 $4$ 个整数。可以验证无法选取 $5$ 个整数。

\subsection*{算法分析}

\begin{enumerate}
\item \textcolor{highlight}{\textbf{问题分析}}\\
我们需要从 $1$ 到 $n$ 中选取尽可能多的数,使得任意两个数互质。

\item \textcolor{keypoint}{\textbf{关键观察}}
\begin{itemize}
\item \textcolor{important}{\textbf{数字 $1$ 与任何数都互质}}: $1$ 必须选
\item \textcolor{important}{\textbf{质数之间必定互质}},因为质数的定义是只有 $1$ 和自身两个正因子
\item \textcolor{important}{\textbf{一个重要的策略}}:我们可以选择所有的质数和 $1$,因为它们两两互质
\item \textcolor{warning}{\textbf{合数一定不能选}}:因为合数必然能拆成若干个质数相乘。例如 $6 = 2 \times 3$,并且这些质数必然小于该合数,所以这些质数(例如 $2, 3$)已经选取了,再选取 $6$ 会导致 $6$ 和 $3$ 不互质。
\end{itemize}

\item \textcolor{keypoint}{\textbf{贪心策略}}
\begin{enumerate}
\item 选择数字 $1$(与所有数互质)
\item 选择所有不超过 $n$ 的质数(质数之间互质)
\end{enumerate}

\newpage

\item \textcolor{keypoint}{\textbf{试除法判断质数}}
\begin{itemize}
\item \textcolor{important}{\textbf{原理}}:检查 $2$ 到 $\sqrt{n}$ 之间是否有 $n$ 的因子
\item \textcolor{important}{\textbf{时间复杂度}}:$O(\sqrt{n})$
\item \textcolor{important}{\textbf{代码实现}}:
\begin{lstlisting}[language=C++]
bool is_prime(int n) {
    if (n < 2) return false;
    for (int i = 2; i * i <= n; i++) {
        if (n % i == 0) return false;
    }
    return true;
}
\end{lstlisting}
\end{itemize}

\item \textcolor{keypoint}{\textbf{完整算法步骤}}
\begin{enumerate}
\item 初始化计数器 $cnt = 1$(包含数字 $1$)
\item 遍历 $2$ 到 $n$ 的所有整数
\item 对每个整数 $i$,使用试除法判断是否为质数
\item 如果是质数,计数器 $cnt$ 加 $1$
\item 输出最终的 $cnt$ 值
\end{enumerate}

\end{enumerate}


\subsection*{\textcolor{important}{\ding{45}} \textbf{其他常见的质数筛}}
\begin{table}[h]
\centering
\begin{tabular}{|c|c|c|c|}
\hline
\textbf{方法名称} & \textbf{时间复杂度} & \textbf{空间复杂度} & \textbf{适用场景} \\
\hline
试除法 & $O(\sqrt{n})$ & $O(1)$ & 单个质数判断 \\
\hline
埃拉托斯特尼筛法 & $O(n \log \log n)$ & $O(n)$ & 批量质数筛选 \\
\hline
欧拉筛法 & $O(n)$ & $O(n)$ & 高效批量筛选 \\
\hline
米勒-拉宾素性检验 & $O(k \log^3 n)$ & $O(1)$ & 大数质数判定 \\
\hline
分段筛法 & $O(n \log \log n)$ & $O(\sqrt{n})$ & 大范围质数筛选 \\
\hline
\end{tabular}
\end{table}

\subsection*{拓展思考}
\begin{itemize}
\item 如果数字范围扩大到 $10^7$ 或更大,算法需要如何优化?
\item 还有很多其他的筛选质数的方法,比如线性筛、欧拉筛等等,你能解释一下它们的相较于其他方法的优缺点吗?
\end{itemize}

\end{document}