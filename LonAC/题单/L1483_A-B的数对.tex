\documentclass{ctexart} % 子文件也要声明类型

% === 引入公共配置 ===
% 这样单独编译时,它就有颜色和代码高亮了
\usepackage{template} 

\begin{document} 
% 注意:主文件读取时,会忽略上面的所有内容,直接从这里开始读取

\section*{\raggedright L1483 A-B 数对 \ding{78}\ding{78}} 

\subsection*{题目描述}
给出一串正整数数列以及一个正整数 $C$,要求计算出所有满足 $A - B = C$ 的数对 $(A, B)$ 的个数。

注意:不同位置的数字如果数值相同,算作不同的数对。

\subsection*{输入输出格式}
\textbf{输入:}第一行包含两个正整数 $N$ 和 $C$($1 \le N \le 2 \times 10^5$, $1 \le C < 2^{30}$)。
第二行包含 $N$ 个正整数,表示数列中的元素(数值 $< 2^{30}$)。

\textbf{输出:}输出一个整数,表示满足条件的数对个数。

\vspace{12pt}
\begin{table}[h]
\centering
\begin{tabularx}{0.85\textwidth}{|X|X|}
\hline
\textbf{输入示例} & \textbf{输出示例}     \\    
\hline
4 1 & 3   \\ 
1 1 2 3 &   \\ 
\hline
\end{tabularx}  
\end{table}

\subsection*{样例解释}
输入的数列为 $\{1, 1, 2, 3\}$,目标差值 $C=1$。
我们需要寻找满足 $A - B = 1$ 的数对。满足条件的数对共有以下 3 对:

\begin{table}[h]
\centering
\begin{tabularx}{0.9\textwidth}{|c|>{\centering\arraybackslash}X|>{\centering\arraybackslash}X|}
\hline
\textbf{序号} & \textbf{数对 $(A, B)$} & \textbf{详细说明} \\ 
\hline
1 & $(2, 1)$ & 数组中\textbf{第3个}元素和\textbf{第1个}元素 \\ 
\hline
2 & $(2, 1)$ & 数组中\textbf{第3个}元素和\textbf{第2个}元素 \\ 
\hline
3 & $(3, 2)$ & 数组中\textbf{第4个}元素和\textbf{第3个}元素 \\ 
\hline
\end{tabularx}  
\end{table}

\noindent \small *注:虽然第1和第2组数对的数值看起来一样,但因为 $B$ 分别对应了数组中\textbf{不同下标位置}的元素,所以算作两个不同的答案。

\subsection*{算法分析}
\begin{enumerate}
\item \textbf{问题分析} \\ 
题目要求统计满足 $A - B = C$ 的数对。
这个等式可以变换为 $B = A - C$。
这意味着,对于数列中的每一个数,如果我们把它当作 $A$,那么我们只需要去查找数列中有多少个数值等于 $A - C$ 的数即可。

\item \textbf{算法选择}
\begin{enumerate}
\item \textbf{解法一 - 暴力枚举 (TLE) : }  
使用双重循环枚举所有的 $A$ 和 $B$,判断差值是否为 $C$。
复杂度为 $O(N^2)$。由于 $N$ 最大可达 $2 \times 10^5$,计算量高达 $4 \times 10^{10}$,显然会超时。

\item \textbf{解法二 - Map 统计 : }  
利用 \texttt{std::map} 记录每个数字出现的次数。然后遍历每个 A,使用map获取
B = A − C 在数组中的数量。时间复杂度 $O(N \log N)$。

\item \textbf{解法三 - 排序 + 二分查找 : }  
先将数组排序,然后对于每个 $A$,使用\textcolor{highlightpink}{二分查找}快速定位 $B = A - C$ 在数组中的数量。
时间复杂度同样为 $O(N \log N)$,但不需要 Map 的额外空间开销,常数更小。
\end{enumerate}

\item \textbf{实现思路}
\begin{itemize}
    \item \textbf{思路一:Map 统计}
    \begin{enumerate}
    \setlength{\itemsep}{1pt}
    \setlength{\parskip}{0pt}
    \item 遍历数组,用 \texttt{map} 统计每个数出现的频率。
    \begin{lstlisting}
mp[a[i]]++;
    \end{lstlisting}
    \item 再次遍历,累加 `mp[a[i] - c]` 即为答案。
    \begin{lstlisting}
cnt += mp[a[i] - c];
    \end{lstlisting}
    \end{enumerate}

    \item \textbf{思路二:排序 + 二分查找}
    \begin{enumerate}
\setlength{\itemsep}{1pt}
\setlength{\parskip}{0pt}
\item \textbf{排序}:
首先对数组进行升序排序,这是二分查找的前提。
\begin{lstlisting}[language=C++]
std::sort(a.begin(), a.end()); // `先对数组进行排序`
\end{lstlisting}

\item \textbf{二分查找(图解)}:
假设排序后的数组为 \texttt{\{1, 1, 2, 2, 2, 3\}},我们要查找目标值 \textbf{2} 的数量。

\begin{center}
\begin{tikzpicture}[
    cell/.style={rectangle, draw=black, minimum size=0.8cm, align=center},
    highlight/.style={fill=highlightpink!20}
]
    % 绘制数组
    \node[cell] (c0) at (0,0) {1};
    \node[cell] (c1) at (0.8,0) {1};
    \node[cell, highlight] (c2) at (1.6,0) {2};
    \node[cell, highlight] (c3) at (2.4,0) {2};
    \node[cell, highlight] (c4) at (3.2,0) {2};
    \node[cell] (c5) at (4.0,0) {3};
    
    % 绘制指针
    % lower_bound
    \draw[->, thick, blue] (1.6, -1.0) -- (1.6, -0.45);
    \node[blue, font=\small] at (1.6, -1.3) {lower\_bound (l)};
    \node[blue, font=\footnotesize] at (1.6, -1.7) {第一个 $\ge 2$ 的位置};

    % upper_bound
    \draw[->, thick, red] (4.0, -1.0) -- (4.0, -0.45);
    \node[red, font=\small] at (4.0, -1.3) {upper\_bound (r)};
    \node[red, font=\footnotesize] at (4.0, -1.7) {第一个 $> 2$ 的位置};

    % 绘制区间标识
    \draw [decorate,decoration={brace,amplitude=5pt,raise=0ex}, thick]
      (1.6, 0.5) -- (4.0, 0.5) node[midway,yshift=1.5em, font=\small] {数量 = $r - l = 3$};

\end{tikzpicture}
\end{center}

\vspace{5pt}
\begin{itemize}
    \item \texttt{lower\_bound}:指向第一个 \textbf{大于等于} 目标值的位置(开始位置)。
    \item \texttt{upper\_bound}:指向第一个 \textbf{大于} 目标值的位置(结束位置的下一个)。
    \item \textbf{目标数量} = \texttt{upper\_bound} 的下标 - \texttt{lower\_bound} 的下标。
\end{itemize}

\begin{lstlisting}[language=C++]
auto l = std::lower_bound(a.begin(), a.end(), a[i] - c);
auto r = std::upper_bound(a.begin(), a.end(), a[i] - c);
cnt += r - l;
\end{lstlisting}
\end{enumerate}
\end{itemize}

\end{enumerate}

\subsection*{拓展思考}
\begin{enumerate}
    \item \textbf{优化 - 双指针法 ($O(N)$)} \\
    在解法三中,我们先排序再二分,总复杂度是 $O(N \log N)$。
    \begin{itemize}
        \item 思考一下:既然数组已经\textcolor{highlightpink}{有序}了,当我们枚举的 $A$ 增大时,对应的 $B$ ($= A-C$) 肯定也是单调递增的。
        \item 是否可以使用两个指针(或三个指针,考虑到重复元素),只需遍历一遍数组就能统计出答案?这样可以将排序后的搜索过程降低到 $O(N)$。
    \end{itemize}
    \item \textbf{经典变式 - 两数之和 ($A + B = C$)} 
    如果题目改为计算 $A + B = C$ 的数对个数,该做哪些修改
\end{enumerate}
\newpage

\end{document}