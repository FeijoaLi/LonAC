\documentclass{ctexart}
\usepackage{template}  % 使用你自定义的模板包
\usepackage{float}     % 用于 [H] 参数
\usepackage{needspace} % 用于防止代码被截断
\usepackage{xcolor}    % 用于代码高亮颜色



\begin{document}

\section*{\raggedright P1081 [NOIP 2012 提高组] 开车旅行 \ding{78}\ding{78}\ding{78}\ding{78}}

\subsection*{题目描述}

小 $\text{A}$ 和小 $\text{B}$ 决定利用假期外出旅行,他们将想去的城市从 $1 $ 到 $n$ 编号,且编号较小的城市在编号较大的城市的西边,已知各个城市的海拔高度互不相同,记城市 $i$ 的海拔高度为$h_i$,城市 $i$ 和城市 $j$ 之间的距离 $d_{i,j}$ 恰好是这两个城市海拔高度之差的绝对值,即 $d_{i,j}=|h_i-h_j|$。
 
旅行过程中,小 $\text{A}$ 和小 $\text{B}$ 轮流开车,第一天小 $\text{A}$ 开车,之后每天轮换一次。他们计划选择一个城市 $s$ 作为起点,一直向东行驶,并且最多行驶 $x$ 公里就结束旅行。    

小 $\text{A}$ 和小 $\text{B}$ 的驾驶风格不同,小 $\text{B}$ 总是沿着前进方向选择一个最近的城市作为目的地,而小 $\text{A}$ 总是沿着前进方向选择第二近的城市作为目的地(注意:本题中如果当前城市到两个城市的距离相同,则认为离海拔低的那个城市更近)。如果其中任何一人无法按照自己的原则选择目的城市,或者到达目的地会使行驶的总距离超出 $x$ 公里,他们就会结束旅行。

在启程之前,小 $\text{A}$ 想知道两个问题:

1、 对于一个给定的 $x=x_0$,从哪一个城市出发,小 $\text{A}$ 开车行驶的路程总数与小 $\text{B}$ 行驶的路程总数的比值最小(如果小 $\text{B}$ 的行驶路程为 $0$,此时的比值可视为无穷大,且两个无穷大视为相等)。如果从多个城市出发,小 $\text{A}$ 开车行驶的路程总数与小 $\text{B}$ 行驶的路程总数的比值都最小,则输出海拔最高的那个城市。

2、对任意给定的 $x=x_i$ 和出发城市 $s_i$,小 $\text{A}$ 开车行驶的路程总数以及小 $\text B$ 行驶的路程总数。

\subsection*{输入格式}

第一行包含一个整数 $n$,表示城市的数目。

第二行有 $n$ 个整数,每两个整数之间用一个空格隔开,依次表示城市 $1$ 到城市 $n$ 的海拔高度,即 $h_1,h_2 ... h_n$,且每个 $h_i$ 都是互不相同的。

第三行包含一个整数 $x_0$。

第四行为一个整数 $m$,表示给定 $m$ 组 $s_i$ 和 $x_i$。

接下来的 $m$ 行,每行包含 $2$ 个整数 $s_i$ 和 $x_i$,表示从城市$s_i$ 出发,最多行驶 $x_i$ 公里。

\subsection*{输出格式}

输出共 $m+1$ 行。

第一行包含一个整数 $s_0$,表示对于给定的 $x_0$,从编号为 $s_0$ 的城市出发,小 $\text A$ 开车行驶的路程总数与小 $\text B$ 行驶的路程总数的比值最小。

接下来的 $m$ 行,每行包含 $2$ 个整数,之间用一个空格隔开,依次表示在给定的 $s_i$ 和 $x_i$ 下小 $\text A$ 行驶的里程总数和小 $\text B$ 行驶的里程总数。

\subsection*{输入输出样例}

% --- 表格 1 (保持原样:带边框、分开、底部留白) ---
\begin{table}[H]
\centering
\renewcommand{\arraystretch}{1.2} 
\setlength{\tabcolsep}{10pt}

\begin{tabular}{|p{0.45\textwidth}|p{0.45\textwidth}|}
\hline
\textbf{输入 \#1} & \textbf{输出 \#1} \\
\hline
\begin{minipage}[t]{\linewidth}
\begin{verbatim}
4 
2 3 1 4 
3 
4 
1 3 
2 3 
3 3 
4 3
\end{verbatim}
\end{minipage}
\vspace{0.8em}
& 
\begin{minipage}[t]{\linewidth}
\begin{verbatim}
1 
1 1 
2 0 
0 0 
0 0
\end{verbatim}
\end{minipage} 
\vspace{0.8em}
\\
\hline
\end{tabular}
\end{table}

% --- 表格 2 (保持原样:带边框、分开、底部留白) ---
\begin{table}[H]
\centering
\renewcommand{\arraystretch}{1.2}
\setlength{\tabcolsep}{10pt}

\begin{tabular}{|p{0.45\textwidth}|p{0.45\textwidth}|}
\hline
\textbf{输入 \#2} & \textbf{输出 \#2} \\
\hline
\begin{minipage}[t]{\linewidth}
\begin{verbatim}
10 
4 5 6 1 2 3 7 8 9 10 
7 
10 
1 7 
2 7 
3 7 
4 7 
5 7 
6 7 
7 7 
8 7 
9 7 
10 7
\end{verbatim}
\end{minipage}
\vspace{0.8em}
& 
\begin{minipage}[t]{\linewidth}
\begin{verbatim}
2 
3 2 
2 4 
2 1 
2 4 
5 1 
5 1 
2 1 
2 0 
0 0 
0 0
\end{verbatim}
\end{minipage} 
\vspace{0.8em}
\\
\hline
\end{tabular}
\end{table}

\subsection*{算法分析}
本题数据范围 $n, m \le 10^5$,且 $x$ 可达 $10^9$,简单的模拟会超时。解题分为两步:预处理“下一跳”和倍增优化模拟。

\begin{enumerate}
\item \textbf{预处理:寻找目标城市 ($O(N \log N)$)} \\ 
题目要求只能向东走,且寻找海拔最近和次近的城市。我们可以\textcolor{blue}{倒序遍历}城市(从 $n$ 到 $1$),维护一个包含当前城市以东所有城市海拔的集合。
\begin{itemize}
    \item 使用 \texttt{set} 存储 $\langle \text{海拔}, \text{ID} \rangle$,
    并预先插入哨兵值以避免边界判断。
    
    % \Needspace{6\baselineskip}
    \begin{lstlisting}[language=C++]
    set<array<int, 2>> st;
    st.insert({inf, n}); st.insert({inf + 1, n});
    st.insert({-inf, n}); st.insert({-inf - 1, n});
    \end{lstlisting}

    \item 对于城市 $i$(倒序遍历),在 \texttt{set} 中查找第一个海拔 $\ge h_i$ 的元素位置,并将该位置前后相邻的 4 个元素(前 2 个,后 2 个)设置为候选点。
    
    \Needspace{13\baselineskip}
    \begin{lstlisting}[language=C++]
    for (int i = n - 1; i >= 0; i--) {
        vector<int> list;
        auto it = st.lower_bound({h[i], i});
        auto tmp = it;
        
        `// lambda: 将合法候选点加入 list`
        auto put = [&](array<int, 2> x) {
            if (x[1] == n) return;
            list.push_back(x[1]);
        };
        put(*tmp++); put(*tmp--); `// 检查后两个`
        put(*--tmp); put(*--tmp); `// 检查前两个`
    \end{lstlisting}
    
    \item 对这 4 个候选点按照题目规则(和 城市 $i$ 距离优先,海拔其次)排序,即可得到小 B 的目标(最近)和小 A 的目标(次近),并记录在数组中。
    
    \Needspace{13\baselineskip}
    \begin{lstlisting}[language=C++]
        `// 排序:距离近优先,海拔低优先`
        sort(list.begin(), list.end(), [&](int &a, int &b) {
            int dis_a = get_dis(a, i);
            int dis_b = get_dis(b, i);
            if (dis_a == dis_b) return h[a] < h[b];
            return dis_a < dis_b;
        });

        if (list.size() >= 2) nxt_a[i] = list[1]; `// A去第二近`
        if (list.size() >= 1) nxt_b[i] = list[0]; `// B去最近`
        st.insert({h[i], i});
    }
    \end{lstlisting}
\end{itemize}

\item \textbf{倍增 (Binary Lifting) 优化 ($O(N \log N)$)} \\ 
由于 A 和 B 是轮流走的,我们将“A 走一步 + B 走一步”看作一个\textcolor{highlightpink}{完整的回合}。
\begin{itemize}
    \item \textbf{初始化 ($j=0$)}:计算 A 走一步 + B 走一步后的到达位置及各自路程。
    
    % \Needspace{9\baselineskip}
    \begin{lstlisting}[language=C++]
    for (int i = 0; i < n; i++) {`// 初始化 j=0 (A+B 各走一步)`
        int a = nxt_a[i]; int b = nxt_b[a];
        if (b == n) continue; `// 无法完成一轮`
        dis_a[i][0] = get_dis(i, a);
        dis_b[i][0] = get_dis(a, b);
        dp[i][0] = b;
    }
    \end{lstlisting}

    \item \textbf{状态转移}:利用 $2^j = 2^{j-1} + 2^{j-1}$ 的性质,递推计算倍增表。
    \begin{itemize}
        \item 从城市 $i$ 先走 $2^{j-1}$ 轮,到达中转点 $mid = dp[i][j-1]$;
        \item 然后从中转点 $mid$ 再走 $2^{j-1}$ 轮,到达最终目的地 $dp[i][j] = dp[mid][j - 1]$。
        \item 将 $mid$ 替换展开,即得到核心递推式:
        \[ dp[i][j] = dp[dp[i][j-1]][j-1] \]
    \end{itemize}
    
    % \Needspace{13\baselineskip}
    \begin{lstlisting}[language=C++]
    `// 倍增递推`
    for (int j = 1; j < 32; j++) {
        for (int i = 0; i < n; i++) {
            int mid = dp[i][j - 1];
            if (mid == n) continue;
            int end = dp[mid][j - 1];
            if (end == n) continue;

            dis_a[i][j] = dis_a[i][j - 1] + dis_a[mid][j - 1];
            dis_b[i][j] = dis_b[i][j - 1] + dis_b[mid][j - 1];
            dp[i][j] = end;
        }
    }
    \end{lstlisting}
\end{itemize}

\item \textbf{处理询问 ($O(M \log N)$)} \\ 
对于给定的起点 $s$ 和限额 $x$,我们利用倍增数组快速跳转:
\begin{enumerate}
    \item \textbf{倍增跳跃}:从最高位(如 $j=31$)向下枚举。如果走 $2^j$ 轮的总距离(A+B)不超过剩余 $x$,则跳过去,累加路程并更新当前位置。
    
    \Needspace{13\baselineskip}
    \begin{lstlisting}[language=C++]
    int a = 0, b = 0;
    `// 倍增跳跃 (A+B 成对跳)`
    for (int i = 31; i >= 0; i--) {
        int _a = dis_a[beg][i];
        int _b = dis_b[beg][i];
        `// 如果能跳且不超距离`
        if (_a + _b > x || dp[beg][i] == n) continue;

        a += _a; b += _b;
        x -= _a + _b;
        beg = dp[beg][i];
    }
    \end{lstlisting}
    
    \item \textbf{最后一步 A 的处理}:倍增只处理了完整的轮次。循环结束后,可能剩余的 $x$ 还足够小 A 单独再走一步(但不够 B 走了),需要特判加上这一步。
    
    \Needspace{6\baselineskip}
    \begin{lstlisting}[language=C++]
    `// 特判 A 的最后一步`
    if (nxt_a[beg] != n && get_dis(nxt_a[beg], beg) <= x) {
        a += get_dis(nxt_a[beg], beg);
    }
    return {a, b};
    \end{lstlisting}
\end{enumerate}

\newpage
\item \textbf{细节处理}
\begin{itemize}
\item \textbf{第一问比较}:比值 $\frac{A}{B}$ 
    的比较应小心处理 $B=0$ 的无穷大情况。
    若比值相同,需输出海拔最高的。
    \begin{lstlisting}[language=C++]
    int s0 = 0;
    long double tot = inf + 1;
    for (int i = 0; i < n; i++) {
        auto tmp = cul(i, x0); `// 计算从 i 出发,限距 x0 的结果`
        int a = tmp[0]; `// 小a走的距离`
        int b = tmp[1]; `// 小b走的距离`

        long double _tot = (b == 0 ? inf : (a + 0.0) / b);
        if (tot > _tot || (tot == _tot && h[i] > h[s0])) {
            s0 = i;
            tot = _tot;
        }
    }
    \end{lstlisting}
\end{itemize}

\end{enumerate}

\subsection*{解题总结与核心考点}

本题是一道结合了\textbf{数据结构预处理}与\textbf{倍增优化}的经典提高组试题,主要考察点如下:

\begin{enumerate}
    \item \textbf{倍增 (Binary Lifting) 思想的应用}:
    \begin{itemize}
        \item 常规的模拟在 $X$ 很大时会超时。利用倍增,我们将“走 $K$ 步”拆分为若干个 $2^i$ 步的组合,将单次查询复杂度从 $O(N)$ 降至 $O(\log N)$。
        \item 本题的特殊性在于 A 和 B 轮流移动,因此定义 $dp[i][j]$ 为“A和B各走 $2^j$ 步”是一个处理不对称移动的经典技巧。
    \end{itemize}
    
    \item \textbf{利用 STL Set 进行邻居查找}:
    \begin{itemize}
        \item 在一维坐标轴上动态查找前驱和后继(最近和次近),利用 \texttt{std::set} 的有序性和 \texttt{lower\_bound} 是标准解法。
        \item \textbf{哨兵技巧}:在 Set 中预先插入极值(INF),可以极大地简化边界条件的判断代码。
    \end{itemize}
    
    \item \textbf{细节与边界处理}:
    \begin{itemize}
        \item \textbf{残余步数}:倍增只能处理成对的步数,循环结束后 A 可能还能单独走一步,这是极其容易遗漏的边界。
        \item \textbf{精度与除零}:在比较 $A/B$ 比值时,需注意 $B=0$ 的情况,以及浮点数比较的精度问题(或转换为乘法比较)。
    \end{itemize}
\end{enumerate}

\end{document}