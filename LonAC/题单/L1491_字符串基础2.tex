\documentclass{ctexart}
\ctexset{
    section = {
        titleformat = \raggedright,
        name = {,、},
        number = \chinese{section}
    }
}
\usepackage[a4paper,top=2.54cm,bottom=2.54cm,left=3.18cm,right=3.18cm]{geometry}
\usepackage{amsmath,amsfonts,amssymb,amsthm}
\usepackage{tikz}
\usepackage{multirow}
\usepackage{array}
\usepackage{fancyhdr}
\usepackage{lastpage}
\usepackage{tabularx}
\usepackage{graphicx}
\usepackage{caption}
\usepackage{pifont}  
\usepackage{xcolor}
%\usepackage{colortbl}  % 添加这一行来支持 \cellcolor
\usepackage{listings}
\usepackage{enumitem}

\definecolor{lightpink}{RGB}{184,83,182}
\definecolor{codebg}{RGB}{245,245,245}
\definecolor{emphcolor}{RGB}{199,21,133}
\definecolor{important}{RGB}{220,20,60}
\definecolor{highlight}{RGB}{0,100,0}
\definecolor{sumregion}{RGB}{255,200,200}
\definecolor{currentcell}{RGB}{255,0,0}
\definecolor{keywordcolor}{RGB}{0,0,139}  % 深蓝色
\definecolor{conceptcolor}{RGB}{139,0,139}  % 深紫色
\definecolor{stepcolor}{RGB}{0,100,0}  % 深绿色

\lstset{
basicstyle=\ttfamily\small,
language=C++,
frame=single,
breaklines=true,
backgroundcolor=\color{codebg},
escapeinside=``,
commentstyle=\color{green!60!black},
keywordstyle=\color{blue},
stringstyle=\color{red},
rulecolor=\color{black},
framesep=5pt,
xleftmargin=10pt,
xrightmargin=10pt
}

\setlist[enumerate,1]{label=\arabic*., leftmargin=2em}
\setlist[enumerate,2]{label=(\arabic*), leftmargin=3em}
\setlist[enumerate,3]{label=\roman*, leftmargin=4em}

\pagestyle{fancy}
\fancyhf{}
\cfoot{解码未来AI教育 \quad 编程题解}
\renewcommand{\headrulewidth}{0pt}

\begin{document}

\section*{L1531 : 字符串基础2 \ding{78}\ding{78}} 

\subsection*{题目描述}
一串长度不超过 255 的 PASCAL 语言代码,只有 \textcolor{keywordcolor}{a, b, c 三个变量},
而且只有\textcolor{keywordcolor}{赋值语句},赋值只能是一个\textcolor{keywordcolor}{一位的数字}或一个\textcolor{keywordcolor}{变量},
每条赋值语句的格式是 \textcolor{red}{[变量]:=[变量或一位整数];}。
\textcolor{keywordcolor}{未赋值的变量值为 0},输出 a, b, c 的值。

\subsection*{输入输出格式}
\textbf{输入:}一串符合语法的 PASCAL 语言代码,\textcolor{keywordcolor}{只有 a, b, c 三个变量},\par
\qquad\quad 而且\textcolor{keywordcolor}{只有赋值语句},赋值只能是一个\textcolor{keywordcolor}{一位的数字}或一个\textcolor{keywordcolor}{变量},\par 
\qquad\quad \textcolor{keywordcolor}{未赋值的变量值为 0}。\par
\textbf{输出:}输出 a, b, c 最终的值。\\[0.5em]

\begin{table}[h]
\centering
\begin{tabularx}{\textwidth}{|X|X|}
\hline
\textbf{输入示例} & \textbf{输出示例}     \\    
\hline
a:=3;b:=4;c:=5; & 3 4 5 \\ 
\hline
a:=3;b:=a;c:=a; & 3 3 3 \\ 
\hline
a:=3;b:=a; & 3 3 0 \\ 
\hline
\end{tabularx}  
\end{table}

\subsection*{样例解释}
\begin{enumerate}
\item \texttt{a:=3;b:=4;c:=5;}\\
变量 a 被赋值为 3,变量 b 被赋值为 4,变量 c 被赋值为 5,所以输出 \texttt{3 4 5}。

\item \texttt{a:=3;b:=a;c:=a;}\\
变量 a 被赋值为 3,变量 b 被赋值为 a 的值(3),变量 c 被赋值为 a 的值(3),所以输出 \texttt{3 3 3}。

\item \texttt{a:=3;b:=a;}\\
变量 a 被赋值为 3,变量 b 被赋值为 a 的值(3),变量 c 没有被赋值(保持初始值 0),所以输出 \texttt{3 3 0}。
\end{enumerate}

\subsection*{算法分析}

\begin{enumerate}
\item \textcolor{highlight}{\textbf{问题分析}}\\
本题需要\textcolor{conceptcolor}{提取字符串中有效的部分},并\textcolor{conceptcolor}{跟踪三个变量 a, b, c 的值变化}。
赋值语句的格式固定,可以\textcolor{conceptcolor}{按分号分割语句},然后逐个解析。

\item \textcolor{highlight}{\textbf{关键步骤}}
\begin{enumerate}
\item \textcolor{stepcolor}{\textbf{初始化变量:}} 创建三个变量 a, b, c,初始值都为 0。
\item \textcolor{stepcolor}{\textbf{分割语句:}} 将输入字符串按分号分割成多个赋值语句。
\item \textcolor{stepcolor}{\textbf{解析每条语句:}} 对于每条语句,提取左侧变量和右侧值。
\item \textcolor{stepcolor}{\textbf{赋值处理:}} 如果右侧是数字,直接赋值;如果右侧是变量,
则使用该字母对应的数值。
\item \textcolor{stepcolor}{\textbf{输出结果:}} 处理完所有语句后,输出 a, b, c 的最终值。
\end{enumerate}

\item \textcolor{highlight}{\textbf{字符串处理技巧}}
\begin{itemize}
\item 使用 \textcolor{conceptcolor}{$map<char,int>$ 记录字母对应的数值}(默认初始值为 0)。
\begin{lstlisting}
map<char, int> mp;
mp['a'] = 0; mp['b'] = 0; mp['c'] = 0;
\end{lstlisting}




\item 使用 $map<char,int>$ 将 \textcolor{conceptcolor}{字符(char 类型)和数字(int 类型)等价}。例如 $mp['1'] = 1$(前者是 $char$ 类型,后者是 $int$ 类型)。
\begin{lstlisting}
for (int i = 0; i < 10; i++) {
    mp['0' + i] = i;
}
\end{lstlisting}
\item 根据一个赋值语句 【$a:=3;$】 占据 5 个字符,
因此把\textcolor{conceptcolor}{5 个字符为一个完整的赋值语句}。
在每次循环中,$a$ 是在0位置,$b$ 是在3位置, 所以只需要将 $mp[s[0]]$ 赋值为 $mp[s[3]]$ 即可。
\begin{lstlisting}
for (int i = 0; i < s.size(); i += 5) {
    mp[s[i]] = mp[s[i + 3]];
}
\end{lstlisting}
\end{itemize}
\end{enumerate}


\subsection*{拓展思考}
\begin{itemize}
    \item 如果\textcolor{conceptcolor}{变量名不止 a, b, c 三个},如何扩展程序?
    \item 如果\textcolor{conceptcolor}{赋值语句不止 3 句},如何处理?
\end{itemize}

\end{document}