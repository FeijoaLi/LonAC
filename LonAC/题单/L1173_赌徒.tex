\documentclass{ctexart}
\ctexset{
    section = {
        titleformat = \raggedright,
        name = {,、},
        number = \chinese{section}
    }
}
\usepackage[a4paper,top=2.54cm,bottom=2.54cm,left=3.18cm,right=3.18cm]{geometry}
\usepackage{amsmath,amsfonts,amssymb,amsthm}
\usepackage{tikz}
\usepackage{multirow}
\usepackage{array}
\usepackage{fancyhdr}
\usepackage{lastpage}
\usepackage{tabularx}
\usepackage{graphicx}
\usepackage{caption}
\usepackage{pifont}  
\usepackage{xcolor}
\usepackage{listings}
\usepackage{enumitem}

\definecolor{lightpink}{RGB}{184,83,182}
\definecolor{codebg}{RGB}{245,245,245}
\definecolor{emphcolor}{RGB}{199,21,133}
\definecolor{important}{RGB}{220,20,60}
\definecolor{highlight}{RGB}{0,100,0}
\definecolor{sumregion}{RGB}{255,200,200}
\definecolor{currentcell}{RGB}{255,0,0}

\lstset{
basicstyle=\ttfamily\small,
language=C++,
frame=single,
breaklines=true,
backgroundcolor=\color{codebg},
escapeinside=``,
commentstyle=\color{green!60!black},
keywordstyle=\color{blue},
stringstyle=\color{red},
rulecolor=\color{black},
framesep=5pt,
xleftmargin=10pt,
xrightmargin=10pt
}

\setlist[enumerate,1]{label=\arabic*., leftmargin=2em}
\setlist[enumerate,2]{label=(\arabic*), leftmargin=3em}
\setlist[enumerate,3]{label=\roman*, leftmargin=4em}

\pagestyle{fancy}
\fancyhf{}
\cfoot{解码未来AI教育 \quad 编程题解}
\renewcommand{\headrulewidth}{0pt}

\begin{document}

\section*{L1532 : 赌徒 \ding{78}\ding{78}\ding{78}} 

\subsection*{题目描述}
有 $n$ 个赌徒打算赌一局。规则是:
每人下一个赌注,赌注为非负整数,且任意两个赌注都不相同。
胜者为赌注恰好是其余任意三个人的赌注之和的那个人。
如果有多个胜者,我们取赌注最大的那个为最终胜者。

例如,$A, B, C, D, E$ 分别下赌注为 $2$、$3$、$5$、$7$、$12$,最终胜者是 $E$,因为 $12 = 2 + 3 + 7$。

\subsection*{输入输出格式}
\textbf{输入:}输入包含多组测试数据。每组首先输入一个整数 $n$($1 \leq n \leq 1000$),
表示赌徒的个数。接下来 $n$ 行每行输入一个非负整数 $b$($0 \leq b < 32768$),
表示每个赌徒下的赌注。当 $n = 0$ 时,输入结束。\par
\textbf{输出:}对于每组输入,输出最终胜者的赌注,如果没有胜者,则输出 \texttt{no solution}。

\begin{table}[h]
\centering
\begin{tabularx}{\textwidth}{|X|X|}
\hline
\textbf{输入示例} & \textbf{输出示例}     \\    
\hline
5 & 12 \\ 
2 & no solution \\ 
3 & \\ 
5 & \\ 
7 & \\ 
12 & \\ 
5 & \\ 
2 & \\ 
16 & \\ 
64 & \\ 
256 & \\ 
1024 & \\ 
0 & \\ 
\hline
\end{tabularx}  
\end{table}

\subsection*{样例解释}
\begin{enumerate}
\item \textbf{第一组数据:}赌注为 $2$、$3$、$5$、$7$、$12$\\
$12 = 2 + 3 + 7$,满足条件,所以输出 $12$。

\item \textbf{第二组数据:}赌注为 $2$、$16$、$64$、$256$、$1024$\\
没有任何一个赌注等于其他三个赌注之和,所以输出 \texttt{no solution}。
\end{enumerate}

\subsection*{算法分析}
\begin{enumerate}
\item \textcolor{highlight}{\textbf{问题分析}}\\
我们需要找最大的 $x$,使得存在三个数字 $y, z, w$,使得 $x = y + z + w$。
如果使用暴力做法,分别枚举 $x, y, z, w$,时间复杂度会达到 $O(n^4)$,超时。

\item \textcolor{highlight}{\textbf{优化思路}}
\begin{itemize}
    \item \textbf{预处理:} 枚举 $y, z$ 通过二重循环得到所有的 $y + z$ 和,
    并记录每个和出现的次数为$cnt[y + z]$。
    \item \textbf{枚举优化:} 再枚举 $x, w$,通过公式 $x = y + z + w$,
    得到 $y + z = x - w$,在预处理的和中查找是否存在 $x - w$。
    \item \textbf{避免重复:} 使用您的判断逻辑来避免重复使用同一个元素:
\begin{itemize}
    \item \textbf{避免重复元素的逻辑分析:}
    
    对于等式 $x - w = y + z$,我们需要确保 $x, y, z, w$ 是四个不同的赌注。
    
    当 $w$ 被重复使用时,会出现以下情况:
    \begin{align*}
    x - w &= y + z \\
    \text{如果 } y = w &\Rightarrow x - w = w + z \Rightarrow x - 2w = z
    \end{align*}
    \begin{itemize}
        \item 如果 $x - 2w > 0$ 且 $cnt[x-2w] = 1$,则必然会出现 $w$ 被重复使用的情况
        \item 如果 $x - 2w > 0$ 且 $cnt[x-2w] > 1$,则存在多组 $\{a,b\}$ 满足条件,可以避免重复
        \item 如果 $x - 2w \leq 0$,则不会出现重复问题
    \end{itemize}
\end{itemize}
\end{itemize}
\end{enumerate}

\subsection*{代码实现}

\text{预处理$y + z$}
\begin{lstlisting}[language=C++]
for (int i = 0; i < n; i++) {
    for (int j = i; j < n; j++) {
        if (i == j) continue;
        s[a[i] + a[j]]++;
    }
}
\end{lstlisting}

\text{枚举$x, w$}
\begin{lstlisting}[language=C++]
for (int i = 0; i < n; i++) {
    for (int j = 0; j < n; j++) {
        if (i == j) continue;
        int ns = a[i] - a[j]; `//x - w`
        int t = a[i] - 2 * a[j];`//z = x - 2w`
        if (ns <= 0) continue;
        if (s[ns] <= 0) continue;
        if (t > 0 && s[ns] <= cnt[t]) continue;
        res = max(res, a[i]);
    }
}
\end{lstlisting}

\subsection*{拓展思考}
\begin{itemize}
    \item 如果改成七个人 改成寻找满足的 $A = B + C + D + E + F + G$ 该如何实现?
\end{itemize}

\end{document}