\documentclass{ctexart}
\ctexset{
    section = {
        titleformat = \raggedright,
        name = {,、},
        number = \chinese{section}
    }
}
\usepackage[a4paper,top=2.54cm,bottom=2.54cm,left=3.18cm,right=3.18cm]{geometry}
\usepackage{amsmath,amsfonts,amssymb,amsthm}
\usepackage{tikz}
\usepackage{multirow}
\usepackage{array}
\usepackage{fancyhdr}
\usepackage{lastpage}
\usepackage{tabularx}
\usepackage{graphicx}
\usepackage{caption}
\usepackage{pifont}  
\usepackage{xcolor}
% \usepackage{colortbl}
\usepackage{listings}
\usepackage{enumitem}

\definecolor{lightpink}{RGB}{184,83,182}
\definecolor{codebg}{RGB}{245,245,245}
\definecolor{emphcolor}{RGB}{199,21,133}
\definecolor{important}{RGB}{220,20,60}
\definecolor{highlight}{RGB}{0,100,0}
\definecolor{sumregion}{RGB}{255,200,200}
\definecolor{currentcell}{RGB}{255,0,0}

\lstset{
basicstyle=\ttfamily\small,
language=C++,
frame=single,
breaklines=true,
backgroundcolor=\color{codebg},
escapeinside=``,
commentstyle=\color{green!60!black},
keywordstyle=\color{blue},
stringstyle=\color{red},
rulecolor=\color{black},
framesep=5pt,
xleftmargin=10pt,
xrightmargin=10pt
}

\setlist[enumerate,1]{label=\arabic*., leftmargin=2em}
\setlist[enumerate,2]{label=(\arabic*), leftmargin=3em}
\setlist[enumerate,3]{label=\roman*, leftmargin=4em}

\pagestyle{fancy}
\fancyhf{}
\cfoot{解码未来AI教育 \quad 编程题解}
\renewcommand{\headrulewidth}{0pt}

\begin{document}

\section*{L1499 : 宽度搜索1 \ding{78}\ding{78}\ding{78}} 

\subsection*{题目描述}
有一个 $n \times m$ 的棋盘,在某个点 $(x,y)$ 上有一个马,要求计算出马到达棋盘上任意一个点最少要走几步。

马每次走时,只能根据象棋的规则,走日字形状。

\subsection*{输入输出格式}
\textbf{输入:}一行四个整数,分别为 $n, m, x, y$($1 \le x \le n \le 400$,$1 \le y \le m \le 400$)。\par
\textbf{输出:}一个 $n \times m$ 的矩阵,代表马到达某个点最少要走几步(不能到达则输出 $-1$)。\par

\begin{table}[h]
\centering
\begin{tabularx}{\textwidth}{|X|X|}
\hline
\textbf{输入示例} & \textbf{输出示例}     \\    
\hline
3 3 1 1 & 0 3 2 \\ 
 & 3 -1 1 \\ 
 & 2 1 4 \\ 
\hline
\end{tabularx}  
\end{table}

\subsection*{样例分析}
以输入 \texttt{3 3 1 1} 为例:

\begin{center}
\begin{tabular}{|c|c|c|}
\hline
0 & 3 & 2 \\
\hline
3 & -1 & 1 \\
\hline
2 & 1 & 4 \\
\hline
\end{tabular}
\end{center}

\begin{itemize}
\item \textbf{(1,1)} 是起点,距离为0

\item \textbf{(1,2)} 需要3步:\texttt{(1,1)→(3,2)→(1,3)→(2,1)→(1,2)}

\item \textbf{(1,3)} 需要2步:\texttt{(1,1)→(2,3)→(1,3)}

\item \textbf{(2,1)} 需要3步:\texttt{(1,1)→(3,2)→(1,3)→(2,1)} 或 \texttt{(1,1)→(2,3)→(3,1)→(2,1)}

\item \textbf{(2,2)} 无法到达

\item \textbf{(2,3)} 需要1步:\texttt{(1,1)→(2,3)}

\item \textbf{(3,1)} 需要2步:\texttt{(1,1)→(2,3)→(3,1)} 或 \texttt{(1,1)→(3,2)→(3,1)}

\item \textbf{(3,2)} 需要1步:\texttt{(1,1)→(3,2)}

\item \textbf{(3,3)} 需要4步:\texttt{(1,1)→(3,2)→(1,3)→(2,1)→(3,3)}
\end{itemize}





\subsection*{算法分析}

\begin{enumerate}
\item \textcolor{highlight}{\textbf{问题分析}}\\
这是一个典型的\textcolor{emphcolor}{\textbf{最短路径}}问题,由于棋盘上每个位置的移动代价相同(每次移动一步),可以使用\textcolor{emphcolor}{\textbf{广度优先搜索(BFS)}}来解决。

\item \textcolor{highlight}{\textbf{马走日的8个方向}}\\
在国际象棋中,马走"日"字,有8个可能的移动方向:
\begin{center}
\begin{tikzpicture}[scale=0.8]
\draw[step=1cm, gray, very thin] (0,0) grid (5,5);
\node at (2.5,2.5) {马};
\draw[->, red, thick] (2.5,2.5) -- (0.5,3.5);
\draw[->, red, thick] (2.5,2.5) -- (1.5,4.5);
\draw[->, blue, thick] (2.5,2.5) -- (3.5,4.5);
\draw[->, blue, thick] (2.5,2.5) -- (4.5,3.5);
\draw[->, green, thick] (2.5,2.5) -- (4.5,1.5);
\draw[->, green, thick] (2.5,2.5) -- (3.5,0.5);
\draw[->, orange, thick] (2.5,2.5) -- (1.5,0.5);
\draw[->, orange, thick] (2.5,2.5) -- (0.5,1.5);
\end{tikzpicture}
\end{center}

用坐标表示这8个方向:
\begin{lstlisting}[language=C++]
int dx[8] = {-2, -1, 1, 2, 2, 1, -1, -2};
int dy[8] = {1, 2, 2, 1, -1, -2, -2, -1};
\end{lstlisting}

\item \textcolor{highlight}{\textbf{BFS算法步骤}}
\begin{enumerate}
\item 初始化距离数组,所有位置设为-1(表示未访问)
\item 将起始位置加入队列,距离设为0
\item BFS遍历:
\begin{itemize}
\item 从队列中取出当前位置
\item 遍历8个方向,计算下一个位置
\item 如果新位置在棋盘内且未被访问,更新距离并入队
\end{itemize}
\end{enumerate}
\end{enumerate}

\subsection*{代码实现}

\begin{lstlisting}[language=C++]
while (!q.empty()) {
    array<int, 2> TMP = q.front();
    int cx = TMP[0], cy = TMP[1];
    q.pop();
    for (int i = 0; i < 8; i++) {
        int nx = cx + dx[i], ny = cy + dy[i];
        `//判断是否越界`
        if (nx < 0 || nx >= n || ny < 0 || ny >= m) continue;
        if (res[nx][ny] > res[cx][cy] + 1) {
            res[nx][ny] = res[cx][cy] + 1;
            q.push({nx, ny});
        }
    }
}
\end{lstlisting}

\subsection*{拓展思考}
\begin{itemize}
\item 如果要求输出具体的行走路径而不仅仅是步数,该如何修改?
\item 如果棋盘上有障碍物,且具有拌马腿的限制,该如何处理?
\end{itemize}

\end{document}