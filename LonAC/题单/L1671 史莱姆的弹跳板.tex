\documentclass{ctexart}
\usepackage{template}
\usepackage{float}
\usepackage{needspace} 
\usepackage{xcolor}

\begin{document}

\section*{\raggedright 史莱姆的弹跳板 \ding{78}\ding{78}\ding{78}}

\subsection*{题目描述}

你是一只史莱姆,被困在了一个 $N \times M$ 的网格迷宫中。你的目标是到达出口 \textcolor{purple}{$E$}。
迷宫包含以下元素:
\begin{itemize}
    \item \texttt{\textcolor{teal}{.}} : 平地,可以通行。
    \item \texttt{\textcolor{red}{\#}} : 墙壁,无法通行,也无法停留。
    \item \texttt{\textcolor{purple}{S}} : 你的起点。
    \item \texttt{\textcolor{purple}{E}} : 出口。
    \item \texttt{\textcolor{blue}{T}} : 弹跳板。
\end{itemize}

\textbf{移动规则}:
\begin{enumerate}
    \item \textbf{普通移动}:向上下左右移动一格,花费 1 单位时间。
    \item \textbf{弹跳移动}:若当前站在 \texttt{\textcolor{blue}{T}} 上,可选择向上下左右一次性跳跃 \textcolor{orange}{\textbf{2}} 格。跳跃可以\textbf{跨越}障碍(如\texttt{\textcolor{red}{\#}}),但落脚点必须合法(非墙壁、不越界)。花费 1 单位时间。
\end{enumerate}

请计算从起点 $S$ 到出口 $E$ 的最少时间。如果无法到达,输出 \textcolor{red}{$-1$}。

\subsection*{输入格式}

第一行包含两个整数 $N$ 和 $M$。
接下来 $N$ 行,每行包含一个长度为 $M$ 的字符串。

\subsection*{输出格式}

一个整数,表示最短时间 $t$。如果不通,输出 $-1$。

\subsection*{输入输出样例}

% --- 表格 1 ---
\begin{table}[H]
\centering
\renewcommand{\arraystretch}{1.2} 
\setlength{\tabcolsep}{10pt}

\begin{tabular}{|p{0.45\textwidth}|p{0.45\textwidth}|}
\hline
\textbf{输入} & \textbf{输出} \\
\hline
\begin{minipage}[t]{\linewidth}
\begin{verbatim}
5 5
S.#..
T.#..
#....
...#.
...E.
\end{verbatim}
\end{minipage}
\vspace{0.8em}
& 
\begin{minipage}[t]{\linewidth}
\begin{verbatim}
3
\end{verbatim}
\end{minipage} 
\vspace{0.8em}
\\
\hline
\end{tabular}
\end{table}

\subsection*{算法分析与代码实现}

本题是在二维网格图上的\textbf{\textcolor{blue}{最短路问题}}。
由于无论是“普通移动”还是“弹跳移动”,花费的时间都是固定的 1 单位,
因此 \textbf{\textcolor{highlightpink}{广度优先搜索 (BFS)}} 
是解决此类的最优算法。

\begin{enumerate}
\item \textbf{基础框架与方向数组} \\ 
我们需要定义方向数组来简化上下左右的移动代码。
同时,我们需要一个 \texttt{dist} 数组来记录起点到每个点的最短距离,
初始化为 \textcolor{red}{$-1$} 表示未到达。

\Needspace{12\baselineskip} % 预留 12 行空间,防断裂
\begin{lstlisting}[language=C++]
const int MAXN = 1005;
char mp[MAXN][MAXN];
int dist[MAXN][MAXN];
int n, m;

`// 方向数组:上、下、左、右`
int dx[4] = {-1, 1, 0, 0};
int dy[4] = {0, 0, -1, 1};

struct Node { int x, y; };
queue<Node> q;
\end{lstlisting}

\item \textbf{BFS 初始化与普通移动} \\ 
从起点 $S$ 开始入队。对于每一个状态 $(x, y)$,
首先尝试普通的“走一步”。

\textbf{注意}:BFS 的核心在于“第一次到达某点时,
步数一定是最少的”,所以如果 \texttt{dist[nx][ny]} 
已经被更新过(不为 \textcolor{red}{-1}),则不需要重复走。

\Needspace{18\baselineskip} % 预留 18 行空间,防断裂
\begin{lstlisting}[language=C++]
    while (!q.empty()) {
        Node curr = q.front(); q.pop();
        int x = curr.x, y = curr.y;

        `// 1. 尝试四个方向的普通移动 (距离 1)`
        for (int i = 0; i < 4; i++) {
            int nx = x + dx[i];
            int ny = y + dy[i];

            `// 越界或撞墙检查`
            if (nx < 1 || nx > n || ny < 1 || ny > m) continue;
            if (mp[nx][ny] == '#') continue;

            `// 如果该点未被访问过`
            if (dist[nx][ny] == -1) {
                dist[nx][ny] = dist[x][y] + 1;
                q.push({nx, ny});
            }
        }
\end{lstlisting}

\item \textbf{特殊机制:弹跳板逻辑} \\ 
在处理当前点 $(x, y)$ 时,
如果当前点字符是 \texttt{\textcolor{blue}{ $'T'$ }},
则可以额外进行一次判定:向四个方向跳跃 
\textcolor{orange}{\textbf{2}} 格。

\textbf{关键点}:跳跃可以跨越障碍,但落脚点必须合法。
这意味着我们不需要检查中间那一个格子是什么,
只需要检查落脚点 $(nnx, nny)$。

\Needspace{18\baselineskip} % 预留 18 行空间,防断裂
\begin{lstlisting}[language=C++]
        `// 2. 如果脚下是弹跳板 T,尝试弹跳移动 (距离 2)`
        if (mp[x][y] == 'T') {
            for (int i = 0; i < 4; i++) {
                `// 坐标变化量乘 2`
                int nnx = x + dx[i] * 2;
                int nny = y + dy[i] * 2;

                `// 检查落脚点是否越界`
                if (nnx < 1 || nnx > n || nny < 1 || nny > m) continue;
                `// 检查落脚点是否是墙 (中间可以跨越,不用管)`
                if (mp[nnx][nny] == '#') continue;

                if (dist[nnx][nny] == -1) {
                    dist[nnx][nny] = dist[x][y] + 1; 
                    `// 弹跳也只花 1 时间`
                    q.push({nnx, nny});
                }
            }
        }
    } 
\end{lstlisting}
\end{enumerate}


\subsection*{拓展思考}

\subsubsection*{1. 如果弹跳板有方向限制?}
如果是“超级马里奥”风格的弹跳板,规定了只能向特定方向弹射(例如 \texttt{U} 表示只能向上弹),代码该如何修改?
我们需要修改 \texttt{mp[x][y] == 'T'} 的判断逻辑,根据当前字符决定遍历方向数组的哪些索引,或者建立字符到方向的映射表。

\subsubsection*{2. 边权变化:Dijkstra 算法}
如果题目改为:普通移动花费 1 秒,但启动弹跳板需要\textbf{蓄力花费 3 秒}。
此时边权不再全部为 1(有 1 和 3 两种边权),
简单的 BFS 队列性质失效(先入队的未必距离更近)。
这时候需要使用 \textbf{\textcolor{purple}{Dijkstra}} 算法 
来求解。

\end{document}