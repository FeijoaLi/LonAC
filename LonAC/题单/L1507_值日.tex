% L1507 : 值日
\documentclass{ctexart}
\ctexset{
    section = {
        titleformat = \raggedright,
        name = {,、},
        number = \chinese{section}
    }
}
\usepackage[a4paper,top=2.54cm,bottom=2.54cm,left=3.18cm,right=3.18cm]{geometry}
\usepackage{amsmath,amsfonts,amssymb,amsthm}
\usepackage{tikz}
\usepackage{multirow}
\usepackage{array}
\usepackage{fancyhdr}
\usepackage{lastpage}
\usepackage{tabularx}
\usepackage{graphicx}
\usepackage{caption}
\usepackage{pifont}  
\usepackage{xcolor}
\usepackage{listings}
\usepackage{enumitem}

\definecolor{lightpink}{RGB}{184,83,182}
\definecolor{codebg}{RGB}{245,245,245}
\definecolor{emphcolor}{RGB}{199,21,133}
\definecolor{important}{RGB}{220,20,60}
\definecolor{highlight}{RGB}{0,100,0}

% 正确设置 listings
\lstset{
basicstyle=\ttfamily\small,
language=C++,
frame=single,
breaklines=true,
backgroundcolor=\color{codebg},
escapeinside=``,
commentstyle=\color{green!60!black},
keywordstyle=\color{blue},
stringstyle=\color{red},
rulecolor=\color{black},
framesep=5pt,
xleftmargin=10pt,
xrightmargin=10pt
}

% 设置 enumerate 层级格式
\setlist[enumerate,1]{label=\arabic*., leftmargin=2em}
\setlist[enumerate,2]{label=(\arabic*), leftmargin=3em}
\setlist[enumerate,3]{label=\roman*, leftmargin=4em}

\pagestyle{fancy}
\fancyhf{}
\cfoot{解码未来AI教育 \quad 编程题解}
\renewcommand{\headrulewidth}{0pt}

\begin{document}

\section*{L1507 : 值日 \ding{78}} 

\subsection*{题目描述}
小杨和小红是值日生,负责打扫教室。\textcolor{emphcolor}{\textbf{小杨每 $m$ 天值日一次}},\textcolor{emphcolor}{\textbf{小红每 $n$ 天值日一次}}。
今天他们两个一起值日,请问至少多少天后,他们会再次同一天值日?

\subsection*{输入输出格式}
\textbf{输入:}第一行,一个正整数 $m$,表示小杨的值日周期;\par
\qquad\quad 第二行,一个正整数 $n$,表示小红的值日周期。\par
\textbf{输出:}一行,一个整数,表示至少多少天后他们会再次同一天值日。

\begin{table}[h]
\centering
\begin{tabularx}{\textwidth}{|X|X|}
\hline
\textbf{输入示例} & \textbf{输出示例}     \\    
\hline
2 & 6   \\ 
3 & \\ 
\hline
\end{tabularx}  
\end{table}

\subsection*{样例解释}
小杨的值日周期 $m = 2$,小红的值日周期 $n = 3$。

从第0天开始,他们各自的值日安排如下表所示:
\begin{table}[h]
\centering
\begin{tabularx}{0.9\textwidth}{|c|*{7}{>{\centering\arraybackslash}X|}}
\hline
\textbf{天数} & 0 & 1 & 2 & 3 & 4 & 5 & 6 \\ 
\hline
小杨 & \textcolor{lightpink}{\textbf{是}} & 否 & \textcolor{lightpink}{\textbf{是}} & 否 & \textcolor{lightpink}{\textbf{是}} & 否 & \textcolor{lightpink}{\textbf{是}} \\ 
\hline
小红 & \textcolor{lightpink}{\textbf{是}} & 否 & 否 & \textcolor{lightpink}{\textbf{是}} & 否 & 否 & \textcolor{lightpink}{\textbf{是}} \\ 
\hline
\end{tabularx}  
\end{table}

从表中可以看出,他们下一次同时值日是第6天,输出结果为6。

\subsection*{背景知识}
\begin{itemize}
\item \textcolor{highlight}{\textbf{最小公倍数(LCM)}}:两个或多个整数公有的倍数中最小的一个。例如,2和3的最小公倍数是6($\text{LCM}(2,3) = 6$),4和8的最小公倍数是8($\text{LCM}(4,8) = 8$)。

\item \textcolor{highlight}{\textbf{最大公约数(GCD)}}:两个或多个整数共有约数中最大的一个。例如,2和3的最大公约数是1($\text{GCD}(2,3) = 1$),12和18的最大公约数是6($\text{GCD}(12,18) = 6$)。

\item \textcolor{highlight}{\textbf{重要关系}}:两个整数的最小公倍数与最大公约数满足:$$\text{GCD}(m, n) \times \text{LCM}(m, n) = m \times n$$
\end{itemize}

\subsection*{算法分析}
\begin{enumerate}
\item \textcolor{important}{\textbf{问题转化}}\\
小杨值日的日期是 $0, m, 2m, 3m, \ldots$,小红值日的日期是 $0, n, 2n, 3n, \ldots$。
因此,小杨和小红再次同一天值日的天数必须\textcolor{emphcolor}{\textbf{既是 $m$ 的倍数,又是 $n$ 的倍数}},从而得出最早的一天就是$m$ 和 $n$ 的\textcolor{important}{\textbf{最小公倍数}}($\text{LCM}(m,n)$)。

\newpage

\item \textcolor{important}{\textbf{计算$\text{LCM}(m,n)$}}
 \begin{enumerate}
    \item 使用 \_\_gcd() 计算 $\text{GCD}(m,n)$ 
    \item 通过最小公倍数和最大公约数关系($\text{GCD} \times \text{LCM}= m \times n$)计算 $\text{LCM}(m,n)$
 \end{enumerate}

\begin{lstlisting}
int GCD = __gcd(n, m);
int LCM = m * n / GCD;
\end{lstlisting}

\item \textcolor{important}{\textbf{欧几里得算法计算$\text{GCD}(m,n)$}}

\textbf{证明:$\text{GCD}(m, n)  = \text{GCD}(n, m \% n)$}
\begin{enumerate}
\item \textbf{基本情况}\\
假设 $m > n$,如果 $m$ 能被 $n$ 整除,那么 $\text{GCD}(m,n) = n$。
\item \textcolor{important}{\textbf{递推关系}}\\
如果 $m$ 不能被 $n$ 整除,我们可以得到关系式:$m = n \times k + r \quad (0 < r < n)$\par
其中:$k$ 是 $m \div n$ 的商,$r$ 是余数。\\
例如:$m = 16, n = 6$ 时,$16 = 6 \times 2 + 4$,所以 $k = 2, r = 4$。
\item \textcolor{important}{\textbf{关键证明}}\\
设 $d = \text{GCD}(m, n)$,则:
$d$ 能整除 $m$ 和 $n$,因此 $d$ 也能整除 $m - n \times k = r$,\par 所以 $d$ 是 $n$ 和 $r$ 的公约数。\par
因此我们得到:$d = \text{GCD}(m, n) = \text{GCD}(n, r) = \text{GCD}(n, m \% n)$
\end{enumerate}

\textbf{算法步骤:}

\begin{enumerate}
\item 计算 $\text{GCD}(m,n)$,当 $n \neq 0$ 时,重复以下步骤:
  \begin{enumerate}
  \item 计算余数 $r = m \% n$,令 $m = n$,$n = r$
  \item 当 $n = 0$ 时,$m$ 即为最大公约数
  \end{enumerate}
\end{enumerate}

\textbf{代码实现:}
    \begin{lstlisting}
while (n != 0) {
    int r = m % n;
    m = n;
    n = r;
}
return m;
    \end{lstlisting}

\textbf{示例:计算 $\text{GCD}(48, 18)$}

\begin{enumerate}[label={}, leftmargin=2em]
\item \textcolor{highlight}{\textbf{第一步}}:$48 \div 18 = 2$……$12$,所以 $\text{GCD}(48, 18) = \text{GCD}(18, 12)$
\item \textcolor{highlight}{\textbf{第二步}}:$18 \div 12 = 1$……$6$,所以 $\text{GCD}(18, 12) = \text{GCD}(12, 6)$  
\item \textcolor{highlight}{\textbf{第三步}}:$12 \div 6 = 2$……$0$,余数出现0,程序终止,所以 $\text{GCD}(12, 6) = 6$
\item \textcolor{important}{\textbf{最终结果}}:$\text{GCD}(48, 18) = 6$
\end{enumerate}


\end{enumerate}

\subsection*{拓展思考}
\begin{itemize}
    \item \textcolor{important}{\textbf{整数溢出问题}}:如果 $n$ 和 $m$ 很大,直接计算 $m \times n$ 可能会导致整数溢出。
    \item \textcolor{important}{\textbf{多人情况}}:如果是3个人、4个人,甚至是10000个人求下次同时值日的天数。
\end{itemize}

\end{document}
