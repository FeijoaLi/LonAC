\documentclass{ctexart}
\usepackage{template} 
\usepackage{float}
\usepackage{needspace} 
\usepackage{xcolor}

\begin{document}

\section*{\raggedright 英语课的单词接龙 \ding{78}\ding{78}}

\subsection*{题目描述}

英语老师为了活跃课堂气氛,组织大家玩单词接龙游戏。游戏规则如下:
后一个单词的\textbf{\textcolor{blue}{首字母}}必须与前一个单词的\textbf{\textcolor{blue}{尾字母}}相同(\textbf{不区分大小写},例如 \texttt{Dog} 后可以接 \texttt{Girl})。 

现老师记录了班长依次说出的 $N$ 个单词,请你编写程序判断该接龙序列是否合法。

\subsection*{输入格式}

第一行一个整数 $N$。

接下来 $N$ 行,每行一个由英文字母组成的字符串。

\subsection*{输出格式}

如果全部符合规则,输出 \texttt{\textcolor{teal}{Valid}}。

否则输出一个整数 \textcolor{red}{$X$},表示第 $X$ 个单词出错了(从 $1$ 开始计数)。

\subsection*{输入输出样例}

% --- 表格 1 ---
\begin{table}[H]
\centering
\renewcommand{\arraystretch}{1.2} 
\setlength{\tabcolsep}{10pt}

\begin{tabular}{|p{0.45\textwidth}|p{0.45\textwidth}|}
\hline
\textbf{输入 \#1} & \textbf{输出 \#1} \\
\hline
\begin{minipage}[t]{\linewidth}
\begin{verbatim}
4
Apple
Egg
Good
Day
\end{verbatim}
\end{minipage}
\vspace{0.8em}
& 
\begin{minipage}[t]{\linewidth}
\begin{verbatim}
Valid
\end{verbatim}
\end{minipage} 
\vspace{0.8em}
\\
\hline
\end{tabular}
\end{table}

\subsection*{算法分析}
本题主要考察 \textbf{\textcolor{highlightpink}{字符串的基本操作}} 与 \textbf{\textcolor{highlightpink}{字符大小写转换}}。

\begin{enumerate}
\item \textbf{数据存储} \\ 
由于题目需要输出出错单词的编号 $X$,我们可以使用 \texttt{string} 
数组(或 \texttt{vector<string>})来存储所有单词,
下标建议从 $1$ 开始,方便直接对应题目要求的编号。

\item \textbf{核心逻辑 (\textcolor{blue}{$O(N)$})} \\ 
我们需要从第 $2$ 个单词开始遍历到第 $N$ 个单词。

对于每一个单词 $s[i]$,检查首字母 $s[i][0]$ 
是否与前一个单词 $s[i-1]$ 的尾字母匹配。
\begin{itemize}
    \item 获取前一个单词的尾字母:
    可以使用 \texttt{s[i-1].back()} 或者 
    \texttt{s[i-1][s[i-1].length()-1]}。
    \item \textbf{忽略大小写}:
    这是本题的易错点。'A' 和 'a' 在 ASCII 码中是不相等的。
    比较时需要统一转换为大写或小写。
\end{itemize}

\item \textbf{字符处理:忽略大小写} \\ 
题目要求不区分大小写,例如 'A' 和 'a' 应视为相同。我们可以将首尾字母统一转换为小写(或大写)后再进行比较。这里介绍两种常用方法:

\begin{itemize}
    \item \textbf{方法一:调用标准库函数(推荐)} \\
    C++ 的 \texttt{<cctype>} 头文件提供了 \texttt{tolower(c)} 函数,可以将字符转换为小写(如果是符号或已经是小写,则保持不变)。
    
    \Needspace{10\baselineskip}
    \begin{lstlisting}[language=C++]
    #include <cctype> 
    // ...
    char head = tolower(s[i][0]);       `// 当前单词首字母转小写`
    char tail = tolower(s[i-1].back()); `// 前一单词尾字母转小写`

    if (head != tail) {
        cout << i << endl; `// 发现不匹配,输出编号`
        return 0;          `// 直接结束程序`
    }
    \end{lstlisting}

    \item \textbf{方法二:利用 ASCII 码手动转换(底层原理)}  \\
    在 ASCII 码表中,大写字母 \texttt{'A'} (65) 与小写字母 \texttt{'a'} (97) 之间相差 \textbf{32}。因此,判断一个字符如果在大写范围内('A' $\sim$ 'Z'),只要加上 32 即可变为对应的小写字母。
    
    \Needspace{8\baselineskip}
    \begin{lstlisting}[language=C++]
    char head = s[i][0];
    char tail = s[i-1].back();
    
    `// 手动转小写:如果是大写字母,加 32 变小写`
    if (head >= 'A' && head <= 'Z') head += 32;
    if (tail >= 'A' && tail <= 'Z') tail += 32;
    
    if (head != tail) {
        cout << i << endl;
        return 0;
    }
    \end{lstlisting}
\end{itemize}

\item \textbf{完整流程} \\ 
如果循环结束后都没有触发 \texttt{return 0},说明所有单词都符合规则,最后输出 \texttt{Valid}。

\end{enumerate}

\subsection*{拓展思考}
\subsubsection*{进阶变式:最长单词接龙 (DFS/图论)}
本题只是\textbf{验证}给定的序列是否合法。在竞赛(如 NOIP 2000 提高组)中,更经典的问题是:
\begin{quote}
    给出一堆单词,如果不规定顺序,问你能拼接出的\textbf{最长}接龙序列有多长?
\end{quote}
\textbf{思路}:
这就从“模拟”变成了“搜索”。我们需要把每个单词看作图中的一个节点,如果单词 A 的尾部能接单词 B 的首部,就建立一条有向边(或在 DFS 中递归跳转)。由于可能存在环或者单词复用次数限制,通常使用 \textbf{\textcolor{purple}{深度优先搜索 (DFS)}} 来寻找最长路径。

\end{document}