\documentclass{ctexart}
\ctexset{
    section = {
        titleformat = \raggedright,
        name = {,、},
        number = \chinese{section}
    }
}
\usepackage[a4paper,top=2.54cm,bottom=2.54cm,left=3.18cm,right=3.18cm]{geometry}
\usepackage{amsmath,amsfonts,amssymb,amsthm}
\usepackage{tikz}
\usepackage{multirow}
\usepackage{array}
\usepackage{fancyhdr}
\usepackage{lastpage}
\usepackage{tabularx}
\usepackage{graphicx}
\usepackage{caption}
\usepackage{pifont}  
\usepackage{xcolor}
\usepackage{listings}
\usepackage{enumitem}

\definecolor{lightpink}{RGB}{184,83,182}
\definecolor{codebg}{RGB}{245,245,245}
\definecolor{emphcolor}{RGB}{199,21,133}
\definecolor{important}{RGB}{220,20,60}
\definecolor{highlight}{RGB}{0,100,0}
\definecolor{sumregion}{RGB}{255,200,200}
\definecolor{currentcell}{RGB}{255,0,0}
\definecolor{formula}{RGB}{0,0,139}
\definecolor{keycode}{RGB}{178,34,34}

\lstset{
basicstyle=\ttfamily\small,
language=C++,
frame=single,
breaklines=true,
backgroundcolor=\color{codebg},
escapeinside=``,
commentstyle=\color{green!60!black},
keywordstyle=\color{blue},
stringstyle=\color{red},
rulecolor=\color{black},
framesep=5pt,
xleftmargin=10pt,
xrightmargin=10pt
}

\setlist[enumerate,1]{label=\arabic*., leftmargin=2em}
\setlist[enumerate,2]{label=(\arabic*), leftmargin=3em}
\setlist[enumerate,3]{label=\roman*, leftmargin=4em}

\pagestyle{fancy}
\fancyhf{}
\cfoot{解码未来AI教育 \quad 编程题解}
\renewcommand{\headrulewidth}{0pt}

\begin{document}

\section*{有趣的数字和 \ding{78}\ding{78}} 

\subsection*{题目描述}
如果一个正整数的二进制表示包含奇数个 1,那么小A就会认为这个正整数是有趣的。

例如,7 的二进制表示为 (111)$_2$,包含 1 的个数为 3 个,所以 7 是有趣的。但是 9=(1001)$_2$ 包含 2 个 1,所以 9 不是有趣的。

给定正整数 $l,r$,请你统计满足 $l \leq n \leq r$ 的有趣的整数 $n$ 之和。

\subsection*{输入输出格式}
\textbf{输入:}一行,两个正整数 $l,r$,表示给定的正整数。\par
\textbf{输出:}一行,一个正整数,表示 $l,r$ 之间有趣的整数之和。\par

\begin{table}[h]
\centering
\begin{tabularx}{\textwidth}{|X|X|}
\hline
\textbf{输入示例} & \textbf{输出示例}     \\    
\hline
3 8 & 19 \\ 
\hline
5 15 & 70 \\
\hline
1 10 & 25 \\
\hline
\end{tabularx}  
\end{table}

\subsection*{样例解释}
\begin{table}[h]
\centering
\begin{tabular}{|c|c|c|c|}
\hline
\textbf{十进制} & \textbf{二进制} & \textbf{1的个数} & \textbf{是否有趣} \\ 
\hline
3 & 0011 & 2个 & 否 \\
\hline
4 & 0100 & 1个 & \textcolor{important}{是} \\
\hline
5 & 0101 & 2个 & 否 \\
\hline
6 & 0110 & 2个 & 否 \\
\hline
7 & 0111 & 3个 & \textcolor{important}{是} \\
\hline
8 & 1000 & 1个 & \textcolor{important}{是} \\
\hline
\end{tabular}
\end{table}

在区间 $[3,8]$ 中,有趣的数字有:4, 7, 8,它们的和为 $4+7+8=19$。

\subsection*{算法分析}

\begin{enumerate}
\item \textcolor{highlight}{\textbf{暴力解法}}\\
最直接的方法是遍历区间 $[l, r]$ 中的每个数,检查其二进制表示中 1 的个数是否为奇数,如果是则累加。

\begin{lstlisting}[language=C++]
long long sum = 0;
for (int i = l; i <= r; i++) {
    if(check(i)) {` // check函数用于判断i是否有趣`
        sum += i;
    }
}
\end{lstlisting}

这种方法的时间复杂度为 $O((r-l+1) \cdot \log r)$,当 $r = 10^9 , l = 1$ 时运算量约为 $10^9$ 级别,会导致超时。

\item \textcolor{highlight}{\textbf{数学规律解法}}
\begin{enumerate}
    \item \textcolor{red}{\textbf{问题拆解:}} 
    将计算区间 $[l, r]$ 的问题转化为计算两个前缀和的差:$f(r) - f(l-1)$,其中 $f(x)$ 表示从 0 到 $x$ 的所有有趣数之和。
    
    所以 用 res(l,r)表示从l 到 r的有趣数字和 , 则有:
    \[
    \textcolor{formula}{res(l,r) = f(r) - f(l-1)}
    \]
    
    \item \textcolor{red}{\textbf{分组讨论:}} 
观察二进制数的规律,可以发现每 4 个连续整数中,有趣数的分布具有周期性:

\begin{center}
\begin{tabular}{cc}
\begin{minipage}{0.45\textwidth}
\centering
\begin{tabular}{|c|c|c|c|}
\hline
\textbf{十进制} & \textbf{二进制} & \textbf{1的个数} & \textbf{是否有趣} \\ 
\hline
0 & 0000 & 0个 & 否 \\
\hline
1 & 0001 & 1个 & \textcolor{important}{是} \\
\hline
2 & 0010 & 1个 & \textcolor{important}{是} \\
\hline
3 & 0011 & 2个 & 否 \\
\hline
\end{tabular}

\end{minipage}
&
\begin{minipage}{0.45\textwidth}
\centering
\begin{tabular}{|c|c|c|c|}
\hline
\textbf{十进制} & \textbf{二进制} & \textbf{1的个数} & \textbf{是否有趣} \\ 
\hline
4 & 0100 & 1个 & \textcolor{important}{是} \\
\hline
5 & 0101 & 2个 & 否 \\
\hline
6 & 0110 & 2个 & 否 \\
\hline
7 & 0111 & 3个 & \textcolor{important}{是} \\
\hline
\end{tabular}

\end{minipage}
\\
& \\
\begin{minipage}{0.45\textwidth}
\centering
\begin{tabular}{|c|c|c|c|}
\hline
\textbf{十进制} & \textbf{二进制} & \textbf{1的个数} & \textbf{是否有趣} \\ 
\hline
8 & 1000 & 1个 & \textcolor{important}{是} \\
\hline
9 & 1001 & 2个 & 否 \\
\hline
10 & 1010 & 2个 & 否 \\
\hline
11 & 1011 & 3个 & \textcolor{important}{是} \\
\hline
\end{tabular}

\end{minipage}
&
\begin{minipage}{0.45\textwidth}
\centering
\begin{tabular}{|c|c|c|c|}
\hline
\textbf{十进制} & \textbf{二进制} & \textbf{1的个数} & \textbf{是否有趣} \\ 
\hline
12 & 1100 & 2个 & 否 \\
\hline
13 & 1101 & 3个 & \textcolor{important}{是} \\
\hline
14 & 1110 & 3个 & \textcolor{important}{是} \\
\hline
15 & 1111 & 4个 & 否 \\
\hline
\end{tabular}

\end{minipage}
\end{tabular}
\end{center}

    观察上述四组数据,可以发现:
    \begin{itemize}
    \item 对于任意四位连续的数字:$...00 , ...01 , ...10 , ...11$ , 其中省略高位部分
    \item 如果 省略的部分中 有奇数个1,那么有趣的数字是:$...01$ 和 $...10$
    \item 如果 省略的部分中 有偶数个1,那么有趣的数字是:$...00$ 和 $...11$
    \item 因为 $\textcolor{formula}{...00} + \textcolor{formula}{...11} = 
    \textcolor{formula}{...01} + \textcolor{formula}{...10}$
    \item 所以无论 省略的部分中 是奇数个1 还是偶数个1,每组四个数中有趣数字之和都等于四个数组之和的一半,也就是:$\textcolor{formula}{\frac{...00 + ...01 + ...10 + ...11}{2}}$
    \item 所以\\ 
$res[0,3] = \frac{0+1+2+3}{2} = 3$ \\
$res[0,7] = \frac{0+1+2+3+4+5+6+7}{2} = 14$ \\
$res[0,11] = \frac{0+1+2+...+11}{2} = 33$ \\
$res[0,x] = \frac{0+1+2+...+x}{2} = 
\frac{x*(x+1)}{4}$ \\
$(x \mod 4 == 3 \quad\text{确保正好按照4个4个
一组分完所有数字})$\\
    \end{itemize}
    
    \item \textcolor{red}{\textbf{求和关系:}} 
    对于可以完整分成的 4 个数一组,该组中有趣数字之和等于所有数字之和的一半。
    \begin{lstlisting}[language=C++]
res += (x + 1) * x / 4; `// \textcolor{keycode}{\textbf{核心公式:完整组的和}}
    \end{lstlisting}
\newpage
    \item \textcolor{red}{\textbf{边界处理:}}
    对于无法组成完整 4 个一组的剩余部分,需要单独判断每个数是否有趣并累加。
    \begin{lstlisting}[language=C++]
bool check(int x) {` // 判断x是否有趣`
    int cnt = 0;
    while (x) {
        if (x & 1) cnt++;
        x >>= 1;
    }
    return (cnt & 1);
}
\end{lstlisting}

\begin{lstlisting}[language=C++]
while (x > 0 && x % 4 != 3) {
    `//从大开始缩小数字,直到这个数字可以完整按照4个4个分`
    if (check(x)) res += x--; `// \textcolor{keycode}{\textbf{处理边界数字}}`
}
    \end{lstlisting}
\end{enumerate}
\end{enumerate}

\subsection*{代码运行步骤展示}
以输入 \texttt{5 17} 为例: 
\textcolor{red}{$res(5 ,17) = f(17) - f(4)$}\\[0.5em]
\textcolor{highlight}{\textbf{1. 计算$f(17)$:}}
\begin{itemize}\setlength{\itemsep}{1pt}\setlength{\parskip}{1pt}
\item 17 mod 4 = 1 ≠ 3,需要边界处理
\item 从17开始向前处理不完整组:
    \begin{itemize}\setlength{\itemsep}{1pt}\setlength{\parskip}{1pt}
    \item $\textcolor{formula}{17}$: 二进制(10001),有$\textcolor{formula}{2}$个1,偶数个,\textcolor{important}{不是有趣的}
    \item $\textcolor{formula}{16}$: 二进制(10000),有$\textcolor{formula}{1}$个1,奇数个,\textcolor{important}{是有趣的},res += 16.
    \item $\textcolor{formula}{15}$: 15 mod 4 = 3,停止边界处理.
    \end{itemize}

\item 此时x = 15,15 mod 4 = 3,可以使用公式( 0-3, 4-7, 8-11, 12-15 都是完整组)
\item 完整组的和 = $\frac{15 * 16}{4} = 60$
\item 所以$f(17) = 16 + 60 = \textcolor{formula}{76}$
\end{itemize}

\textcolor{highlight}{\textbf{2. 计算$f(4)$:}}
\begin{itemize}\setlength{\itemsep}{1pt}\setlength{\parskip}{1pt}
\item 4 mod 4 = 0 ≠ 3,需要边界处理
\item 从4开始向前处理不完整组:
    \begin{itemize}\setlength{\itemsep}{1pt}\setlength{\parskip}{1pt}
    \item $\textcolor{formula}{4}$: 二进制(100),有$\textcolor{formula}{1}$个1,奇数个,\textcolor{important}{是有趣的},res += 4.
    \item $\textcolor{formula}{3}$: 3 mod 4 = 3,停止边界处理.
    \end{itemize}
\item 此时x = 3,3 mod 4 = 3,可以使用公式( 0-3 是完整组)
\item 完整组的和 = $\frac{3 * 4}{4} = 3$
\item 所以$f(4) = 4 + 3 = \textcolor{formula}{7}$
\end{itemize}

\textcolor{highlight}{\textbf{3. 最终结果:}}
$res(5,17) = f(17) - f(4) = 76 - 7 = 69$

\subsection*{拓展思考}
\begin{itemize}
    \item 除了寻找数学规律,这种按位的题目还可以考虑使用\textcolor{red}{数位dp}来解决。
\end{itemize}

\end{document}