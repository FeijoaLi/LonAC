% L1212 : 汉明距离
\documentclass{ctexart}
\ctexset{
    section = {
        titleformat = \raggedright,
        name = {,、},
        number = \chinese{section}
    }
}
\usepackage[a4paper,top=2.54cm,bottom=2.54cm,left=3.18cm,right=3.18cm]{geometry}
\usepackage{amsmath,amsfonts,amssymb,amsthm}
\usepackage{tikz}
\usepackage{multirow}
\usepackage{array}
\usepackage{fancyhdr}
\usepackage{lastpage}
\usepackage{tabularx}
\usepackage{graphicx}
\usepackage{caption}
\usepackage{pifont}  
\usepackage{xcolor}
\usepackage{listings}

\definecolor{lightpink}{RGB}{184,83,182}
\definecolor{codebg}{RGB}{245,245,245}

% 正确设置 listings
\lstset{
    basicstyle=\ttfamily\small,
    language=C++,
    frame=single,
    breaklines=true,
    backgroundcolor=\color{codebg},
    escapeinside=``,
    commentstyle=\color{gray},
    rulecolor=\color{black},
    framesep=5pt,
    xleftmargin=10pt,
    xrightmargin=10pt
}

\pagestyle{fancy}
\fancyhf{}
\cfoot{解码未来AI教育 \quad 编程题解}
\renewcommand{\headrulewidth}{0pt}
\renewcommand{\labelenumii}{\Alph{enumii}.}

\begin{document}

\section*{L1212 : 汉明距离 \ding{78}} 

\subsection*{题目描述}
汉明距离是指\textcolor{lightpink}{\textbf{两个整数对应二进制位不同的数量}}。

例如,如果 $x = 1$(二进制表示为 0001),$y = 4$(二进制表示为 0100),那么它们的汉明距离是 2。

\subsection*{输入输出格式}
\textbf{输入:}一行两个数 $x$ $y$,表示需要计算的两个整数。其中($0 \leq x,y \leq 2^{30}$)\par
\textbf{输出:}$x$ 和 $y$ 的汉明距离。


\vspace{18pt}
\begin{table}[h]
\centering
\begin{tabularx}{\textwidth}{|X|X|}
\hline
\textbf{输入示例} & \textbf{输出示例}     \\    
\hline
1 4 & 2   \\ 
\hline
\end{tabularx}  
\end{table}

\subsection*{样例解释}
\begin{center}
$x = 1_{10} = 0001_2$ \\
$y = 4_{10} = 0100_2$
\end{center}

在二进制下,$x$ 和 $y$ 的表示如下:

\begin{table}[h]
\centering
\begin{tabularx}{0.9\textwidth}{|X|X|X|X|}
\hline
\textbf{位数} & \textbf{x} & \textbf{y} & \textbf{是否相同} \\   
\hline
第1位 & 1 & 0 & \textcolor{red}{不同} \\
\hline
第2位 & 0 & 0 & \textcolor{green!60!black}{相同} \\
\hline
第3位 & 0 & 1 & \textcolor{red}{不同} \\
\hline
第4位 & 0 & 0 & \textcolor{green!60!black}{相同} \\
\hline
\end{tabularx}  
\end{table}

$x$ 和 $y$ 二进制下有两处不同,所以汉明距离为 2。

\subsection*{背景知识}
位运算是对整数在二进制表示下直接对位进行的操作:
\begin{itemize}
\item \textbf{按位与(\&)}:两位都为1时结果为1
\item \textbf{按位或($|$)}:两位至少有一个为1时结果为1  
\item \textbf{按位异或(\^{} / $\oplus$)}:两位不同时结果为1,相同时结果为0
\item \textbf{左移($<<$)}:将所有位向左移动,右边补0
\item \textbf{右移($>>$)}:将所有位向右移动,左边补0
\end{itemize}

\subsection*{算法分析 : 使用位运算计算汉明距离}
\begin{enumerate}
\item \textbf{使用异或运算找出不同位}\\
使用\textcolor{lightpink}{异或($\oplus$)}运算的特性:相同为0,不同为1\\
计算 $x \oplus y$,使得 $x$ 和 $y$ 不同的部分标记为1,相同的部分标记为0。
\begin{lstlisting}
int t = x ^ y; `// 计算异或结果`
\end{lstlisting}

\item \textbf{使用与运算检查最后一位}\\
使用 \textcolor{lightpink}{\&} 运算判断最后一位是否为1:\\
对于 $( t \hspace{4pt} \& \hspace{4pt} 1 )$,如果值为1,则说明二进制下的 $t$ 在最后一位是1(因为只有 1\&1 = 1)。\\
如果值为0,则说明二进制下的 $t$ 在最后一位是0。
\begin{lstlisting}
cnt += t & 1; `// 检查最后一位是否为1`
\end{lstlisting}

\item \textbf{使用右移运算逐位处理}\\
使用\textcolor{lightpink}{右移($>>$)}运算逐位统计1的个数:\\
通过循环处理最后一位,如果是1(意味着此处 $x$ 和 $y$ 不同)则计数器加1,\\
然后将数字右移一位,直到数字变为0。
\end{enumerate}

\vspace{15pt}

\noindent\textbf{\large 示例:计算 $x=1$, $y=4$ 的汉明距离}
\begin{itemize}
\item \textbf{计算 $x \oplus y$}:
\begin{center}
$x = 1_{10} = 0001_2$ \\
$y = 4_{10} = 0100_2$ \\
$x \oplus y = 0101_2$
\end{center}

\item \textbf{统计 $x \oplus y$ 中1的个数}:
\vspace{10pt}
\begin{table}[h]
\centering
\begin{tabularx}{0.95\textwidth}{|>{\centering\arraybackslash}X|>{\centering\arraybackslash}X|>{\centering\arraybackslash}X|}
\hline
\textbf{当前目标值} & \textbf{最后一位数值} & \textbf{计数器数值} \\    
\hline 
0101 & 1(\textcolor{red}{$x$ 和 $y$ 不同}) & 1 \\ 
\hline
010 & 0(\textcolor{green!60!black}{$x$ 和 $y$ 相同}) & 1 \\ 
\hline
01 & 1(\textcolor{red}{$x$ 和 $y$ 不同}) & 2 \\ 
\hline
0 & 0(\textcolor{green!60!black}{$x$ 和 $y$ 相同}) & 2 \\ 
\hline
\end{tabularx}  
\end{table}
\end{itemize}

\subsection*{核心代码总览}
\begin{lstlisting}
int t = x ^ y; `// 计算异或结果`
int cnt = 0;   `//  计数器`
while (t) {    `// 当目标值为0时终止`
    cnt += t & 1; `// 检查最后一位是否为1`
    t >>= 1;      `// 右移一位`
}
\end{lstlisting}

\end{document}