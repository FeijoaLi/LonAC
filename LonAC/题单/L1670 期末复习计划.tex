\documentclass{ctexart}
\usepackage{template}
\usepackage{float}
\usepackage{needspace} 
\usepackage{xcolor}

\begin{document}

\section*{\raggedright 期末复习计划 \ding{78}\ding{78}\ding{78}}

\subsection*{题目描述}

考试周即将到来,你制定了一个为期 $N$ 天的复习计划。在第 $i$ 天,你计划背诵 $a_i$ 个单词。 

现在你有 $Q$ 个询问,每个询问给出两个整数 $L$ 和 $R$,要求你计算从第 $L$ 天到第 $R$ 天(包含第 $L$ 天和第 $R$ 天)之间总共背诵的单词数量。

\subsection*{输入格式}

第一行包含两个整数 $N$ 和 $Q$,表示天数和询问次数。

第二行包含 $N$ 个整数,表示每天背诵的单词数 $a_i$。

接下来 $Q$ 行,每行包含两个整数 $L$ 和 $R$。

\subsection*{输出格式}

输出共 $Q$ 行,每行输出对应时间段 $[L, R]$ 的单词总数。

\subsection*{输入输出样例}

% --- 表格 1 ---
\begin{table}[H]
\centering
\renewcommand{\arraystretch}{1.2} 
\setlength{\tabcolsep}{10pt}

\begin{tabular}{|p{0.45\textwidth}|p{0.45\textwidth}|}
\hline
\textbf{输入} & \textbf{输出 \#1} \\
\hline
\begin{minipage}[t]{\linewidth}
\begin{verbatim}
5 3
10 20 10 5 30
1 3
2 4
1 5
\end{verbatim}
\end{minipage}
\vspace{0.8em}
& 
\begin{minipage}[t]{\linewidth}
\begin{verbatim}
40
35
75
\end{verbatim}
\end{minipage} 
\vspace{0.8em}
\\
\hline
\end{tabular}
\end{table}

\subsection*{算法分析与代码实现}

本题是 \textbf{\textcolor{blue}{前缀和 (Prefix Sum)}} 的经典入门题。我们将从原理分析入手,逐步展示对应的代码实现。

\begin{enumerate}
\item \textbf{暴力法的局限性} \\ 
如果对于每个询问 $(L, R)$,我们都写一个 $for$ 循环来累加求和:
\begin{itemize}
    \item 单次查询的最坏时间复杂度为 $O(N)$。
    \item 总共有 $Q$ 次查询,总时间复杂度为 $O(N \times Q)$。
    \item 题目中 $N, Q \le 10^5$,乘积高达 $10^{10}$,远超计算机 1 秒约 $10^8$ 次运算的限制,会导致 \textbf{\textcolor{red}{TLE (Time Limit Exceeded)}}。
\end{itemize}

\item \textbf{数据类型定义:十年OI一场空,不开long long见祖宗} \\ 
题目中 $a_i \le 10^9$,且 $N \le 10^5$。极端情况下,
区间总和可能达到 $10^5 \times 10^9 = 10^{14}$。

而 $int$ 类型的最大值约为 $2 \times 10^9$。

因此,\textbf{累加数组 $S$ 必须使用 \texttt{\textcolor{orange}{long long}} 类型},否则会发生整型溢出。

\item \textbf{预处理:构造前缀和数组 ($O(N)$)}  \\ 
我们定义一个数组 $S$,其中 $S[i]$ 表示前 $i$ 天背诵的单词总数。
\[ S[i] = a[1] + a[2] + \dots + a[i] \]
在代码中,我们不需要每次都重新从头加,而是利用递推公式:
\[ S[i] = S[i-1] + a[i] \quad (\text{其中 } S[0] = 0) \]
这意味着:\textbf{当前的前缀和 = 前一天的前缀和 + 今天的单词数}。

\Needspace{10\baselineskip}
\begin{lstlisting}[language=C++]
    for (int i = 1; i <= n; i++) {
        long long x;
        cin >> x;
        `// 核心递推:利用 S[i-1] 快速计算 S[i]`
        S[i] = S[i-1] + x;
    }
\end{lstlisting}

\item \textbf{查询:$O(1)$ 响应} \\ 
利用预处理好的 $S$ 数组,区间 $[L, R]$ 的和可以表示为“前 $R$ 天的总数”减去“前 $L-1$ 天的总数”:
\[ \text{Sum}(L, R) = S[R] - S[L-1] \]
这样,无论区间多长,我们都只需要做一次减法即可得到答案,时间复杂度降为 \textbf{O(1)}。

\Needspace{10\baselineskip}
\begin{lstlisting}[language=C++]
    while (q--) {
        int l, r;
        cin >> l >> r;
        `// 利用公式直接输出,无需循环`
        cout << S[r] - S[l-1] << "\n";
    }
\end{lstlisting}

\end{enumerate}

\subsection*{拓展思考}

\subsubsection*{1. 逆向思维:差分 (Difference Array)}
如果题目需求变了:不是“查询”区间和,而是让你“修改”区间(例如:第 $L$ 天到第 $R$ 天,每天多背 $V$ 个单词),最后只询问一次最终的数组,该怎么办?
这时候就需要用到\textbf{\textcolor{blue}{差分数组}}。差分是前缀和的逆运算,可以将区间修改操作从 $O(N)$ 优化到 $O(1)$。

\subsubsection*{2. 动态修改与查询}
如果题目既有大量的“区间修改”,又有大量的“区间查询”,前缀和(修改慢)和差分(查询慢)就都不够用了。这时候我们需要更高级的数据结构,如\textbf{树状数组 (Binary Indexed Tree)} 或 \textbf{线段树 (Segment Tree)},它们能将两种操作都维持在 $O(\log N)$ 的复杂度。

\end{document}