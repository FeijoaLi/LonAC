\documentclass[12pt, a4paper]{ctexart}

% --- 宏包加载 ---
\usepackage[utf8]{inputenc}
\usepackage{amsmath, amsfonts}
\usepackage{geometry}
\usepackage{hyperref}
\usepackage{xcolor}
\usepackage{enumitem}
\usepackage{mdframed}
\usepackage{graphicx}
\usepackage{fancyhdr}
\usepackage{tcolorbox}

% --- 页面设置 ---
\geometry{left=2.5cm, right=2.5cm, top=3cm, bottom=3cm}
\hypersetup{
    colorlinks=true, linkcolor=blue, urlcolor=cyan,
}
\pagestyle{fancy}
\fancyhf{}
\fancyhead[L]{算法竞赛入门与进阶}
\fancyhead[R]{二分算法:红蓝边界模型}
\fancyfoot[C]{\thepage}

% --- 自定义颜色 ---
\definecolor{algored}{RGB}{230, 57, 70}
\definecolor{algoblue}{RGB}{29, 53, 87}
\definecolor{notegh}{RGB}{240, 240, 240}

% --- 标题信息 ---
\title{\textbf{\huge 告别死循环:二分算法的“红蓝区域法”}}
\author{\Large 讲师:李俊飞 (Feijoa\_Li)}
\date{}

\begin{document}

\maketitle

% --- 课程导读 ---
\begin{tcolorbox}[colback=notegh, colframe=gray, title=\textbf{写在前面}]
很多同学在写二分时,经常纠结这几个问题:
\begin{itemize}
    \item $mid$ 到底是加一还是减一?
    \item 循环条件是 $l < r$ 还是 $l \le r$?
    \item 最后的答案到底是 $l$ 还是 $r$?
\end{itemize}
今天我们要介绍的\textbf{“双开区间(红蓝区域)法”},能够让你不再纠结这些细节,用一种极其直观的物理模型,一劳永逸地解决所有二分问题。
\end{tcolorbox}

\tableofcontents
\newpage

\section{二分算法核心:红蓝区域法与双开区间模型}

本节将介绍一种在信奥竞赛中极其实用的二分思维模型。它抛弃了复杂的边界判断,将二分问题抽象为物理上的“领地夹逼”过程。

\subsection{直观模型:红蓝颜色的对抗}

在做二分题时,请摒弃“我在找数字 5”这种具体数值的思维,转而建立\textbf{“寻找性质分界线”}的全局观。

假设我们面对一个有序序列或数轴区间。尽管未知具体的数值,但我们可以根据题目给出的判断条件(Check 函数),将整个定义域划分为两种颜色的领地:

\begin{itemize}
    \item \textbf{\textcolor{red}{红色区域 (Red)}}:左侧\textbf{不满足}条件的位置,视作红色(False/0)。
    \item \textbf{\textcolor{blue}{蓝色区域 (Blue)}}:右侧\textbf{满足}条件的位置,视作蓝色(True/1)。
\end{itemize}

$$ \underbrace{\textcolor{red}{\blacksquare \ \blacksquare \ \blacksquare \ \dots \ \blacksquare}}_{\text{不满足条件 (Red)}} \quad \Big| \quad \underbrace{\textcolor{blue}{\blacksquare \ \blacksquare \ \dots \ \blacksquare \ \blacksquare}}_{\text{满足条件 (Blue)}} $$

我们的任务变得非常单纯:\textbf{定位红蓝两种颜色交界的那条唯一“缝隙”。}

\subsection{执行流程:双指针哨兵机制}

为了精准捕获这条分界线,我们派出两名“哨兵”指针——$l$(红军大将)和 $r$(蓝军大将),采用\textbf{双开区间 $(l, r)$} 的策略进行搜索。

\subsubsection*{1. 定义领地(初始化)}
为了防止越界并覆盖所有可能解,初始化时指针不应站在具体的格子上,而应站在\textbf{数组范围之外}:
\begin{itemize}
    \item \textbf{$l$ (Left)}:初始位于第一个元素的\textbf{左侧}(虚拟下标 $-1$)。它承诺:\textit{“我所站立及我左侧的地方,全都是红色的。”}
    \item \textbf{$r$ (Right)}:初始位于最后一个元素的\textbf{右侧}(虚拟下标 $N$)。它承诺:\textit{“我所站立及我右侧的地方,全都是蓝色的。”}
\end{itemize}

\subsubsection*{2. 领地扩张(迭代过程)}
只要 $l$ 和 $r$ 之间还有未探索的未知区域(即 $l + 1 < r$),探索就继续:
\begin{enumerate}
    \item 计算中点 $mid$。
    \item 派侦察兵查看 $mid$ 的颜色:
    \begin{itemize}
        \item 若 $mid$ 是\textbf{\textcolor{blue}{蓝色}}:说明 $mid$ 及其右侧皆为蓝军领地。于是 $r$ 左移至 $mid$ 处守卫新边界。
        \item 若 $mid$ 是\textbf{\textcolor{red}{红色}}:说明 $mid$ 及其左侧皆为红军领地。于是 $l$ 右移至 $mid$ 处守卫新边界。
    \end{itemize}
\end{enumerate}
\textbf{核心原则:} $l$ 永远只在红色区域移动,$r$ 永远只在蓝色区域移动,绝不越界。

\subsubsection*{3. 完美的结局(循环终止)}
当 $l$ 和 $r$ 终于紧挨在一起(即 $l + 1 == r$)时,循环结束。此时的状态是唯一且确定的:
\begin{itemize}
    \item $l$ 指向\textbf{最后一个红色元素}。
    \item $r$ 指向\textbf{第一个蓝色元素}。
    \item 分界线就在 $l$ 和 $r$ 之间。
\end{itemize}

\begin{center}
    \fcolorbox{black}{gray!10}{
    \parbox{0.9\linewidth}{
    \centering
    \textbf{【答案提取指南】}\\
    题目问“最大的不满足数” $\rightarrow$ 输出 $l$\\
    题目问“最小的满足数” $\rightarrow$ 输出 $r$\\
    \textit{无需纠结 $mid \pm 1$,答案天然就在 $l$ 或 $r$ 手中。}
    }
    }
\end{center}

\subsection{数值细节:中点计算的避坑指南}

在计算 $mid$ 时,教科书常见的 $(l+r)/2$ 写法存在隐患。建议养成使用\textbf{防溢出写法}的肌肉记忆。

\begin{quote}
    \textbf{推荐公式:} $mid = l + (r - l) / 2$
\end{quote}

该公式包含两层重要的工程意义:
\begin{enumerate}
    \item \textbf{防止整型溢出 (Integer Overflow)}:
    若 $l, r$ 均为接近 $2 \times 10^9$ 的大整数,直接相加会造成数据溢出(变为负数),导致程序崩溃。先减后加则永远安全。
    
    \item \textbf{向下的引力 (Floor Division)}:
    C++ 的整数除法默认向下取整。你可以理解为 $mid$ 总是受到 $l$ 的“引力”。在 $l$ 和 $r$ 相邻之前,$mid$ 永远不会触碰到 $r$,这从数学上彻底杜绝了死循环的可能性。
\end{enumerate}





\section{进阶应用:二分答案——“猜”出来的最优解}

二分答案是算法竞赛中二分思想最高频的考点,也是一种**思维方式的逆转**。

\subsection{从“直接求解”到“试探验证”}

很多时候,题目会要求我们求一个“最优值”(例如:求最少需要多少时间、求最大的最小距离等)。
\begin{itemize}
    \item \textbf{正向思维(求解):} 直接计算出最优解 $X$。这往往非常困难,可能涉及复杂的动态规划或枚举,时间复杂度难以接受。
    \item \textbf{逆向思维(验证):} 我们不去直接算 $X$,而是先\textbf{猜}一个答案 $mid$,然后问自己:\textit{“如果答案是 $mid$,在这个限制下,我能满足题目的要求吗?”}
\end{itemize}

这种方法被称为\textbf{“答案的判定性转化”}。只要“验证一个答案是否可行”比“直接求出答案”要容易得多(例如验证通常只需要 $O(N)$ 的贪心),且答案具有\textbf{单调性},我们就可以用二分来地毯式搜索最优解。

\subsection{两类经典模型:最大化与最小化}

二分答案题目通常有着非常鲜明的提问方式,对应着红蓝模型中不同的寻找方向:

\subsubsection{1. 最小值最大化 (Maximize the Minimum)}
\textbf{场景:} “求 $N$ 个点之间最近距离的最大值”、“求让最穷的人分到的钱尽可能多”。

\textbf{单调性分析:}
假设我们验证的数值为 $x$(例如距离)。
\begin{itemize}
    \item 如果 $x$ 满足条件(即距离为 $x$ 时能放得下),那么比 $x$ 更小的值($x-1, x-2 \dots$)肯定更能放得下(条件更宽松)。
    \item 我们的目标是找到\textbf{最后一个满足条件}的值。
\end{itemize}

\textbf{红蓝分布:}
$$ \underbrace{\textcolor{blue}{\text{可行 (True)}}}_{\text{小数值}} \ \dots \ \textcolor{blue}{\text{可行}} \ \Big| \ \textcolor{red}{\text{不可行}} \ \dots \ \underbrace{\textcolor{red}{\text{不可行 (False)}}}_{\text{大数值}} $$
\textbf{搜索目标:} 蓝色区域的右边界($r$ 或 $l$,取决于具体写法,双开区间法中取 $l$)。

\subsubsection{2. 最大值最小化 (Minimize the Maximum)}
\textbf{场景:} “求耗时最长的任务所需的最少时间”、“求段和最大值的最小值”。

\textbf{单调性分析:}
假设我们验证的数值为 $x$(例如时间限制)。
\begin{itemize}
    \item 如果 $x$ 满足条件(即限时 $x$ 秒能做完),那么比 $x$ 更大的值($x+1, x+2 \dots$)肯定更能做完(时间更充裕)。
    \item 我们的目标是找到\textbf{第一个满足条件}的值。
\end{itemize}

\textbf{红蓝分布:}
$$ \underbrace{\textcolor{red}{\text{不可行 (False)}}}_{\text{小数值}} \ \dots \ \textcolor{red}{\text{不可行}} \ \Big| \ \textcolor{blue}{\text{可行}} \ \dots \ \underbrace{\textcolor{blue}{\text{可行 (True)}}}_{\text{大数值}} $$
\textbf{搜索目标:} 蓝色区域的左边界($r$ 或 $l$,取决于具体写法,双开区间法中取 $r$)。

\subsection{解题四步走}
遇到这就两类问题,按以下步骤思考:
\begin{enumerate}
    \item \textbf{确定解空间}:答案的可能的最小值 $L$ 和最大值 $R$ 是多少?
    \item \textbf{设计 Check 函数}:给你一个固定的值 $mid$,你能否在 $O($能过$)$ 时间内判断它是否合法?(通常配合贪心算法)。
    \item \textbf{画出红蓝条}:判断单调性方向,大值可行还是小值可行?
    \item \textbf{套用模板}:使用双开区间模板,根据 Check 的结果收缩边界。
\end{enumerate}

\section{数学进阶:实数域上的二分}

当我们的战场从离散的整数点转移到连续的实数轴时(例如计算几何中的线段交点、物理学中的抛物线时间计算),二分的“红蓝边界”依然存在,但搜索策略需要进行重要的调整。

\subsection{精度与效率的博弈}

在实数域中,我们通常寻找的是一个近似解。传统教科书常教授 \texttt{while (r - l > eps)} 的写法,但在竞赛实战中,这往往是一个陷阱。

\subsubsection{浮点数的陷阱:为什么 \texttt{eps} 不靠谱?}
设定一个固定的精度阈值(如 `eps = 1e-7`)存在两个严重隐患:

\begin{enumerate}
    \item \textbf{\textcolor{red}{精度稀疏问题 (Precision Loss)}}:
    计算机中的浮点数分布不是均匀的。随着数值绝对值的增大,可表示的浮点数之间的间隙(Gap)也会变大。
    
    \textit{例如:} 当答案高达 $10^{12}$ 时,相邻两个 `double` 类型的数值差可能已经超过了你设定的 `1e-7`。这意味着 $l$ 和 $r$ 即使紧挨着,差值也永远大于 `eps`,导致程序陷入\textbf{死循环}。
    
    \item \textbf{收敛速度的不确定性}:
    在某些构造的极端数据下,区间收缩到 `eps` 附近所需的迭代次数可能超出预期,导致超时(TLE)。
\end{enumerate}

\subsection{黄金策略:固定迭代次数法}

为了彻底规避精度死循环和超时问题,我强烈建议在实数二分中采用\textbf{“固定循环次数”}的策略。

我们知道,每一次二分都会将区间长度减半。假设初始区间长度为 $L$,经过 $K$ 次迭代后,区间长度变为:
$$ L_{final} = L \times \left(\frac{1}{2}\right)^K $$

\textbf{为什么选 100 次?}
若我们循环 $100$ 次,精度缩减倍率为 $2^{-100} \approx 10^{-30}$。
\begin{itemize}
    \item 哪怕初始区间是整个宇宙的大小($10^{26}$ 米),100 次二分后误差也小于一个原子核。
    \item 这个精度远远超过了 `double` 甚至 `long double` 的有效位数。
\end{itemize}

\begin{mdframed}[linecolor=blue!50, linewidth=2pt, roundcorner=5pt, backgroundcolor=blue!5]
\textbf{【最佳实践指南】}

在处理实数二分时,请遵循以下步骤:
\begin{enumerate}
    \item \textbf{摒弃 While}:不要写 `while (r - l > eps)`。
    \item \textbf{使用 For 循环}:直接根据题目精度要求,写一个固定次数的循环。
    \begin{itemize}
        \item 一般精度要求:循环 \textbf{100 次}。
        \item 极高精度要求(Long Double):循环 \textbf{200 次}。
    \end{itemize}
    \item \textbf{无脑收缩}:在循环体内,直接令 $mid = (l+r)/2$。如果 $mid$ 满足条件,则 $l=mid$(或 $r=mid$),无需考虑 $+1/-1$ 的整数边界问题。
\end{enumerate}
这种写法时间复杂度恒定为 $O(100)$,在保证绝对精度的同时,彻底杜绝了死循环的风险。
\end{mdframed}

\section{进阶武器:三分查找与单峰极值}

当函数的性质不再是简单的“单调”(一路升或一路降),而是变成了\textbf{“单峰”(Unimodal)}——即先增后减(像一座山峰)或先减后增(像一个山谷)时,二分法就失效了。

此时,我们需要引入更高级的策略——\textbf{三分查找 (Ternary Search)}。

\subsection{双探针逻辑:如何在黑暗中爬山?}

假设我们要寻找函数 $f(x)$ 的\textbf{最大值}(山顶),且已知 $f(x)$ 是开口向下的单峰函数。

二分法之所以行不通,是因为只看中间一个点,我们不知道自己是在“上坡”还是“下坡”(除非求导,但离散函数无法求导)。因此,我们需要\textbf{两个探针}来测定地势的走向。

我们在区间 $(l, r)$ 内部设立两个采样点 $m_1$ 和 $m_2$,将区间大致三等分:
$$ m_1 = l + \frac{r - l}{3}, \quad m_2 = r - \frac{r - l}{3} $$
显然有位置关系:$l < m_1 < m_2 < r$。



通过比较这两个探针的高度 $f(m_1)$ 和 $f(m_2)$,我们可以安全地\textbf{“砍掉”}一段绝对不可能包含山顶的区间:

\begin{itemize}
    \item \textbf{情形一:} $f(m_1) < f(m_2)$ \\
    \textbf{地势分析:} 右边的 $m_2$ 比左边的 $m_1$ 高。
    \textbf{推论:} 既然我们要找最高点,且山只有一个顶,那么最高点绝不可能在 $m_1$ 的左侧(否则 $m_1$ 处于下坡阶段,但这违背了$m_2$更高的事实,或意味着有两个峰)。
    \item[] $\Rightarrow$ \textbf{操作:} 极值一定在 $(m_1, r)$ 中。安全\textbf{\textcolor{red}{舍弃左侧区间}},令 $l = m_1$。

    \item \textbf{情形二:} $f(m_1) > f(m_2)$ \\
    \textbf{地势分析:} 左边的 $m_1$ 比右边的 $m_2$ 高。
    \textbf{推论:} 同理,最高点绝不可能在 $m_2$ 的右侧(否则 $m_2$ 之后还在升高,那 $m_1$ 到 $m_2$ 就成了下坡,形成凹陷,违背单峰性质)。
    \item[] $\Rightarrow$ \textbf{操作:} 极值一定在 $(l, m_2)$ 中。安全\textbf{\textcolor{blue}{舍弃右侧区间}},令 $r = m_2$。
\end{itemize}

\subsection{工程实现:离散域的“暴力收尾”}

在实数域上三分很简单(使用固定循环次数即可),但在\textbf{整数域}上三分极其容易写挂。

当 $l$ 和 $r$ 距离很近时(例如 $r - l < 3$),$m_1$ 和 $m_2$ 可能会计算重合,或者陷入死循环。为了在考场上万无一失,建议采用\textbf{“三分缩范围 + 暴力定答案”}的混合策略:

\begin{enumerate}
    \item \textbf{主循环收缩:} 当区间长度比较大时(例如 $l + 2 < r$),正常执行三分逻辑,不断缩小范围。
    \item \textbf{暴力扫尾:} 当循环结束时,区间内只剩下 3 到 5 个整数点。此时不要再纠结边界了,直接一个 \texttt{for} 循环遍历 $[l, r]$ 这一小段,算出最大值即可。
\end{enumerate}

这种写法既保留了 $O(\log N)$ 的高效,又完全规避了小区间内的数学边界陷阱。

\section{0/1 分数规划:化除为减的魔法}

这是二分思想在代数领域最惊艳的应用之一。它解决了一类让人头疼的“最优比率”问题。

\subsection{问题的本质}
假设你有一堆物品,每个物品有两个属性:价值 $a_i$ 和 代价 $b_i$。我们需要从中选出一组物品(集合 $S$),在满足题目特定约束(比如选 $k$ 个,或者连通)的前提下,让总价值与总代价的比值最大:
\begin{equation}
\text{Maximize } \quad V = \frac{\sum_{i \in S} a_i}{\sum_{i \in S} b_i}
\end{equation}

直接求这个比率非常困难,因为分子和分母都会随着选择变化,且互相牵制。

\subsection{核心推导:不等式的变形}
我们换个思路:与其直接求最大值,不如去**猜**这个比率是多少。

假设我们猜测最终答案能达到 $mid$。
如果存在一种选择方案 $S$,使得比率 $\ge mid$,那么数学上意味着:
$$ \frac{\sum a_i}{\sum b_i} \ge mid $$
我们将分母乘过去(注意 $b_i$ 通常为正,不改变符号):
$$ \sum a_i \ge mid \cdot \sum b_i $$
移项,将含 $mid$ 的项挪到左边:
$$ \sum a_i - \sum (mid \cdot b_i) \ge 0 $$
合并求和符号:
\begin{equation}
\sum_{i \in S} (a_i - mid \cdot b_i) \ge 0
\end{equation}

\subsection{算法流程:新权值的诞生}
通过上面的变形,我们发现了一个惊人的事实:
\begin{quote}
    \textbf{如果我们把每个物品的新权值定义为 $w_i = a_i - mid \cdot b_i$,那么问题就变成了:能否选出一组物品,使得它们的新权值之和非负?}
\end{quote}

这彻底消除了讨厌的除法!现在的解题套路变成了标准的三步走:

\begin{enumerate}
    \item \textbf{二分答案}:在实数域上二分比率 $mid$(建议迭代 100 次)。
    \item \textbf{重置权值}:在 `check(mid)` 函数中,将所有物品的权值重置为 $w_i = a_i - mid \cdot b_i$。
    \item \textbf{判定可行性}:
    \begin{itemize}
        \item 如果题目要求选 $K$ 个:那就贪心地选 $w_i$ 最大的 $K$ 个,看和是否 $\ge 0$。
        \item 如果题目要求生成树:那就用 $w_i$ 跑最大生成树,看树权是否 $\ge 0$。
        \item 如果题目要求找环:那就看图中是否存在正环。
    \end{itemize}
    如果最大权值和 $\ge 0$,说明 $mid$ 定小了(或者刚好),答案在右边;否则答案在左边。
\end{enumerate}

\subsection{总结}
0/1 分数规划的本质,就是通过二分答案,将非线性的\textbf{“比率最大化”}问题,转化为线性的\textbf{“判定权值和非负”}问题。掌握了这个转化,你就能把以前学过的所有算法(贪心、DP、最短路)都套上“最优比率”的外壳。

\section{高阶模型:维度变换与 WQS 二分}

当题目要求“恰好选 $K$ 个物品”时,朴素的 DP 往往需要一维状态来记录个数,导致复杂度飙升到 $O(N^2)$。

\textbf{WQS 二分(又称 Aliens Trick / 带权二分)}提供了一种降维打击的思路:通过引入一个“隐形的手”,把刚性的数量限制转化为弹性的价格调节。

\subsection{直观理解:用“价格”控制“销量”}

\subsubsection{硬约束 vs. 软调节}
假设你在经营一家商店,题目要求你\textbf{“必须卖出恰好 $K$ 个苹果”}以获得最大利润。这很难,因为你得时刻盯着销量。

但如果你换一种思路:
\begin{enumerate}
    \item 取消“恰好 $K$ 个”的限制,允许随便卖(这就变成了简单的贪心或 $O(N)$ DP)。
    \item 但是,每卖出一个苹果,必须向系统缴纳 \textbf{$\lambda$ 元的“手续费”}。
\end{enumerate}

这就引入了博弈:
\begin{itemize}
    \item 如果手续费 $\lambda$ \textbf{太贵}(例如 100 万/个),为了利润最大化,你可能一个都不卖(选出的个数 $< K$)。
    \item 如果手续费 $\lambda$ \textbf{太便宜}(甚至倒贴钱),你会疯狂卖苹果(选出的个数 $> K$)。
    \item \textbf{中间必然存在一个神奇的价格 $\lambda_{mid}$},使得你在追求利润最大化的无拘无束的决策中,\textbf{“自愿”}且\textbf{“恰好”}卖出了 $K$ 个苹果!
\end{itemize}

这就是 WQS 二分的本质:\textbf{二分这个手续费 $\lambda$,直到最优策略自动选出 $K$ 个物品。}

\subsection{适用前提:边际效应递减(凸性)}

不是所有问题都能用这招。WQS 二分生效的前提是问题的解具有\textbf{凸性(Convexity)}。

通俗地说,就是\textbf{“边际效应递减”}:
\begin{quote}
    当你选第 1 个物品时,获利巨大;选第 2 个时,获利稍少……选第 100 个时,获利微乎其微。
\end{quote}
如果我们画出图表:横坐标是“选择物品的个数 $x$”,纵坐标是“最大收益 $f(x)$”。
如果 $f(x)$ 的图像是一个\textbf{上凸的形状}(斜率逐渐变小),那么我们就可以用 WQS 二分。


\subsection{几何意义:斜率切线法}

\subsubsection{为什么要减去 $\lambda \cdot x$?}
我们在无限制的情况下求解的最大利润,实际上是在求:
$$ g(\lambda) = \max_{x} \{ f(x) - \lambda \cdot x \} $$
移项变形一下:
$$ f(x) = \lambda \cdot x + g(\lambda) $$
这看起来像什么?$y = kx + b$!
\begin{itemize}
    \item $f(x)$ 是 $y$ 坐标(总收益)。
    \item $x$ 是 $x$ 坐标(选择个数)。
    \item $\lambda$ 是\textbf{斜率}。
    \item $g(\lambda)$ 是\textbf{截距}。
\end{itemize}

当我们固定斜率 $\lambda$ 并试图最大化截距 $g(\lambda)$ 时,几何上等价于:\textbf{用一条斜率为 $\lambda$ 的直线去切 $f(x)$ 的图像。} 切点对应的横坐标,就是在该手续费下的最优选择个数。

\subsection{解题三步走}

\begin{enumerate}
    \item \textbf{二分斜率 $\lambda$}:
    范围通常在 $[- \text{inf}, \text{inf}]$。
    
    \item \textbf{带权 DP / 贪心 (Check 函数)}:
    在 `check(mid)` 中,忽略个数 $K$ 的限制,但把每个物品的权值减去 $mid$(手续费)。
    跑一遍基础算法,算出此时的\textbf{最优总价值} $val$ 和\textbf{最优选取的物品数} $cnt$。
    
    \item \textbf{调整边界与还原答案}:
    \begin{itemize}
        \item 如果 $cnt \ge K$:说明手续费太便宜了,或者刚好。我们记录答案,并尝试增加手续费(让 $\lambda$ 变大,$cnt$ 变小),即 $l = mid + 1$(或根据写法调整)。
        \item 如果 $cnt < K$:说明手续费太贵了,需要降价。
    \end{itemize}
    
    \textbf{最终答案还原公式:}
    $$ \text{真实答案} = \text{带权最优值} + K \cdot \text{最终的手续费} $$
    即:$f(K) = g(\lambda_{ans}) + K \cdot \lambda_{ans}$。
\end{enumerate}

\section{整体二分(Parallel Binary Search):批发式的高效分治}

这是二分算法的最终形态。它解决的是这样一种困境:
\begin{quote}
    你有 $10$ 万个询问,每个询问都可以用二分答案解决。但是,每次 `check` 函数都要跑一遍 $O(N)$ 的操作。如果一个个做(零售),总复杂度是 $O(Q \cdot N \cdot \log V)$,直接超时。
\end{quote}

整体二分的核心思想是**“批发”**:既然所有询问都在同一个答案值域 $[Min, Max]$ 内找答案,不如把它们捆在一起,通过一次遍历,同时更新所有人的进度。

\subsection{核心逻辑:数据流的分拣}

我们要定义一个递归过程 \texttt{solve(L, R, QueryList)},含义是:\textit{“我知道这堆询问 QueryList 的最终答案,一定都在数值区间 $[L, R]$ 里,请帮我进一步缩小范围。”}

这个过程就像在一个巨大的\textbf{分拣工厂}里工作:

\begin{enumerate}
    \item \textbf{设定中点(切割):}
    取值域中点 $mid = (L + R) / 2$。我们将把所有操作和询问按照“是否 $\le mid$”这一标准分为两类。

    \item \textbf{发放物资(贡献计算):}
    遍历时间轴上所有的修改操作(例如“加入一个数 $x$”):
    \begin{itemize}
        \item 如果 $x \le mid$:说明这个数属于“左半边”的较小值。它会对当前的判定产生影响(比如贡献了 $1$ 个排名)。我们将它加入数据结构(如树状数组)。
        \item 如果 $x > mid$:说明这个数太大了,暂时不考虑,之后会归入右半边递归。
    \end{itemize}

    \item \textbf{客户分流(询问判定):}
    遍历 \texttt{QueryList} 中的每一个询问(例如“查区间第 $K$ 小”):
    查询数据结构,看在只考虑 $\le mid$ 的数值时,当前区间里有多少个数(记为 $cnt$)。
    \begin{itemize}
        \item \textbf{\textcolor{blue}{左侧分流 (Happy List)}}:
        若 $cnt \ge K$:说明答案 $\le mid$(光靠左半边的数就够凑齐 $K$ 个了)。
        $\Rightarrow$ 该询问归入 \texttt{left\_list},继续在 $[L, mid]$ 里找。
        
        \item \textbf{\textcolor{red}{右侧分流 (Hungry List)}}:
        若 $cnt < K$:说明答案 $> mid$(左半边的数不够,还需要在右半边再找 $K - cnt$ 个)。
        $\Rightarrow$ \textbf{关键操作:} 令 $K \leftarrow K - cnt$(扣除已有的贡献),然后该询问归入 \texttt{right\_list},继续在 $[mid+1, R]$ 里找。
    \end{itemize}

    \item \textbf{还原现场(Rollback):}
    \textbf{这是极其重要的一步!} 在进入下一层递归之前,必须把步骤 2 中加入数据结构的操作全部撤销(清空树状数组)。因为下一层递归是独立的,不能受当前层的残余影响。

    \item \textbf{递归分治:}
    分别调用 \texttt{solve(L, mid, left\_list)} 和 \texttt{solve(mid+1, R, right\_list)}。
\end{enumerate}

\subsection{复杂度分析:为什么它快?}
虽然看起来很复杂,但我们算一算账:
\begin{itemize}
    \item \textbf{深度:} 答案的值域为 $V$,递归层数只有 $\log V$ 层。
    \item \textbf{广度:} 在每一层递归中,每个修改操作和每个询问只会被处理一次(要么去左边,要么去右边,类似归并排序)。
\end{itemize}
配合 $O(\log N)$ 的数据结构,总的时间复杂度为 $O((N + Q) \log N \log V)$。这比单独二分快了整整 $N$ 倍!

\begin{mdframed}[linecolor=orange, backgroundcolor=orange!5]
\textbf{【总结:整体二分的三个特征】}
\begin{enumerate}
    \item \textbf{答案可二分}:单个询问具有单调性。
    \item \textbf{修改独立}:操作之间互不影响,可以按值域划分。
    \item \textbf{贡献可加}:判定时可以把一部分贡献先扣除(如 $K \leftarrow K - cnt$),去子问题里找剩下的。
\end{enumerate}
\end{mdframed}

\section{二分核心知识点全景汇总}

\begin{table}[h]
\centering
\renewcommand{\arraystretch}{1.2}
\begin{tabular}{@{}llll@{}}
\toprule
\textbf{层次} & \textbf{知识点} & \textbf{核心逻辑} & \textbf{典型应用} \\ \midrule
基础 & 整数二分 & 双开边界 $l+1 < r$ & 有序数组定位 \\
基础 & 实数二分 & 固定迭代 100 次 & 几何/计算几何 \\
核心 & 二分答案 & 判定性转化 & 资源分配/最大最小化 \\
进阶 & 三分查找 & 三等分点采样对比 & 单峰函数求极值 \\
进阶 & 分数规划 & 权值重定义为 $a-kb$ & 效益比最优化 \\
高阶 & WQS 二分 & 凸优化与代价抵消 & 带约束的最优选法 \\
高阶 & 整体二分 & 全局询问分治 & 动态区间第 K 大 \\ \bottomrule
\end{tabular}
\end{table}

\section{结语:二分的灵魂是“舍弃”}
二分查找之所以高效,是因为它每一步都在果断地舍弃一半的解空间。在学习过程中,大家应重点体会这种“分而治之”的确定性。希望本讲义能帮助大家彻底掌握二分算法,在竞赛中做到边界不乱、逻辑清爽。

\end{document}