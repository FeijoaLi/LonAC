\documentclass{ctexart}
\ctexset{
    section = {
        titleformat = \raggedright,
        name = {,、},
        number = \chinese{section}
    }
}

\usepackage[a4paper,top=2.54cm,bottom=2.54cm,left=3.18cm,right=3.18cm]{geometry}
\usepackage{eso-pic,graphicx}
\usepackage{amsmath,amsfonts,amssymb,amsthm}
\usepackage{tikz}
\usepackage{multirow}
\usepackage{array}
\usepackage{fancyhdr}
\usepackage{lastpage}
\usepackage{tabularx}
\usepackage{graphicx}
\usepackage{caption}
\usepackage{pifont}  
\usepackage{listings}
\usepackage{xcolor}
\usepackage{hyperref}

\definecolor{highlightpink}{RGB}{255,0,180}  % 粉红色
\definecolor{codebg}{RGB}{245,245,245}       % 代码背景色
\definecolor{lightgray}{gray}{0.8}           % 浅灰色(替代gray!20)
\definecolor{commentcolor}{RGB}{0,128,0}     % 注释颜色(深绿色)
\definecolor{highlight1}{RGB}{255,50,50}     % 红色高亮
\definecolor{highlight2}{RGB}{0,100,200}     % 蓝色高亮
\definecolor{highlight3}{RGB}{150,0,150}     % 紫色高亮


\AddToShipoutPictureFG{%
  \AtPageCenter{%
    \begin{tikzpicture}[remember picture,overlay]
      \node[opacity=0.1] at (0,0) {\includegraphics[width=0.8\paperwidth,height=0.6\paperheight,keepaspectratio]{sy.png}};
    \end{tikzpicture}%
  }%
}


% 设置超链接样式
\hypersetup{
    colorlinks=true,
    linkcolor=blue,
    filecolor=magenta,      
    urlcolor=blue,
    pdftitle={LeetCode 2654 题解},
    pdfauthor={Feijoa_Li}
}



% 设置 listings 样式
\lstset{
basicstyle=\ttfamily\small,
language=C++,
frame=single,
breaklines=true,
backgroundcolor=\color{codebg},
escapeinside=``,
commentstyle=\color{commentcolor},
keywordstyle=\color{blue},
stringstyle=\color{red},
rulecolor=\color{black},
framesep=5pt,
xleftmargin=10pt,
xrightmargin=10pt
}

\title{\href{https://leetcode.cn/problems/minimum-number-of-operations-to-make-all-array-elements-equal-to-1/}{LeetCode 2654. 使数组所有元素变成 1 的最少操作次数}}
\author{Feijoa\_Li}
\date{}

\begin{document}

\maketitle

\section{题目描述}
给你一个下标从 0 开始的 正 整数数组 \texttt{nums} 。你可以对数组执行以下操作 任意 次:

选择一个满足 $0 \leq i < n - 1$ 的下标 $i$ ,将 \texttt{nums[i]} 或者 \texttt{nums[i+1]} 两者之一替换成它们的最大公约数。

请你返回使数组 \texttt{nums} 中所有元素都等于 1 的 最少 操作次数。如果无法让数组全部变成 1 ,请你返回 -1 。

两个正整数的最大公约数指的是能整除这两个数的最大正整数。

\section{解题思路}
本题的关键在于\textcolor{highlight1}{如何生成第一个1},因为一旦有了一个1,就可以通过它把相邻的元素逐个变成1。

\textcolor{highlight1}{\textbf{核心观察:}}
\begin{itemize}
    \item 如果数组中已经存在1,那么只需要将其他非1元素逐个变成1,操作次数为 $n - count(1)$
    \item 如果数组中不存在1,我们需要通过操作生成第一个1
    \item 生成1的最短方式:找到最短的连续子数组,使得这些数的gcd为1
\end{itemize}

\textcolor{highlight1}{\textbf{算法步骤:}}
\begin{enumerate}
    \item 检查数组中是否已有1,如果有则直接返回 $n - count(1)$
    \item 寻找最短的连续子数组使得gcd为1,设最短长度为 $minLen$
    \item 如果找不到这样的子数组,返回-1
    \item 否则,总操作次数为:$n + minLen - 2$
\end{enumerate}

\textcolor{highlight3}{\textbf{数学解释:}}
\begin{itemize}
    \item 生成第一个1需要 $minLen - 1$ 次操作(在最短gcd为1的子数组上操作)
    \item 用这个1感染其他 $n - 1$ 个元素需要 $n - 1$ 次操作
    \item 总操作次数:$(minLen - 1) + (n - 1) = n + minLen - 2$
\end{itemize}

\section{代码实现}
\begin{lstlisting}[language=c++]
class Solution {
public:
    const int inf = 1e9;
    int minOperations(vector<int>& nums) {
        int cnt = inf;
        `\textcolor{highlight1}{// 计算从位置x开始,需要多少次操作能得到gcd=1}`
        auto cul = [&](int x) {
            int GCD = nums[x];
            int cnt = 0;
            for (int i = x + 1; i < nums.size() && GCD != 1; i++){
                GCD = gcd(GCD, nums[i]);
                cnt++;
            }
            return (GCD == 1 ? cnt : inf);
        };

        `\textcolor{highlight1}{// 寻找生成1的最短距离}`
        for (int i = 0; i < nums.size(); i++) cnt = min(cnt, cul(i));
        if(cnt == inf)return -1;
        
        `\textcolor{highlight1}{// 总操作次数 = 生成第一个1的操作 + 用1感染其他元素的操作}`
        int res = nums.size() + cnt - 1;
        `\textcolor{highlight1}{// 如果已有1,直接计算感染操作}`
        if(find(nums.begin(),nums.end(),1) != nums.end())
            res = nums.size() - count(nums.begin(),nums.end() , 1);
        return res;
    }
};
\end{lstlisting}

\section{复杂度分析}
\begin{itemize}
    \item \textbf{时间复杂度:} $O(n^2)$,其中 $n$ 是数组长度。最坏情况下需要检查所有子数组。
    \item \textbf{空间复杂度:} $O(1)$,只使用了常数级别的额外空间。
\end{itemize}

\section{算法优化}
实际上,我们可以通过预处理来优化gcd的计算,但考虑到 $n \leq 50$,当前的解法已经足够高效。

\textcolor{highlight1}{\textbf{关键洞察:}}
\begin{itemize}
    \item 问题的核心是找到生成第一个1的最短路径
    \item 一旦有了1,剩下的操作就是线性的
    \item 使用gcd的单调性:连续子数组的gcd随着长度增加是非递增的, 从而利用线段树优化到 $O(nlogn)$
\end{itemize}

\end{document}