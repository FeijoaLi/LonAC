\documentclass{ctexart}
\ctexset{
    section = {
        titleformat = \raggedright,
        name = {,、},
        number = \chinese{section}
    }
}

\usepackage[a4paper,top=2.54cm,bottom=2.54cm,left=3.18cm,right=3.18cm]{geometry}
\usepackage{eso-pic,graphicx}
\usepackage{amsmath,amsfonts,amssymb,amsthm}
\usepackage{tikz}
\usepackage{multirow}
\usepackage{array}
\usepackage{fancyhdr}
\usepackage{lastpage}
\usepackage{tabularx}
\usepackage{graphicx}
\usepackage{caption}
\usepackage{pifont}  
\usepackage{listings}
\usepackage{xcolor}
\usepackage{hyperref}

\definecolor{highlightpink}{RGB}{255,0,180}  % 粉红色
\definecolor{codebg}{RGB}{245,245,245}       % 代码背景色
\definecolor{lightgray}{gray}{0.8}           % 浅灰色(替代gray!20)
\definecolor{commentcolor}{RGB}{0,128,0}     % 注释颜色(深绿色)
\definecolor{highlight1}{RGB}{255,50,50}     % 红色高亮
\definecolor{highlight2}{RGB}{0,100,200}     % 蓝色高亮
\definecolor{highlight3}{RGB}{150,0,150}     % 紫色高亮

% 设置超链接样式
\hypersetup{
    colorlinks=true,
    linkcolor=blue,
    filecolor=magenta,      
    urlcolor=blue,
    pdftitle={LeetCode 3289 题解},
    pdfauthor={Feijoa_Li}
}

% 设置 listings 样式
\lstset{
basicstyle=\ttfamily\small,
language=C++,
frame=single,
breaklines=true,
backgroundcolor=\color{codebg},
escapeinside=``,
commentstyle=\color{commentcolor},
keywordstyle=\color{blue},
stringstyle=\color{red},
rulecolor=\color{black},
framesep=5pt,
xleftmargin=10pt,
xrightmargin=10pt
}

\title{\href{https://leetcode.cn/problems/the-two-sneaky-numbers-of-digitville/}{LeetCode 3289. 数字小镇中的捣蛋鬼}}
\author{Feijoa\_Li}
\date{}

\begin{document}

\maketitle

\section{题目描述}
数字小镇 Digitville 中,存在一个数字列表 \texttt{nums},其中包含从 0 到 n - 1 的整数。每个数字本应只出现一次,然而,有两个顽皮的数字额外多出现了一次,使得列表变得比正常情况下更长。

为了恢复 Digitville 的和平,作为小镇中的名侦探,请你找出这两个顽皮的数字。

返回一个长度为 2 的数组,包含这两个数字(顺序任意)。

\section{解题思路}
本题可以使用哈希表(字典)来统计每个数字出现的次数。由于题目保证恰好包含两个重复的元素,我们只需要:
\begin{enumerate}
    \item 遍历整个数组,使用哈希表记录每个数字出现的次数
    \item 再次遍历哈希表,找出出现次数大于1的两个数字
    \item 将这两个数字放入结果数组中返回
\end{enumerate}



\section{代码实现}
\begin{lstlisting}[language=C++]
class Solution {
public:
    vector<int> getSneakyNumbers(vector<int>& nums) {
        map<int, int> mp;
        `\textcolor{highlight1}{// 统计每个数字出现的次数}`
        for (auto x : nums) mp[x]++;
        
        vector<int> res;
        `\textcolor{highlight1}{// 找出出现次数大于1的数字}`
        for (auto [x, cnt] : mp) {
            if (cnt > 1) res.push_back(x);
        }
        return res;
    }
};
\end{lstlisting}

\end{document}