\documentclass{ctexart}
\ctexset{
    section = {
        titleformat = \raggedright,
        name = {,、},
        number = \chinese{section}
    }
}

\usepackage[a4paper,top=2.54cm,bottom=2.54cm,left=3.18cm,right=3.18cm]{geometry}
\usepackage{eso-pic,graphicx}
\usepackage{amsmath,amsfonts,amssymb,amsthm}
\usepackage{tikz}
\usepackage{multirow}
\usepackage{array}
\usepackage{fancyhdr}
\usepackage{lastpage}
\usepackage{tabularx}
\usepackage{graphicx}
\usepackage{caption}
\usepackage{pifont}  
\usepackage{listings}
\usepackage{xcolor}
\usepackage{hyperref}

\definecolor{highlightpink}{RGB}{255,0,180}  % 粉红色
\definecolor{codebg}{RGB}{245,245,245}       % 代码背景色
\definecolor{lightgray}{gray}{0.8}           % 浅灰色(替代gray!20)
\definecolor{commentcolor}{RGB}{0,128,0}     % 注释颜色(深绿色)
\definecolor{highlight1}{RGB}{255,50,50}     % 红色高亮
\definecolor{highlight2}{RGB}{0,100,200}     % 蓝色高亮
\definecolor{highlight3}{RGB}{150,0,150}     % 紫色高亮

% 设置超链接样式
\hypersetup{
    colorlinks=true,
    linkcolor=blue,
    filecolor=magenta,      
    urlcolor=blue,
    pdftitle={LeetCode 3217 题解},
    pdfauthor={Feijoa_Li}
}

% 设置 listings 样式
\lstset{
basicstyle=\ttfamily\small,
language=C++,
frame=single,
breaklines=true,
backgroundcolor=\color{codebg},
escapeinside=``,
commentstyle=\color{commentcolor},
keywordstyle=\color{blue},
stringstyle=\color{red},
rulecolor=\color{black},
framesep=5pt,
xleftmargin=10pt,
xrightmargin=10pt
}

\title{\href{https://leetcode.cn/problems/delete-nodes-from-linked-list-present-in-array/}{LeetCode 3217. 从链表中移除在数组中存在的节点}}
\author{Feijoa\_Li}
\date{}

\begin{document}

\maketitle

\section{题目描述}
给定一个整数数组 \texttt{nums} 和一个链表的头节点 \texttt{head}。从链表中移除所有存在于 \texttt{nums} 中的节点后,返回修改后的链表的头节点。

\section{解题思路}
本题要求从链表中删除所有在给定数组 \texttt{nums} 中出现的节点。由于链表的删除操作需要知道前驱节点,因此我们需要维护前驱指针。

\textcolor{highlight1}{\textbf{核心思路:}}
\begin{itemize}
    \item \textcolor{highlight2}{预处理数组:} 将数组 \texttt{nums} 排序,以便使用二分查找快速判断节点值是否在数组中
    \item \textcolor{highlight2}{处理头节点:} 如果头节点的值在数组中,需要不断移动头指针,直到找到第一个不在数组中的节点
    \item \textcolor{highlight2}{遍历链表:} 使用两个指针 \texttt{cur} 和 \texttt{nxt} 遍历链表,
    当 \texttt{nxt} 节点的值在数组中时,跳过该节点(也就是将 $cur->next$ 指向 $nxt->next$).
\end{itemize}

\textcolor{highlight1}{\textbf{算法步骤:}}
\begin{enumerate}
    \item 对数组 \texttt{nums} 进行排序
    \item 定义二分查找函数判断值是否在数组中
    \item 处理头节点,确保头节点不在删除列表中
    \item 使用双指针遍历链表,删除需要移除的节点
\end{enumerate}
\newpage
\section{代码实现}
\begin{lstlisting}[language=C++]
sort(nums.begin(), nums.end());
auto ok = [&](int x) {
    auto it = lower_bound(nums.begin(), nums.end(), x);
    if (it != nums.end() && *it == x)
        return true;
    return false;
};

while (ok(head->val)) {
    head = head->next;
}

auto cur = head;
auto nxt = head->next;

while (nxt != nullptr) {
    if (ok(nxt->val)) {
        cur->next = nxt->next;
    } else {
        cur = cur->next;
    }
    nxt = cur->next;
}
\end{lstlisting}

\section{复杂度分析}
\begin{itemize}
    \item \textbf{时间复杂度:} $O(n \log n + m \log n)$,其中 $n$ 是数组 \texttt{nums} 的长度,$m$ 是链表的长度。排序需要 $O(n \log n)$,每个节点的二分查找需要 $O(\log n)$。
    \item \textbf{空间复杂度:} $O(1)$,除了排序使用的栈空间外,只使用了常数额外空间。
\end{itemize}

\end{document}