\documentclass{ctexart}
\ctexset{
    section = {
        titleformat = \raggedright,
        name = {,、},
        number = \chinese{section}
    }
}

\usepackage[a4paper,top=2.54cm,bottom=2.54cm,left=3.18cm,right=3.18cm]{geometry}
\usepackage{eso-pic,graphicx}
\usepackage{amsmath,amsfonts,amssymb,amsthm}
\usepackage{tikz}
\usepackage{multirow}
\usepackage{array}
\usepackage{fancyhdr}
\usepackage{lastpage}
\usepackage{tabularx}
\usepackage{graphicx}
\usepackage{caption}
\usepackage{pifont}  
\usepackage{listings}
\usepackage{xcolor}
\usepackage{hyperref}

\definecolor{highlightpink}{RGB}{255,0,180}  % 粉红色
\definecolor{codebg}{RGB}{245,245,245}       % 代码背景色
\definecolor{lightgray}{gray}{0.8}           % 浅灰色(替代gray!20)
\definecolor{commentcolor}{RGB}{0,128,0}     % 注释颜色(深绿色)
\definecolor{highlight1}{RGB}{255,50,50}     % 红色高亮
\definecolor{highlight2}{RGB}{0,100,200}     % 蓝色高亮
\definecolor{highlight3}{RGB}{150,0,150}     % 紫色高亮

% 设置超链接样式
\hypersetup{
    colorlinks=true,
    linkcolor=blue,
    filecolor=magenta,      
    urlcolor=blue,
    pdftitle={LeetCode 1526 题解},
    pdfauthor={Feijoa_Li}
}

% 设置 listings 样式
\lstset{
basicstyle=\ttfamily\small,
language=C++,
frame=single,
breaklines=true,
backgroundcolor=\color{codebg},
escapeinside=``,
commentstyle=\color{commentcolor},
keywordstyle=\color{blue},
stringstyle=\color{red},
rulecolor=\color{black},
framesep=5pt,
xleftmargin=10pt,
xrightmargin=10pt
}

\title{\href{https://leetcode.cn/problems/minimum-number-of-increments-on-subarrays-to-form-a-target-array/description/}{LeetCode 1526. 形成目标数组的子数组最少增加次数}}
\author{Feijoa\_Li}
\date{}

\begin{document}

\maketitle

\section{题目描述}
给定一个整数数组 \texttt{target} 和一个初始数组 \texttt{initial},\texttt{initial} 数组与 \texttt{target} 数组维度相同且初始所有元素为0。每次操作可以选择任意子数组,并将子数组中每个元素增加1。返回从 \texttt{initial} 得到 \texttt{target} 的最少操作次数。



\section{解题思路}
本题可以使用动态规划的思想解决。
\textcolor{highlight1}{$dp[i]$} 表示将前 $i+1$ 个元素
从目标值减少到 $0$ 所需的最少操作次数。
注意,从目标数组减少到 $0$ 的操作次数与从 $0$ 
增加到目标数组的操作次数相同。

\textcolor{highlight1}{\textbf{初始状态:}}
\[
dp[0] = target[0]
\]
\\
\textcolor{highlight1}{\textbf{状态转移方程:}}
对于 $i \geq 1$,考虑 $target[i]$ 与 $target[i-1]$ 的关系:
\begin{itemize}
\item \textcolor{highlight2}{如果 $target[i] \leq target[i-1]$},
则在减少前 $i$ 个元素的过程中,对于 $target[i-1]$ 
必然执行了 $target[i-1]$ 次操作(因为 $target[i-1] == 0$ )\par
因此我们大可将每一步包含着 $target[i-1]$ 的操作
再往右延伸一格,影响到 $target[i]$ 上.\par
所以此时:
\[
dp[i] = dp[i - 1]
\]
\\
\item \textcolor{highlight2}{如果 $target[i] > target[i-1]$}
我们依旧用上面的思考方式,但此时,$target[i-1]$ 的操作
无法使 $target[i]$ 减少到 $0$ ,
还需要额外增加 $target[i] - target[i-1]$ 次操作.\par
所以此时:
\[
dp[i] = dp[i - 1] + (target[i] - target[i-1])
\]
\\
\end{itemize}
因此,整合两个式子得出\textcolor{highlight3}{\textbf{状态转移方程}}为:
\[
\textcolor{highlight3}{dp[i] = dp[i-1] + \max(0, target[i] - target[i-1])}
\]

所以最终答案为 $dp[n-1]$,其中 $n$ 是数组长度。

由于 $dp[i]$ 只依赖于 $dp[i-1]$,我们可以使用一个变量来滚动计算,节省空间。

\section{代码实现}
\begin{lstlisting}[language=C++]
int cnt = target[0]; `\textcolor{highlight1}{// 初始化 dp[0]}`
for (int i = 1; i < target.size(); i++) {
    `\textcolor{highlight3}{// 状态转移: dp[i] = dp[i-1] + max(0, target[i] - target[i-1])}`
    cnt += max(0, target[i] - target[i - 1]);
}
return cnt;
\end{lstlisting}

\section{复杂度分析}
\begin{itemize}
    \item \textbf{时间复杂度:} $O(n)$,其中 $n$ 是数组长度。只需遍历数组一次。
    \item \textbf{空间复杂度:} $O(1)$,仅使用常数额外空间。
\end{itemize}

\end{document}