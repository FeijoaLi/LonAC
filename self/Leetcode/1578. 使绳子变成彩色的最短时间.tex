\documentclass{ctexart}
\ctexset{
    section = {
        titleformat = \raggedright,
        name = {,、},
        number = \chinese{section}
    }
}

\usepackage[a4paper,top=2.54cm,bottom=2.54cm,left=3.18cm,right=3.18cm]{geometry}
\usepackage{eso-pic,graphicx}
\usepackage{amsmath,amsfonts,amssymb,amsthm}
\usepackage{tikz}
\usepackage{multirow}
\usepackage{array}
\usepackage{fancyhdr}
\usepackage{lastpage}
\usepackage{tabularx}
\usepackage{graphicx}
\usepackage{caption}
\usepackage{pifont}  
\usepackage{listings}
\usepackage{xcolor}
\usepackage{hyperref}

\definecolor{highlightpink}{RGB}{255,0,180}  % 粉红色
\definecolor{codebg}{RGB}{245,245,245}       % 代码背景色
\definecolor{lightgray}{gray}{0.8}           % 浅灰色(替代gray!20)
\definecolor{commentcolor}{RGB}{0,128,0}     % 注释颜色(深绿色)
\definecolor{highlight1}{RGB}{255,50,50}     % 红色高亮
\definecolor{highlight2}{RGB}{0,100,200}     % 蓝色高亮
\definecolor{highlight3}{RGB}{150,0,150}     % 紫色高亮


\AddToShipoutPictureFG{%
  \AtPageCenter{%
    \begin{tikzpicture}[remember picture,overlay]
      \node[opacity=0.1] at (0,0) {\includegraphics[width=0.8\paperwidth,height=0.6\paperheight,keepaspectratio]{sy.png}};
    \end{tikzpicture}%
  }%
}


% 设置超链接样式
\hypersetup{
    colorlinks=true,
    linkcolor=blue,
    filecolor=magenta,      
    urlcolor=blue,
    pdftitle={LeetCode 1578 题解},
    pdfauthor={Feijoa_Li}
}



% 设置 listings 样式
\lstset{
basicstyle=\ttfamily\small,
language=C++,
frame=single,
breaklines=true,
backgroundcolor=\color{codebg},
escapeinside=``,
commentstyle=\color{commentcolor},
keywordstyle=\color{blue},
stringstyle=\color{red},
rulecolor=\color{black},
framesep=5pt,
xleftmargin=10pt,
xrightmargin=10pt
}

\title{\href{https://leetcode.cn/problems/minimum-time-to-make-rope-colorful/description/}{LeetCode 1578. 使绳子变成彩色的最短时间}}
\author{Feijoa\_Li}
\date{}

\begin{document}

\maketitle

\section{题目描述}
Alice 把 n 个气球排列在一根绳子上。给你一个下标从 0 开始的字符串 \texttt{colors},其中 \texttt{colors[i]} 是第 i 个气球的颜色。

Alice 想要把绳子装扮成\textbf{五颜六色的},且她不希望两个连续的气球涂着相同的颜色,所以她喊来 Bob 帮忙。Bob 可以从绳子上移除一些气球使绳子变成彩色。给你一个下标从 0 开始的整数数组 \texttt{neededTime},其中 \texttt{neededTime[i]} 是 Bob 从绳子上移除第 i 个气球需要的时间(以秒为单位)。

返回 Bob 使绳子变成彩色需要的\textbf{最少时间}。

\section{解题思路}
本题可以使用\textcolor{highlight1}{动态规划}的方法解决。

\textcolor{highlight1}{\textbf{状态定义:}}
定义 \texttt{dp[i][j]} 表示处理到第 i 个气球,且第 i 个气球的颜色为 j 时的最小花费时间。其中 j 的范围是 0-25(对应 26 个小写字母)和 26(表示移除该气球)。

\textcolor{highlight1}{\textbf{初始状态:}}
\begin{itemize}
    \item 第一个气球被移除:\texttt{dp[26] = neededTime[0]}
    \item 第一个气球保留:\texttt{dp[colors[0]-'a'] = 0}
    \item 其他状态初始化为无穷大
\end{itemize}

\textcolor{highlight1}{\textbf{状态转移:}}
对于第 i 个气球(i ≥ 1):
\begin{itemize}
    \item 如果当前气球被移除,那么前一个气球可以是任意状态,总时间增加 \texttt{neededTime[i]}
    \item 如果当前气球保留(颜色为 \texttt{col}),那么前一个气球不能是相同颜色,需要找到前一个气球所有非 \texttt{col} 颜色的最小花费
\end{itemize}

\textcolor{highlight3}{\textbf{状态转移方程}}为:
\[
\begin{cases}
\texttt{dp[j] += neededTime[i]} & \text{对于所有 j(移除当前气球)} \\
\texttt{dp[col] = min(dp[col], Min)} & \text{其中 Min 是前一个状态中非 col 颜色的最小值}
\end{cases}
\]

\section{代码实现}
\begin{lstlisting}[language=c++]
class Solution {
public: `\textcolor{highlight1}{// 二维数组dp}` 
const int inf = 1e9 + 7;
    int minCost(string colors, vector<int>& neededTime) {
        int n = neededTime.size();
        vector<vector<int>> dp(n,vector<int>(27,inf));
        dp[0][26] = neededTime[0];
        dp[0][colors[0] - 'a'] = 0;
        for(int i = 1 ; i < n ; i ++){
            int Min = inf;
            for(int j = 0 ; j <= 26 ; j ++) {
                if(j != colors[i] - 'a')Min = min(dp[i - 1][j] , Min);
                dp[i][j] = min(dp[i][j] , dp[i - 1][j] + neededTime[i]);
            }
            dp[i][colors[i] - 'a'] = min(dp[i][colors[i] - 'a'] , Min);
        }
        
        return ranges::min(dp[n - 1]);
    }
};
\end{lstlisting}
\begin{lstlisting}[language=C++]
class Solution {
public: `\textcolor{highlight1}{// 滚动数组优化dp}` 
    const int inf = 1e9 + 7; 
    int minCost(string colors, vector<int>& neededTime) {
        int n = neededTime.size();
        vector<int> dp(27, inf);
        `\textcolor{highlight1}{// 初始状态:第一个气球}` 
        dp[26] = neededTime[0];  `\textcolor{commentcolor}{// 移除第一个气球}`
        dp[colors[0] - 'a'] = 0; `\textcolor{commentcolor}{// 保留第一个气球}`
        
        for (int i = 1; i < n; i++) {
            int Min = inf;
            int col = colors[i] - 'a';
            `\textcolor{highlight3}{// 找到前一个状态中非当前颜色的最小花费}`
            for (int j = 0; j <= 26; j++) {
                if (j != col) Min = min(dp[j], Min);
                `\textcolor{highlight3}{// 移除当前气球,所有状态时间增加}`
                dp[j] += neededTime[i];
            }
            `\textcolor{highlight3}{// 保留当前气球,只能从前一个非当前颜色的状态转移}`
            dp[col] = min(dp[col], Min);
        }

        return ranges::min(dp);
    }
};
\end{lstlisting}

\section{复杂度分析}
\begin{itemize}
    \item \textbf{时间复杂度:} $O(27 \times n)$,其中 $n$ 是数组长度。对于每个气球需要遍历 27 个状态。
    \item \textbf{空间复杂度:} $O(27)$,只需要维护一个大小为 27 的状态数组。
\end{itemize}

\section{算法优化}
实际上,本题还有更优的贪心解法:对于每一段连续相同颜色的气球,我们保留其中移除时间最大的那个,移除其他的气球。这样可以达到最优解。
\begin{lstlisting}[language=c++]
class Solution {
public: `\textcolor{highlight1}{// 贪心}` 
    const int inf = 1e9 + 7;
    int minCost(string colors, vector<int>& neededTime) {
        int res = 0, col = ' ', bef = 0;
        for (int i = 0; i < neededTime.size(); i++) {
            if (col == colors[i]) {
                res += min(bef, neededTime[i]);
                bef = max(bef, neededTime[i]);
            } else {
                col = colors[i];
                bef = neededTime[i];
            }
        }
        return res;
    }
};
\end{lstlisting}
\end{document}